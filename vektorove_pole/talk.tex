\documentclass[12pt]{article}

\input ../talks.tex

\begin{document}


\section*{Uvod}

Dobrý den, v této přednášce si spojíme dovednosti získané v kapitolách věnovaných derivacícm a v kapitolách věnovaným lineární algebře. Derivaci jsme poznali jako veličinu, která umožní kvantifikovat, co se děje ve studovaném systému nebo materiálu z hlediska prostorových nebo časových změn veličin. To je
důležité, protože přímo toto je vyjadřovací jazyk přirozený pro popis přírodních dějů a technickýh úloh. Lineární algebra nám zase dává do rukou nástroj, jak pracovat s veličinami, které mají směr v rovině či prostoru. To je také důležité, protože přírodní materiály a útvary mají vnitřní strukuturu, která jim dodává jiné vlastnosti v různých směrech. To je skvělé pro případné technické využití, ale pro matematický popis to je noční můra. Můžeme ji odstranit spojením modelů založených na derivacích s maticovým aparátem. Tím vznikne velmi silný nástroj, umožňující popsat jednotným způsobem libovolné transportní děje, ať již transport energie nebo hmoty. O toto se budeme snažit v dnešní přednášce a naformulujeme si framework umožňující jednotným způsobem popsat vedení tepla, vedení vody v porézním materiálu jako je dřevo nebo půda, transport energie či vzduchu z atmosféře, transport vody v řečišti a řadu dalších transportních jevů.



\subsection{Transporní jevy}

Budeme se zabývat případem, kdy se stav studovaného materiálu či objektu dá vyjádřit nějakou vhodnoou veličinou. Tuto veličinu budeme nazývat stavová veličina. Například při studiu transportu vlhkostního pole v materiálu je vhodnou stavovou veličinou hustota množství vody v daném místě materiálu, tj. podíl množství vody v daném místě a objemu, ve kterém je toto množství vody soustředěno. Při transportu energie je možnou stavovou veličinou hustota vnitřní energie, ale protože je tato veličina obtížně měřitelná, používáme jinou veličinu, teplotu. Obě veličiny jsou ekvivalentní ve smyslu, že pro daný materiál jsou jedna druhé přímo úměrné. 

Stavová veličina se může v daném místě kumulovat a může se přenášet prostředím. Například se může zvyšovat vlhkost v daném místě a difuzí se mohou molekuly vody v materiálu přemisťovat. Podobně, dané místo se může při sledování transferu energie ohřívat a energie se může ve formě tepla předávat z jednoho místa v materiálu do jiného. Někdy stavová veličina může vznikat nebo zanikat. Tady si situaci zjednodušíme: abychom nemuseli samostatně uvažovat zdroje a spotřebiče, budeme spotřebiče uvažovat jako zdroje se zápornou vydatností. Podobně, abychom nemuseli samostatně uvažovat přítok a odtok, budeme přítok uvažovat jako záporně vzatý odtok. Potom má smysl bilance, vyjadřující, že přírůstek množství veličiny je součtem přírůstku ze zdrojů a přírůstku způsobeného tokem. Přírůstek množství je derivace podle času, zdroje jsou v definici materiálu a jediné co musíme do této bilance dodat, abychom měli použitelný vztah ve formě kvantitativního modelu, je veličina měřící přírůstek způsobený tokem. Tedy bilanci mezi přítokem a odtokem do daného místa. Musíme nejprve tedy kvantifikovat tok stavové veličiny daným místem a poté posoudit bilanci mezi přítokem a odtokem. 

\subsection{Konstitutivní zákony}

Nejprve se budeme věnovat tomu, jak vůbec vzniká tok, přenášející stavovou veličinu. Tento tok je iniciován hnacím podnětem, kterým je v naprosté většině případů právě nerovnoměrnost v prostorovém rozložení stavové veličiny. Například pokud je v jednom místě materiálu vyšší teplota, teče teplo z tohoto místa do místa chladnějšího. Pokud je v jednom místě vyšší vlhkost, dostávají se difuzí molekuly vody z míst s vysokou vlhkostí do míst s nižší vlhkostí.

Nerovnoměrnost v prostorovém rozložení charakterizuje gradient. V ustáleném stavu je pro široké rozmezí fyzikálních problémů závislost intenzity toku na gradientu lineární. A protože nulovému gradientu (nulovému stimulu) odpovídá nulový tok (nulová odezva), bude tato lineární funkce přímou úměrností. A protože musíme zohledňovat vektorový charakter podnětu i odezvy, bude konstanta úměrnosti mít maticový charakter. Jedině tak je totiž možné modelovat případy, kdy směr podnětu a odezvy není stejný.

Známými případy jsou Fickův zákon pro difuzi, kdy je difuzní tok úměrný poklesu koncentrace, tj. záporně vzatému gradientu koncentrace. Tento zákon využijeme například pči modelování sušení dřeva.

Dalším podobným příkladem je Darcyho zákon po tok podzemní vody, kdy je tok úměrný poklesu tlaku, tj. záporně vzatému gradientu tlaku. Protože tlak je obtížně měřitelná veličina, pro praktické použití používáme někdy jinou veličiny, například piezometrickou hladinu, tj. výšku, do které by vystoupala voda, kdyby se v daném místě vyvrtala studna.

Třetím zákonem, stejným ale pro jiný děj, je Fourierův zákon. Týká se vedení tepla a udává, že tok tepla je úměrný poklesu teploty, tj. záporně vzatému gradientu teploty.

Výše uvedené zákony byly odvozeny poprvé pro jednorozměrné případy, kdy konstanty úměrnosti byla čísla. Při studiu materiálů s jinými vlastnostmi v jiných směrech však je nutné použít moderní formulaci, kdy konstanty úměrnosti jsou matice. Tyto matice navíc splňují některé další požadavky, vyjadřující fyzikální relevantnost úlohy, a proto se jim v aplikacích říká tenzory a uvedené konstituční zákony se nazývají tenzorové.

Jak tedy vypadá takový tenzorový konstituční zákon po rozepsání do složek? Gradient je parciální derivace podle prostorových souřadnic. Ve trojrozměrném případě mají gradient i tok tři komponenty a úměrnost je v tenzorovém případě zprostředkována 3$\times$3 maticí. V nejobecnějším případě to dopadne tak jak vidíme ve videu a není další prostor pro zjednodušení. Snad kromě toho, že matice bývá v naprosté většině případů symetrická a neáme tedy devět ale jenom šest nezávislých prvků tenzoru, šest materiálových konstant. Pokud však zvolíme osy ve vlastních směrech materiálu, tj. ve směrech kdy podnět a odezva mají stejný tvar, je matice diagonální a i ve složkách se rovnice značně redukují. Na každé pravé straně zůstane jenom jeden člen. Matice má jenom tři materiálové konstanty. Pokud by navíc byl materiál izotropní, situace by se dále zjednodušila. V tomto případě by všechny tři konstanty byly stejné a měli bychom tedy jenom jednu materiálovou konstantu.

Pomocí matice (nebo tenzoru) charakterizující materiálové vlastnosti a záporně vzatého gradientu, charakterizujícího směr a intenzitu poklesu stavové veličiny jsme schopni podchytit tok stavové veličiny, tj. její přenos prostředím. Potřebujeme být schopni popsat, jestli tento tok zesiluje či zeslabuje, k tomu nám poslouží pojem divergence, představený níže. 

\subsection{Divergence}

V následujícím se budeme zajímat o to, jestli v daném bodě tok nabírá na intenzitě nebo slábne, tj. jestli množství stavové veličiny, které z daného bodu vyteče, je větší nebo menší v porovnání s množstvím, které za stejnou dobu přiteče. Úloha je snadná, pokud si uvědomíme, že tok v prostoru je možno rozložit na tři nezávislé komponenty se směru os. A potom si již tok ve směru osy $x$ můžeme představit jako tok v trubici vedoucí tímto směrem. Pokud se v této trubici intenzita proudění směrem doprava navyšuje, znamená to, že víc stavové veličiny odtéká než přitéká a tok nabírá na intenzitě. Jak intenzivně se navyšuje zjistíme, pokud určíme, jak $x$-ová komponenta toku roste ve směru osy $x$. A to není nic jiného, než derivace $x$-ové komponenty toku podle $x$. Podobně to provedeme s ostatními směry, tedy $y$-ovou komponentu toku derivujeme podle $y$ a $z$-ovou komponentu podle $z$ a všechny příspěvky sečteme. Dostaneme celkové navýšení toku v daném bodě, nazývané divergence vektorového toku. 

Pokud například při difusi vody ve dřevě je divergence toku $10\mu \mathrm g \mathrm {cm}^{-3} \mathrm {min}^{-1}$, znamená to, že v centimetru krychlovém je za minutu celková bilance mezi přítokem a odtokem vody rovna deset mikrogramů ve prospěch toku ven. Tedy, že v daném místě materiál vysychá, protože vody ubývá, nebo je v daném místě zdroj vody, případně kombinace obojího. 

Někdy jsme v situaci že víme, že proudění je stacionární. Tedy množství stavové veličiny nezávisí na čase a kladnost divergence může být způsobena jenom přítomností zdrojů. Zápornost naopak přítomností spotřebičů. Pokud je divergence nulová, nazývá se takové pole nezřídlové.

Divergence je lokální charakteristika a proto je poslední úvahu nutno chápat tak, že funguje, pokud se v tomto centimetru krychlovém vlastnosti pole nemění. Případně uvažovat menší objekt než krychličku centimetr krát centimetr krát centimetr a na jednotku objemu celkovou bilanci přepočítat. Podobně jako když okamžitou rychlost určujeme tak v metrech za sekundu tak, že počítáme průměrnou rychlost na velmi krátkém intervalu, kratším než jedna sekunda. V limitě dokonce na intervalu nulové délky.


\subsection{Rovnice kontinuity}

Při studiu transportních jevů je základním nástrojem rovnice udávající celkovou bilanci množství stavové veličiny. V libovolném bodě platí, že přírůstek stavové veličiny za jednotku času je dán množstvím, které v tomto bodě vygenerují za jednotku času zdroje a množstvím, které se v tomto bodě oddělí z toku protékajícího tímto místem a zůstane zde. Každou z těchto veličin umíme kvantifikovat. Přírůstek stavové veličiny za jednotku času je derivace veličiny podle času. Vydatnost zdrojů je dána povahou úlohy, označme ji například sigma. A množství, které se v daném místě oddělí od toku zeslabí tento tok. Bude se tedy jednat o záporně vzaté zesílení toku a to umíme kvantifikovat pomocí divergence toku.

Odvozený matematický model se nazývá rovnice kontinuity. Je to jakási celková bilance veličiny, nebo, chceme-li, zákon zachování stavové veličiny se započítáním zdrojů a spotřebičů.

Rovnice kontinuity napsaná pro konkrétní případ nemusí obsahovat všechny části. Například, pokud chybí derivace podle času a je nahrazena nulou, znamená to, že popisujeme stacionární děj, kdy hodnota stavové veličiny nezávisí na čase. To je například po dosažení rovnováhy. Někdy jsou přítomny pouze členy s derivacemi a chybní člen bez derivací, v našem označení $\sigma$. To znamená, že vydatnost zdrojů a spotřebičů je nulová a stavová  veličina nemůže vznikat ani zanikat. Taková rovnice se nazývá bezzdrojová.

Jednoduchým příkladem využití rovnice kontinuity je rovnice mělké vody. Používá se například pro modelování toku vody v korytě. Úloha je jednodimenzinální, popisujeme situaci podél koryta. Stavovou veličinou je množství vody v daném místě a toto množství meříme pomocí obsahu průřezu vodního toku v daném místě kolmo na tok. pokud neuvažujeme prosakování dnem nebo boční přítoky, množství vody v korytě se zachovává a rovnice je bezzdrojová. Divergence se v jedné dimenzi redukuje na derivaci podle prostorové proměnné a pokud množství vody označíme $A$ a tok $Q$, má rovnice kontinuity tvar uvedený na obrazovce. Tento tvar se nazývá rovnice mělké vody., ale používá se obecněji, například pro modelování vln tsunami. Pro stavený tvar koryta se dá tato rovnice ještě dále konkretizovat, to již je však případ od případu jiné.

Se speciálním případem rovnice kontinuity se většina středoškolských studentů seznámí ve fyzice při uvažování stacionárního proudění nestlačitelné tekutiny. V takovém případě se naše rovnice redukuje na středoškolský tvar, kdy je součin rychlosti a průřezu konstantní. Pokud však potřebujeme započítat nestacionárnost, není tento středoškolský tvar použitelný a musíme uvažovat obecnější formulaci zde uvedenou.

\subsection{Difuzní rovnice}

Seznámili jsme se s rovnicí kontinuity. Ta vyjadřuje, že rychlost, s jakou v daném místě roste množství stavové veličiny je dána součtem množství, které v daném místě vygenerují zdroje a množstvím, které se oddělí z toku a zůstane v tomto místě. 
V naprosté většině případů je tok stavové veličiny determinován prostorovým rozložením. V naprosté většině případů se příroda snaží hodnoty stavové veličiny vyrovnávat, například teplo teče z místa o vyšší teplotě do místa s nižší teplotou tak dlouho, dokud se teploty nevyrovnají. Tento děj je tím intenzivnější, čím vyšší je nesoulad mezi hodnotami, například čím vyšší je teplotní rozdíl. Toto jsme dokázali modelovat pomocí konstitučních zákonů, kdy je tok úměrný záporně vzatému gradientu stavové veličiny. Pokud spojíme konstituční zákon s rovnicí kontinuity, dostáváme rovnici, která se nazývá \textit{difuzní rovnice}. Zpravidla minus z konstitučního zákona dáme podle pravidla o derivaci konstantního násobku před divergenci a tím nám ze dvou záporných znamének vznikne plus a rovnice má tvar, který vidíte na obrazovce.

Význam členů je stejný jako v rovnici kontinuity: je zde člen popisující růst stavové veličiny v daném místě, další člen charakterizující vydatnost zdrojů a člen popisující, jak se mění tok stavové veličiny při transportu v prostředí.

Konkrétním příkladem je rovnice transportu energie, kdy se energie předává vedením tepla. Hustota energie v daném místě se měří obtížně, z fyziky však víme, že tato veličina je úměrná teplotě a proto difuzní rovnici neformulujeme pro hustotu energie, ale pro teplotu. Proto potřebujeme na levé straně ještě fyzikální konstanty, které přepočítají rychlost růstu teploty na rychlost růstu hustoty energie. Tyto konstanty dodá fyzika a jedná se o hustotu a měrnou tepelnou kapacitu. Na pravé straně chybí člen vyjadřující zdroje či spotřebiče, protože zpravidla neuvažujeme vznik či zánik energie. Rovnice umožňuje například modelovat, jak rychle se mění teplota ve vzorku dřeva při tepelné modifikaci dřeva. Pokud bychom pravou stranu nahradili nulou, dostali bychom rovnici, která definuje rozložení teploty při tepelném namáhání. Například rozložení teploty ve stěně při konstantních podmínkách a poté, co jsme systému poskytli dostatek času pro nastolení rovnováhy. V jednorozměrném případě tuto rovnici již známe z úvodní přednášky. Teď jsme se seznámili s její vícerozměrnou variantou. Dokonce máme i nástroje, jak do této bilance zařadit zdroje či spotřebiče. To pomůže při studiu městské oblasti vystavené žáru slunce a ochlazované díky evapostranspiraci stromů, nebo při studiu chování přehrady vylité betonem, kde beton při tuhnutí uvolňuje teplo.

Jiným příkladem transportního pohybu je difuze vody ve dřevě. Příslušná rovnice se používá například k modelování sušení dřeva. Stavovou veličinou je hmotnost vody v jednotkovém množství materiálu. Podobně jako u transferu energie, ani zde zpravidla neuvažujeme zdroje či spotřebiče. V kontextu některých věd se tato rovnice často nazývá II. Fickův zákon, protože příslušným konstitučním vztahem je I. Fickův zákon.

Jinou užitečnou konkretizací difuzní rovnice je rovnice podzemní vody. Zde se uvažuje voda prosakující půdou, jakýmsi podzemním kolektorem vyplněným hlínou, který se nazývá zvodeň. Stavovou veličinou je množství vody, ale protože se tato veličina podobně jako u rovnice vedení tepla špatně měří, nahrazujeme ji lépe měřitelnou veličinou. Tato lépe měřitelná veličina se nazývá piezometrická výška a podobně jako u rovnice vedení tepla potřebujeme konstantu, která nám změny této veličiny převede na změny stavové veličiny. To všechno mají hydrologové k dispozici a výsledkem je rovnice, nazývaná rovnice podzemní vody. Používá se k posouzení vydatnosti zdrojů vody, k ochraně před kontaminací nebo k nalezení co nejefektivnějšího postupu dekontaminace. 

\section{Rovnice vedení tepla ve 2D v různých podmínkách}

Rovnici kontinuity umíme vyřešit ručně jenom ve velmi speciálních případech, proto ji často řešíme numericky. Pro zadání této rovnice do řešičů a identifikaci konstant, které se mají během výpočtu použít, je často nutné rovnici umět zapsat v kartézských souřadnicích. Musíme proto rozepsat difuzní člen s divergencí toku do parciálních derivací podle jednotlivých souřadnic. Také je nutné být schopen spolehlivě identifikovat rovnice stacionární, nebo rovnice bezzdrojové. To si natrénujeme nyní. 

Nejprve difuzní člen. Budeme pracovat s ortotropním materiálem v souřadné soustavě, jejíž osy směrují ve vlastních směrech difuzní matice.  Potom je difuzní matice diagonální a obsahuje tři konstanty. Gradient je parciální derivace stavové veličiny podle jednotlivých proměnných. Po vynásobení záporným znaménkem máme záporně vzatý gradient, tedy vektorovou veličinu udávající směr a intenzitu poklesu stavové veličiny. Vynásobení difuzními koeficienty, prvky z hlavní diagonály difuzní matice, dostáváme tok stavové veličiny. Pokud jednotlivé komponenty toku zderivujeme podle odpovídajících souřadnic a všechny souřadnice sečteme, dostáváme divergenci toku. Jak vidno, v této divergenci figurují derivace derivací, tedy něco jako druhé derivace, ale mezi prvním a druhým derivováním je veličina násobena ještě difuzními koeficienty. Proto se tyto objekty nazývají kvaziderivace. 

Počet kvaziderivací determinuje dimenzi úlohy. Pro trojrozměrné úlohy jsou zde tři členy, tři kvaziderivace. Pro dvourozměrné úlohy nemáme proměnnou $z$ a proto jsou kvaziderivace jenom dvě. Podobně, pro jednorozměrné úlohy je kvaziderivace jenom jedna. V některých případech je možné kvaziderivace dále upravit. To je v případě konstantního difuzní ho koeficientu. Pokud je difuzní koeficient konstantní, je možno kvaziderivace zapsat s využitím pravidla pro derivaci konstantního násobku jako součin difuzního koeficientu a druhé derivace. Tím se úloha dále zjednoduší a je snáze numericky zpracovatelná. Zbývá si ujasnit, kdy je difuzní koeficient konstantní. To bude v případě, že nezávisí ani na poloze, ani na hodnotě stavové veličiny. Pokud je materiál homogenní, potom má ve všech místech stejné vlastnosti a difuzní koeficient nezávisí explicitně na poloze. Závislost na poloze by však mohla být zprostředkována stavovou veličinou, pokud by byl difuzní koeficient jiný pro jiné hodnoty stavové veličiny. Například některé materiály mění při změnách teplot své fyzikální vlastnosti a mění i svoji tepelnou vodivost. To nesmíme připustit, pokud chceme mít difuzní koeficient konstantní. Materiálová odezva, kdy je difuzní koeficient nezávislý na stavové veličině, přesně kopíruje lineární závislost mezi podnětem a materiálovou odezvou a proto se o takových materiálech říká, že mají lineární materiálovou odezvu, nebo stručněji, že jsou lineární. Shrnuto, pokud je materiál jednoduchý ve smyslu že je homogenní a má lineární materiálovou odezvu, je možné členy s kvaziderivacemi zjednodušit do tvaru s druhými derivacemi. Nezapomeňme, že musí být splněny dvě podmínky: lineární materiálová odezva a homogenní materiál.  Tedy difuzní koeficient stejný pro všechny hodnoty stavové veličiny a materiál se stejnými vlastnostmi ve všech místech. 

Kromě difuzního členu v obecné difuzní rovnici figuruje člen s derivací podle času. Ten umožňuje modelovat časový vývoj systému, umožňuje modelovat, jak se množství stavové veličiny v jednotlivých místech s časem mění. Pokud člen s derivací podle času v rovnici není, je derivace podle času nulová. Taková rovnice je jednodušší, protože neobsahuje čas, ale její použití je omezenější. Umožňuje popisovat jenom rovnice modelující materiál v podmínkách, kdy se situace nemění v čase, tj. je stacionární. To je typicky v případě, že materiál je v neměnných podmínkách a na uběhlo dostatečné množství času na to, aby se hodnoty stavové veličiny ustálily na rovnovážných hodnotách, které odpovídají tomuto stavu.

Pokud je v rovnici člen neobsahující ani derivace podle času ani derivace podle polohy, jedná se o člen, který charakterizuje působení zdrojů, pokud je příspěvek kladný, nebo spotřebičů, pokud je příspěvek záporný.

Pokud tedy studujeme difuzní rovnici, díváme se na to, jestli rovnice obsahuje derivaci podle času. Podle toho vidíme, jestli se jedná o nestacionární či stacionární rovnici. Podle toho vidíme, jestli je rovnice schopna modelovat časový vývoj, nebo zda dokáže podchytit pouze rovnovážný stav. Dále se díváme, jestli rovnice obsahuje člen kde není ani derivace podle času, ani derivace podle polohy. Pokud ano, je rovnice schopna podchytit přítomnost zdrojů či spotřebičů. Pokud ne, je rovnice sice jednodušší, ale zdroje či spotřebiče do úvah zachytit nedokážeme. Poslední na co se díváme jsou difuzní členy. Podle počtu prostorových souřadnic vidíme, jestli je úloha jedno, dvoj nebo trojdimenzionální. Podle toho, zda jsou difuzní členy ve formě s kvaziderivacemi, nebo ve formě s druhými derivacemi, poznáme, zda je rovnice použitelná pro modelování pouze homogenního materiálu s lineární materiálovou odezvou, nebo zda je možné rovnici použít i pro studium materiálu obecnějšího. 

\section{Závěr}

V této přednášce jsme spojili dva zdánlivě nesourodé světy. První s nich, Svět derivací. Svět, kde umíme měřit a vyhodnocovat rychlosti změn. Umíme pracovat s rychlostí změny v čase a určovat, zda množství měnící se veličiny v čase roste nebo klesá a počítat stav v budoucnu nebo v minulosti. Umíme pracovat s rychlostí změny v prostoru a určit, jak prudce sledovaná veličina klesá a kterým směrem. Díky aparátu lineární algebry tento směr můžeme přepočítat na tok stavové veličiny a navíc ještě to můžeme udělat takovým způsobem, že tento popis je co nejúspornější. Například u ortotropního materiálu nepotřebujeme uvažovat plnou matici, ale jenom její hlavní diagonálu, což například ve trojrozměrném případě redukuje počet materiálových konstant ze šesti na tři. Pokud máme k dispozici tok, můžeme se opět obrátit s důvěrou na diferenciální počet a sledovat, jestli tento tok o kousek dál je intenzivnější nebo slabší a z toho usuzovat zda a jak intenzivně slábne nebo sílí. A jakmile máme tento aparát nachystaný, můžeme jej použít k matematickému vyjádření toho, že rychlost s jakou se zvyšuje množství stavové veličiny v daném bodě je součtem rychlosti s jakou v daném bodě veličinu generují zdroje a rychlosti, s jakou se tato veličina odpojuje z toku protékajícího tímto místem. To potom není nic jiného než rovnice kontinuity, případně difuzní rovnice. Jedná se o rovnici popisující obecně transportní jevy. Díky své obecnosti v sobě zahrnuje vedení tepla, difuzi vody, proudění podzemní vody, proudění mělké vody, proudění větrných mas v atmosféře a mnoho dalšího. Je to jedna z rovnic, se kterými se při studiu technických úloh i popisu biologických objektů setkáváme nejčastěji.

\end{document}
