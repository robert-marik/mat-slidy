\documentclass[12pt]{article}

\input ../talks.tex

\begin{document}


\section*{Uvod}

Dobrý den, v této přednášce si spojíme dovednosti získané v kapitolách věnovaných derivacícm a v kapitolách věnovaným lineární algebře. Derivaci jsme poznali jako veličinu, která umožní kvantifikovat, co se děje ve studovaném systému nebo materiálu z hlediska prostorových nebo časových změn veličin. To je
důležité, protože přímo toto je vyjadřovací jazyk přirozený pro popis přírodních dějů a technickýh úloh. Lineární algebra nám zase dává do rukou nástroj, jak pracovat s veličinami, které mají směr v rovině či prostoru. To je také důležité, protože přírodní materiály a útvary mají vnitřní strukuturu, která jim dodává jiné vlastnosti v různých směrech. To je skvělé pro případné technické využití, ale pro matematický popis to je noční můra. Můžeme ji odstranit spojením modelů založených na derivacích s maticovým aparátem. Tím vznikne velmi silný nástroj, umožňující popsat jednotným způsobem libovolné transportní děje, ať již transport energie nebo hmoty. O toto se budeme snažit v dnešní přednášce a naformulujeme si framework umožňující jednotným způsobem popsat vedení tepla, vedení vody v porézním materiálu jako je dřevo nebo půda, transport energie či vzduchu z atmosféře, transport vody v řečišti a řadu dalších transportních jevů.



\subsection{Transporní jevy}

Budeme se zabývat případem, kdy se stav studovaného materiálu či objektu dá vyjádřit nějakou vhodnoou veličinou. Tuto veličinu budeme nazývat stavová veličina. Například při studiu transportu vlhkostního pole v materiálu je vhodnou stavovou veličinou hustota množství vody v daném místě materiálu, tj. podíl množství vody v daném místě a objemu, ve kterém je toto množství vody soustředěno. Při transportu energie je možnou stavovou veličinou hustota vnitřní energie, ale protože je tato veličina obtížně měřitelná, používáme jinou veličinu, teplotu. Obě veličiny jsou ekvivalentní ve smyslu, že pro daný materiál jsou jedna druhé přímo úměrné. 

Stavová veličina se může v daném místě kumulovat a může se přenášet prostředím. Například se může zvyšovat vlhkost v daném místě a difuzí se mohou molekuly vody v materiálu přemisťovat. Podobně, dané místo se může při sledování transferu energie ohřívat a energie se může ve formě tepla předávat z jednoho místa v materiálu do jiného. Někdy stavová veličina může vznikat nebo zanikat. Tady si situaci zjednodušíme: abychom nemuseli samostatně uvažovat zdroje a spotřebiče, budeme spotřebiče uvažovat jako zdroje se zápornou vydatností. Podobně, abychom nemuseli samostatně uvažovat přítok a odtok, budeme přítok uvažovat jako záporně vzatý odtok. Potom má smysl bilance, vyjadřující, že přírůstek množství veličiny je součtem přírůstku ze zdrojů a přírůstku způsobeného tokem. Přírůstek množství je derivace podle času, zdroje jsou v definici materiálu a jediné co musíme do této bilance dodat, abychom měli použitelný vztah ve formě kvantitativního modelu, je veličina měřící přírůstek způsobený tokem. Tedy bilanci mezi přítokem a odtokem do daného místa. Musíme nejprve tedy kvantifikovat tok stavové veličiny daným místem a poté posoudit bilanci mezi přítokem a odtokem. 

\subsection{Konstitutivní zákony}

Nejprve se budeme věnovat tomu, jak vůbec vzniká tok, přenášející stavovou veličinu. Tento tok je iniciován hnacím podnětem, kterým je v naprosté většině případů právě nerovnoměrnost v prostorovém rozložení stavové veličiny. Například pokud je v jednom místě materiálu vyšší teplota, teče teplo z tohoto místa do místa chladnějšího. Pokud je v jednom místě vyšší vlhkost, dostávají se difuzí molekuly vody z míst s vysokou vlhkostí do míst s nižší vlhkostí.

Nerovnoměrnost v prostorovém rozložení charakterizuje gradient. V ustáleném stavu je pro široké rozmezí fyzikálních problémů závislost intenzity toku na gradientu lineární. A protože nulovému gradientu (nulovému stimulu) odpovídá nulový tok (nulová odezva), bude tato lineární funkce přímou úměrností. A protože musíme zohledňovat vektorový charakter podnětu i odezvy, bude konstanta úměrnosti mít maticový charakter. Jedině tak je totiž možné modelovat případy, kdy směr podnětu a odezvy není stejný.

Známými případy jsou Fickův zákon pro difuzi, kdy je difuzní tok úměrný poklesu koncentrace, tj. záporně vzatému gradientu koncentrace. Tento zákon využijeme například pči modelování sušení dřeva.

Dalším podobným příkladem je Darcyho zákon po tok podzemní vody, kdy je tok úměrný poklesu tlaku, tj. záporně vzatému gradientu tlaku. Protože tlak je obtížně měřitelná veličina, pro praktické použití používáme někdy jinou veličiny, například piezometrickou hladinu, tj. výšku, do které by vystoupala voda, kdyby se v daném místě vyvrtala studna.

Třetím zákonem, stejným ale pro jiný děj, je Fourierův zákon. Týká se vedení tepla a udává, že tok tepla je úměrný poklesu teploty, tj. záporně vzatému gradientu teploty.

Výše uvedené zákony byly odvozeny poprvé pro jednorozměrné případy, kdy konstanty úměrnosti byla čísla. Při studiu materiálů s jinými vlastnostmi v jiných směrech však je nutné použít moderní formulaci, kdy konstanty úměrnosti jsou matice. Tyto matice navíc splňují některé další požadavky, vyjadřující fyzikální relevantnost úlohy, a proto se jim v aplikacích říká tenzory a uvedené konstituční zákony se nazývají tenzorové.

Jak tedy vypadá takový tenzorový konstituční zákon po rozepsání do složek? Gradient je parciální derivace podle prostorových souřadnic. Ve trojrozměrném případě mají gradient i tok tři komponenty a [úměrnost je v tenzorovém případě zprostředkována 3$\times$3 maticí. V nejobecnějším případě to dopadne tak jak vidíme ve videu a není další prostor pro zjednodušení. Snad kromě toho, že matice bývá v naprosté většině případů symetrická a neáme tedy devět ale jenom šest nezávislých prvků tenzoru, šest materiálových konstant. Pokud však zvolíme osy ve vlastních směrech materiálu, tj. ve směrech kdy podnět a odezva mají stejný tvar, je matice diagonální a i ve složkách se rovnice značně redukují. Na každé pravé straně zůstane jenom jeden člen. Matice má jenom tři materiálové konstanty. Pokud by navíc byl materiál izotropní, situace by se dále zjednodušila. V tomto případě by všechny tři konstanty byly stejné a měli bychom tedy jenom jednu materiálovou konstantu.

Pomocí matice (nebo tenzoru) charakterizující materiálové vlastnosti a záporně vzatého gradientu, charakterizujícího směr a intenzitu poklesu stavové veličiny jsme schopni podchytit tok stavové veličiny, tj. její přenos prostředím. Potřebujeme být schopni popsat, jestli tento tok zesiluje či zeslabuje, k tomu nám poslouží pojem divergence, představený níže. 

\subsection{Divergence}

V následujícím se budeme zajímat o to, jestli v daném bodě tok nabírá na intenzitě nebo slábne, tj. jestli množství stavové veličiny, které z daného bodu vyteče, je větší nebo menší v porovnání s množstvím, které za stejnou dobu přiteče. Úloha je snadná, pokud si uvědomíme, že tok v prostoru je možno rozložit na tři nezávislé komponenty se směru os. A potom si již tok ve směru osy $x$ můžeme představit jako tok v trubici vedoucí tímto směrem. Pokud se v této trubici intenzita proudění směrem doprava navyšuje, znamená to, že víc stavové veličiny odtéká než přitéká a tok nabírá na intenzitě. Jak intenzivně se navyšuje zjistíme, pokud určíme, jak $x$-ová komponenta toku roste ve směru osy $x$. A to není nic jiného, než derivace $x$-ové komponenty toku podle $x$. Podobně to provedeme s ostatními směry, tedy $y$-ovou komponentu toku derivujeme podle $y$ a $z$-ovou komponentu podle $z$ a všechny příspěvky sečteme. Dostaneme celkové navýšení toku v daném bodě, nazývané divergence vektorového toku. 

Pokud například při difusi vody ve dřevě je divergence toku $10\mu \mathrm g \mathrm {cm}^{-3} \mathrm {min}^{-1}$, znamená to, že v centimetru krychlovém je za minutu celková bilance mezi přítokem a odtokem vody rovna deset mikrogramů ve prospěch toku ven. Tedy, že v daném místě materiál vysychá, protože vody ubývá, nebo je v daném místě zdroj vody, případně kombinace obojího. 

Někdy jsme v situaci že víme, že proudění je stacionární. Tedy množství stavové veličiny nezávisí na čase a kladnost divergence může být způsobena jenom přítomností zdrojů. Zápornost naopak přítomností spotřebičů. Pokud je divergence nulová, nazývá se takové pole nezřídlové.

Divergence je lokální charakteristika a proto je poslední úvahu nutno chápat tak, že funguje, pokud se v tomto centimetru krychlovém vlastnosti pole nemění. Případně uvažovat menší objekt než krychličku centimetr krát centimetr krát centimetr a na jednotku objemu celkovou bilanci přepočítat. Podobně jako když okamžitou rychlost určujeme tak, že počítáme průměrnou rychlost na velmi krátkém intervalu, v limitě dokonce na intervalu nulové délky.




\end{document}
