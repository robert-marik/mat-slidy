\documentclass[12pt]{article}

\input ../talks.tex

\begin{document}


\section*{Inverzní matice}

V závěru přednášky o maticích jsme viděli, že soustavu rovnic (bylo to pro teplotu na tepelně vodivé desce) můžeme napsat jako maticovou rovnici $$AX=B.$$
U rovnice stejného typu ale s reálnými čísly, například $3x=7$, je krokem k vyřešení dělení číslem $3$. To u matic bohužel nemáme. Máme však něco, co je pro čísla v podstatě ekvivalentní: násobení převrácenou hodnotou.

Převrácená hodnota funguje tak, že součin čísla a jeho převrácené hodnoty je roven jedné. Jednička je význačná tím, že se jedná o neutrální prvek vzhledem k násobení. Analogickým neutrálním prvkem u maticového součinu je jednotková matice a poté již definice inverzní matice nepřekvapí: součin matice a matice inverzní je roven neutrálnímu prvku. Inverzní matice je tedy jakési zobecnění převrácené hodnoty čísla do světa matic. 

Nekomutativita maticového součinu si vynucuje některé zvláštnosti, na které nejsme u komutativních operací zvyklí. Například u čísel je převrácená hodnota součinu rovna součinu převrácených hodnot. U matic a inverzních matic musíme být opatrnější:  inverze součinu matic je součin inverzních matic, ale v \textit{opačném pořadí}. Toto je snadné si zapamatovat, pokud například uvažujeme matice zprostředkovávající vztah mezi dvěma konfiguracemi Rubikovy kostky. Násobení příslušnou permutační maticí zprostředkovává přechod na jinou konfiguraci, násobení inverzní maticí přechod zpět. A pokud otočím nejprve boční a poté horní stranou, pro návrat do původního stavu musíme nejprve vrátit horní stranu a teprve potom boční. Operace se opravdu vrací v opačném pořadí.

Hledání inverzních matic je komplikovaná a numericky špatně podmíněná úloha, v některých speciálních případech to však je snadné. Je například jednoduché najít inverzní matici k pootočení o nějaký úhel. Opravdu, stačí pootočit o stejný úhel na druhou stranu, tedy změnit znaménko, například místo $\theta$ použít $-\theta$. Tady po využití toho, že funkce konsinus je sudá a sinus lichá, tedy že změna znaménka argumentu nemá vliv na hodnotu kosinu a mění znaménko sinu, vidíme, že inverzní matice je stejná jako transponovaná matice. Málokdy máme štěstí, že to je tak snadné, ale u některých matic to tak opravdu je. A protože tato vlastnost je spojena s důležitou třídou matic, s maticemi, které reprezentují změnu souřadnic zachovávající kolmost, vysloužily si matice kde pojmy inverzní a transponovaná matice splývají vlastní název, ortogonální matice.

\section*{Matice přechodu}

Ukážeme si, že pomocí matic je možné přepočítávat souřadnice mezi jednotlivými souřadnými soustavami. Praktické využití je studium ortotropních materiálů v situaci, kdy orientace os zohleďňující geometrii tělesa není výhodná pro popis materiálových vlastností. Například dřevěný kvádr je vhodné studovat tak, že hrany kvádru jsou rovnoběžné se souřadnými osami. Materiálové vlastnosti dřeva jsou dány v anatomických směrech dřeva. Pokud tyto směry nejsou nejsou rovnoběžné s osami (kvádr je nařezaný našikmo), je potřeba mezi souřadnými soustavami přecházet. To se dá elegantně udělat pomocí maticového násobení a inverzní matice. 

Mějme v dvourozměrném prostoru dvě souřadné soustavy, navzájem pootočené. Při takovém pootočení je přepočet mezi souřadnicemi jedné a druhé souřadné soustavy dán maticí rotace. Opravdu. Stačí zkontrolovat body na osách v jednotkové vzdálenosti od počátku a pro všechny ostatní body poté platí transformační rovnice automaticky, díky lineárním kombinacím. Protože při násobení vektorem $(1,0)$ dostaneme první sloupec a při násobením vektorem $(0,1)$ druhý sloupec, stačí souřadnice modrého a červeného puntíku vyjádřit v soustavě $x$, $y$. To ale už umíme z kapitoly o vektorech. Modrý puntík má souřadnice dány funkcemi kosinus a sinus úhlu pootočení. Pro červený puntík je to podobné. Matice transformace mezi souřadnicemi je tedy přímo matice rotace. Vztah, ke kterému jsme došli, umožní převádět souřadnice z čárkované do nečárkované souřadné soustavy. Pro opačnou transformaci pochopitelně můžeme využít matici inverzní.

V inženýrských příručkách se často sinus úhlu otočení zkracuje velkým písmenem $S$ a kosinus velkým $C$. Transformační rovnice potom vypadá přehledněji, ale je to jenom jiný zápis téhož.

Když umíme převádět souřadnice bodů mezi dvěma souřadnými soustavami můžeme se podívat na to, jak vypadají zobrazení vyjádřená v různých soustavách. Předpokládejme, že zobrazení je dáno maticí $A$, tedy že obraz vektoru nebo bodu  $X$ je $AX$. Tedy $$Y=AX.$$ Pokud dosadíme transformační vztahy s maticí přechodu například $P$, je nutné pro vyjádření v čárkované soustavě souřadnic osamostatnit $Y'$. To uděláme tak, že zleva vynásobíme maticí inverzní a použijeme asociativní zákon. Vidíme, že v otočených souřadnicích nalezneme obraz bodu $X'$ vynásobením s maticí $P^{-1}AP$. V praxi toto provádíme v případech, kdy matice $P^{-1}AP$ je jednodušší než matice $A$. Často například můžeme dosáhnout toho, že matice $P^{-1}AP$ je matice diagonální, tedy má nuly mimo hlavní diagonálu. To dramaticky redukuje složitost řešení úloh s maticemi spojenými. Než se k tomuto však dostaneme čeká nás spousta práce a proto se v závěru povídání o transformacích zaměříme na transformaci matic.

Ještě malá poznámka k terminologii, vysvětlím slovo tenzor, což je matematicky poněkud obtížněji uchopitelný objekt, přesto často používaný v materiálovém inženýrství. Bude nám naštěstí stačit povrchní popis. K vyjádření některých materiálových vlastností nestačí jedno číslo, ale je nutné použít matici. Například  pokud tenzor reprezentuje vztah mezi spádem teploty a tokem tepla v materiálu, který má lepší vodivost v jednom směru a tedy tendenci usměrňovat tok tepla do tohoto směru. Matice je závislá na zvolené soustavě souřadnic, ale někdy je vhodné se na konkrétní soustavu souřadnic nevázat. Proto fyzika zavedla pojem tenzor. My budeme pracovat vždy v nějaké zvolené soustavě souřadnic a proto pro nás tenzory budou vlastně matice, které, aby byly fyzikálně relevantní, musí splňovat ještě něco dalšího, například být symetrické podle diagonály. 

\section{Praktická aplikace: transformace tenzoru}

Ilustrační úloha je z Požgajovy monografie Štruktúra a vlastnosti drevá a spočívá v tom, že abychom mohli popsat chování dřeva v reakci na vnější působící sílu, musíme znát silové působení v jednotlivých anatomických směrech dřeva. Na obrázcích je takzvaný objemový element materiálu, což je kousek materiálu ve tvaru krychle, na který umíme snadno zachytit působení sil. Na obrázku vlevo máme například materiál namáhaný tlakem ve svislém směru a podélný směr je vůči tomuto směru skloněný. Potřebujeme najít tenzor napětí. To bude matice, která má v hlavní diagonále normálové napětí ve směru souřadných os a mimo diagonálu smykové napětí. Pro náš případ máme napětí v ose $y$ rovno 10MPa a jinak nuly. Druhé číslo v hlavní diagonále bude 10MPa a jinak v matici budou jenom nuly. Tento stav potřebujeme rozložit do směru podél vláken a napříč, protože potom budeme moci použít příslušné moduly elasticity a určit, jaká je v každém směru změna tvaru. Uděláme to stejnými transformačními rovnicemi jako pro zobrazení. Násobíme tedy z jedné strany maticí rotace a z druhé strany maticí inverzní k matici rotace. Po výpočtu máme tah podél vláken 7.5MPa a tah napříč 2.5MPa. Nenulová je i mimodiagonální složka se smykovým namáháním. To znamená, že pokud bychom si do původního obrázku nakreslili čtvereček s hranami podél vláken, silové působení by způsobovalo deformaci pravých úhlů v tomto čtverečku. 

Stejná úloha se řeší například pokud chceme určit namáhání šikmého lepeného spoje. Je-li ve směru tyčky tah 10 MPa, je tenzor napětí dán maticí mající na příslušné pozici hodnotu 10MPa a nuly jinde. Po otočení o 30 stupňů vidíme ve výsledné matici normálové napětí, které působí kolmo na spoj a snaží se ho roztrhnout a tečné napětí podél spoje, které se snaží spoj natáhnout podélně jako celek. Zajímá nás normálové napětí a jeho porovnání s údajem od výrobce, kdy je garantováno neselhání spoje. Protože vychází nenulová i mimodiagonální složka, je v daném místě i smykové napětí. Klidně si to vyzkoušejte. Vezměte si pružný pásek, nakreslete pod úhlem 30 stupňů čáru jako spoj, na spoj nakreslete element ve tvaru čtverce a natáhněte. Měli byste pozorovat, že se čtvereček deformuje tak, že už nemá pravé úhly. 

V praxi, pokud je počítání denním chlebem nějakého výpočtáře, je obvyklé používat jiné postupy než maticový počet. Oblíbená je buď grafická metoda Mohrovy kružnice nebo přímo použití vzorců pro transformaci komponent tenzoru. V takovém případě je výhodnější napsat komponenty tenzoru jako složky vektoru a je možné rovnou odvodit vzorce, které jsou v každé strojařské učebnici nebo například v Požgajově monografii o dřevě. Ale všechny tyto metody jsou už využitím obecného postupu pro transformaci. 

Využitím krátkého výpočtu s derivacemi je dokonce možné ukázat, že derivace diagonálních složek jsou co do velikosti dvojnásobky mimodiagonálních složek. Proto jsou nulové či nenulové současně. To znamená, že extrémy diagonálních prvků jsou v situaci, kdy se mimodiagonální složky nulují. Tedy například tahové napětí je maximální ve směru osy, pokud osy zvolíme tak, aby ve zvolené soustavě souřadnic byl nulový smyk (čtverečky nakreslené na materiálu se při deformaci mohou deformovat na obdélníčky, ale zůstane zachován pravý úhel mezi hranami). Ostatně u šikmého lepeného spoje byly obě normálová napětí menší než 10 Mpa a větší než 0Mpa, což byly hodnoty, pro které se nuluje číslo stojící v tenzoru napětí mimo diagonálu.


\section{Role vlastních vektorů při transformaci matic}

Pokusme se hledat transformaci, která zařídí, že matice transformace nebo tenzor budou jednoduché. Nebo přesněji, diagonální. Pokud transformační vztah $$P^{-1}AP=D$$ vynásobíme zleva maticí $P$, dostaneme vztah $AP=PD$. Vpravo je součin matice $P$ a diagonální matice, To znamená, že matice vpravo vznikne tak, že se sloupce matice $P$ vynásobí prvky matice $D$ v daném pořadí. V tomto součinu můžeme chápat sloupce matice $P$ jako vzory, matici $A$ jako zobrazení a sloupce matice $PD$ jako obrazy. To ale znamená, že vektory ve sloupcích mají tu vlastnost, že se zobrazují na své násobky. Tedy jsou vlastními vektory a vlastní hodnoty jsou tyto násobky, tj. prvky z hlavní diagonály matice $D$.

Toto ukazuje důležitou vlastnost: pokud transformujeme zobrazení nebo tenzor do souřadné soustavy, ve které jsou vlastní směry ve směru souřadných os, je toto zobrazení nebo tenzor vyjádřeno maticí v diagonálním tvaru a v diagonále jsou vlastní čísla.

Pokud bychom vlastní směry měli v trojrozměrném prostoru všechny tři a znali vlastní čísla, dokázali bychom takovou souřadnou soustavu zavést ihned a diagonální matice z vlastních čísel by byla k dispozici také okamžitě. Otázka je, zda to není příliš optimistická vidina.

Že není nám ukazuje následující věta. Týká se symetrických matic, ale to jsou v materiálovém inženýrství v podstatě jediné zajímavé matice.  První věta říká, že symetrická 3 krát 3 matice má dostatečné množství vlastních čísel a druhá, že má v prakticky zajímavých případech i dostatečný počet vlastních směrů pro definici souřadnic prostoru. Tedy je vždy možné zvolit soustavu souřadnic takovou, že tenzor je diagonální. To značně redukuje složitost úlohy.

Z praktických důvodů preferujeme při volbě pořadí nebo směrů os takovou souřadnou soustavu, aby matice $P$ měla determinant roven jedné. Potom se totiž jedná o otočení a pravotočivá soustava zůstává pravotočivou. Pokud by byl determinant matice $P$ roven číslu $-1$, stačí otočit jednu z os nebo vyměnit jejich pořadí. 

Pořád jsme ještě nevyřešili otázku, jak ta vlastní čísla a vlastní vektory vlastně najít. K tomu nám pomůže pojem determinant představený dále.

\section{Determinant}

Abychom pochopili význam determinantu pro naše úvahy, je nejprve nutné si připomenout, že soustavu lineárních rovnic můžeme zapsat v maticovém tvaru $AX=B$, kde $A$ je matice a $X$ a $B$ jsou sloupcové vektor. Podívejme se nejprve na řešitelnost v případě, že by se jednalo o lineární rovnici v oboru reálných čísel. V takovém  případě je zásadní otázka, zda je v rovnici $ax=b$ s neznámou $x$ koeficient $a$ nulový či nenulový. Pokud je nenulový, je situace snadná, stačí vydělit rovnici koeficientem $a$ a dostáváme $x$ ve tvaru $x=\frac ba$. Nebo vynásobit inverzí k $a$, pokud bychom se chtěli vyhnout podílu, který u matic nemáme. Tedy $x=b a^{-1}.$ Horší je situace, pokud je $a$ nulové. Pokud je i $b$ nulové, rovnice platí pro libovolné $x$ a řešení je nekonečně mnoho. Pokud je $b$ nenulové, rovnice naopak neplatí nikdy a rovnice nemá řešení.

Podobné to je i se soustavami, ale obdoba nulovosti nenulovosti a nenulovosti koeficientu stojícího v součinu s neznámou hraje speciální číslo přiřazené čtvercové matici, determinant. Definice determinantu pro matice libovolného stupně je poměrně komplikovaná, ukazuje, jak se dá rozepsat determinant čtvercové matice libovolné velikosti pomocí determinantu matice o řádek a sloupec menší a postupným opakováním tohoto postupu je možné se dostat k determinantu matice řádu jedna, který je definován přímo. Nám tato obecná definice nebude příliš užitečná, stačí nám obecný koncept a výpočet determinantu matic řádu dva a tři a proto si ukážeme jenom pravidla pro tyto determinanty. Ještě však zmiňme, že matice mající nulový determinant se nazývají singulární a matice s nenulovým determinantem regulární.

Výpočet matice dva-krát-dva se realizuje křižovým pravidlem, kdy se vynásobí čísla v hlavní diagonále a od tohoto součinu se odečete součin čísel ve vedlejší diagonále.

Pro determinant matice tři-krát-tři je nutné sofistikovanjěší pravidlo, protože tento determinant se skládá ze šesti součinů. Příslušné vzorec pro výpočet se nazývá Sarrusovo pravidlo a mnemotechnická pomůcka je následující. První a druhá řádek opíšeme a násobíme v hlavní diagonále a ve dvou šikmých řadách pod ní. Tyto součiny sečteme. Poté vynásobíme vedlejší diagonálu a dvě řady pod ní a tyto součiny odečteme od předešlého.

Situace je extrémně jednoduchá, pokud je matice ve schodovitém tvaru, kdy každý řádek má na začátku vice nul než řádek předchozí. Potom stačí vynásobit prvky v hlavní diagonále a toto pravidlo platí pro determinant libovolného řádu.

Kriticky důležitá věta, kvůli které si determinant představujeme, je následující věta. Vyjadřuje to, co jsme si řekli v úvodu tohoto videa. Pokud je matice $A$ čtvercová, je existence inverzní matice $A^{-1}$ ekvivalentní nenulovosti determinantu $|A|$ a to je ekvivalentní tomu, že soustava rovnic $AX=B$ má jediné řešení. To je dále ekvivalentní tomu, že soustava s nulovými pravými stranami má pouze nulové řešení. Tato skutečnost je zásadní pro nalezení vlastních čísel a vlastních vektorů matice. Aby nenulový vektor $\vec q$ byl vlastním vektorem, musí existovat vlastní číslo $ \lambda$ takové, že součin matice $A$ s vektorem $\vec q$ je roven součinu čísla $\lambda$ s vektorem $\vec q$.


\end{document}
