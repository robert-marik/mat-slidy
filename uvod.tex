\documentclass[12pt]{article}

\input talks.tex

\begin{document}

Dobrý den, vítejte v předmětu Matematika. Moje jméno je Robert Mařík, jsem přednášejícím předmětu a jeden z cvičících a jsem velice rád, že vám můžu zprostředkovat krásy a užitečnost matematiky. Vím, že pokud budete chtít, budete tento aparát moci použít k tomu, abyste lépe pochopili předměty v rámci své odbornosti nebo dokonce posunuli hranice poznání. To není těžké uhodnout, stačí se podívat na poslední vývoj ve vašich oborech a tam se to matematikou a matematickým modelováním často jen hemží.

Jsou mezi námi krajináři, dřevaři, nábytkáři, dřevostavbaři, arboristé. Všechno to vnímám jako velmi důležité obory pečující o naše životní prostředí. Buď přímo nebo využíváním obnovitelných zdrojů jako je v našem případě dřevo. Ve všech těchto oborech je plánování nesmírně důležité a matematické modelování nám při tomto plánování dokáže zprostředkovat podklady pro správná rozhodnutí. Všechny obory na fakultě přímo či nepřímo pracují se dřevem. S materiálem, který je sice neuvěřitelně krásný, ale velmi komplikovaný, protože má svoji vnitřní strukturu a díky tomu v různých směrech různé vlastnosti. To je pro jakékoliv modelování výrazná komplikace, kterou se naučíme překonávat.

O nástroje pro modelování různými přístupy nám v předmětu Matematika půjde v hlavní míře. Nevnímejte proto prosím předmět středoškolsky jako cestu od zadání příkladu k dvakrát podrtženém výsledku. Vnímejte ji jako pomocníka, který vám dá superchopnost modelovat a předvídat co se bude dít a umožní vám lépe proniknout pod kůži oboru vašeho zájmu.

Výuka bude rozdělena na přednášky a cvičení, bude doprovázena domácími úkoly a na konci semestru ve zkouškovém období bude předmět ukončen písemnou zkouškou. Jedná se tedy o jiný přístup než jaký znáte ze střední školy. Zejména je kladena vysoká zodpovědnost na vás, abyste se učili průběžně i když ověření znalostí bude až v lednu.

Základem studia na vysoké škole je samostudium. Nečekejte, že se naučíte látku na hodinách, tak jednoduché to bohužel není. Přesto je účast na výuce nesmírně důležitá. Na přednáškách se dozvíte, co se máte učit. Ve cvičení si přednesenou látku osaháte na konkrétních situacích. V domácích úlohách si potom můžete sami ověřit, zda látce rozumíte. Kromě toho učiteli ukážete, že tomu rozumíte a jako vedlejší produkt učiteli také dokážete, že se předmětu věnujete soustavně a během semestru. Vzhledem k tomuto vás budu moci za domácí úlohy silně zvýhodnit u závěrečné zkoušky. Podle toho jak udete úspěšní si mnozí přinesete jistou dávku bodů ke zkoušce právě za domácí úkoly. Zejména slabším studentům nejistým v matematice tuto taktiku doporučuji. Zvýhodnění za domácí úkoly bude vlastně dvojí. Řešením domácích úloh se také budete připravovat na zkoušku, protože domácími úlohami jsem se snažil pokrýt celou látku předmětu a je proto přirozené, že zkouška bude sestavena z příkladů, které uvidíte v domácích úlohách.

Zkouška bude ve stylu openbook, tedy taková, že si můžete přinést literaturu. Dbejte na to, abyste měli dobře zorganizované zápisky a dokázali se v nich orientovat a vyhledat požadované informace. Více než na memorování snadno dohledatelných faktů se učte v informacích orientovat a pracovat s nimi.

COVID situace nám připomenula důležitost akademické pospolitosti. Nesnažte se s žádným předmětem, ani s matematikou, bojovat sami. Komunikujte. Pokud něco není jasné k organizaci předmětu nebo  k jeho náplni, ptejte se vyučujících. Pokud vám to jenom prostě zatím nejde, diskutujte o problémech s kolegy z vašeho oboru nebo z příbuzných oborů. Nebojte se při učení spolupracovat. Více hlav přece jenom více ví. Jediné, kdy se musíte spolehnout sami na sebe je práce, kterou budete odevzdávat jako své dílo. Ideální postup může být například prodiskutovat domácí úlohy v příjemném prostředí (hospůdka, arboretum, koleje, cokoliv jiného) se spolužáky a následovně své zadání samostatně vyřešit a odevzdat.

A co se naučíme? V prvních třech týdnech si osvojíme dovednost pracovat s pojmem rychlosti změny. Jednak najít rychlost s jakou se mění zadaná funkce, ale hlavně uměto použít tuto rychlost při stanovení matematického modelu nějakého fyzikálního procesu. To druhé je neuvěřitelně zásadní, protože nám to umožní povýšit poněkud omezenou středoškolskou matematiku a fyziku na vědy umožňující řešit opravdu problémy reálného světa. Jako motivačí obrázky tu máme měření vlhkosti dřeva elektrickou cestou, numerické výpočty na počítačích a problematiku optimlaizace trámu vyřezaného z kulatiny, ale aplikací bude mnohem více.

V dalších dvou týdnech si proces obrátíme a budeme ze známé rychlosti hledat měnící se veličinu. V šestém týdnu  oba přístupy určitým způsobem spojíme a budeme se věnovat řešení modelů, ve kterých rychlost změny hledané veličiny souvisí s hledanou veličinou. To je naprostá většina fyzikálních zákonů, takové modely se naučíme ze slovního zadání fomrulovat matematicky už v úvodních týdnech a v šestém týdnu si tuto znalost prohloubíme a naučíme se tyto modely i řešit.

V dalších třech týdnech se naučíme zohledňovat to, že v přírodních materiálech je často přítomna jakási struktura, která těmto materiálům vtiskne to čemu říkáme anizotroní charakter. To je skutečnost, že v různých směrech má materiál různé vlastnosti. Kdo štípal dřevo, což je asi každý, tak se o této anizotropii mechanických vlastností přesvědčil na vlastní kůži.

V dalším týdnu spojíe do jisté míry předchozí poznatkz a naučíme se neuvěřitelně užitečnou dovednost: naučíme se popisovat jakýkoliv transportní jev v materiálu. Potom jednotným přístupem budeme schopní modelovat tak rozdílné procesy jako je proudění podzemní vody, vedení tepla, difuze vody ve dřevě a sušení či vlhnutí dřeva, pohyb vzduchu v atmosféře a další.

V dalších týdnech si znalosti posuneme ještě o kousíček dále, ale tím bych nechtěl unavovat. Chtěl jsem hlavně říci, že všechno čemu se budeme věnovat má silnou motivaci spojenou s popisem fyzikálních polí v materiálu nebo s matematickou formulací fyzikálních zákonů. A strom, krajina, dřevo nebo voda, podzmení či povrchová, pochopitelně fungují podle těchto zákonitostí. Pokud máte k dispozici aparát pro modelování těchto zákonistí, dojdete při poznání toho jak to ve dřevě, krajině nebo ve stromě funguje mnohem dál.

Přeji vám na této cestě hodně úspěchů a kdyby byl s nečím problém, neváhejte se na mne obrátit. Buď naživo přímo při výuce nebo asynchronně v MS Teams. Věřím, že na dotazy budu schopen reagovat nejpozději do druhého dne.

Na shledanou a ať vás studium baví.

\end{document}

