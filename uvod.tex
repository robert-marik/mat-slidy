\documentclass[12pt]{article}

\input talks.tex

\begin{document}

Dobrý den, vítejte v předmětu Matematika. Moje jméno je Robert Mařík, jsem přednášejícím předmětu a jeden z cvičících a jsem velice rád, že vám můžu zprostředkovat krásy a užitečnost matematiky. Vím, že pokud budete chtít, budete tento aparát moci použít k tomu, abyste lépe pochopili předměty v rámci své odbornosti nebo dokonce posunuli hranice poznání. To není těžké uhodnout, stačí se podívat na poslední vývoj ve vašich oborech a tam se to matematikou a matematickým modelováním často jen hemží.

Jsou mezi námi krajináři, dřevaři, nábytkáři, dřevostavbaři, arboristé. Všechno to vnímám jako velmi důležité obory pečující o naše životní prostředí. Buď přímo nebo využíváním obnovitelných zdrojů jako je v našem případě dřevo. Ve všech těchto oborech je plánování nesmírně důležité a matematické modelování nám při tomto plánování dokáže zprostředkovat podklady pro správná rozhodnutí. Všechny obory na fakultě přímo či nepřímo pracují se dřevem. S materiálem, který je sice neuvěřitelně krásný, ale velmi komplikovaný, protože má svoji vnitřní strukturu a díky tomu v různých směrech různé vlastnosti. To je pro jakékoliv modelování výrazná komplikace, kterou se naučíme překonávat.

O nástroje pro modelování různými přístupy nám v předmětu Matematika půjde v hlavní míře. Nevnímejte proto prosím předmět středoškolsky jako cestu od zadání příkladu k dvakrát podtrženém výsledku. Vnímejte ji jako pomocníka, který vám dá superchopnost modelovat a předvídat co se bude dít a umožní vám lépe proniknout pod kůži oboru vašeho zájmu.

Výuka bude rozdělena na přednášky a cvičení, bude doprovázena domácími úkoly a na konci semestru ve zkouškovém období bude předmět ukončen písemnou zkouškou. Jedná se tedy o jiný přístup než jaký znáte ze střední školy. Zejména je kladena vysoká zodpovědnost na vás, abyste se učili průběžně i když ověření znalostí bude až v lednu.

Základem studia na vysoké škole je samostudium. Nečekejte, že se naučíte látku na hodinách, tak jednoduché to bohužel není. Přesto je účast na výuce nesmírně důležitá. Na přednáškách se dozvíte, co se máte učit. Ve cvičení si přednesenou látku osaháte na konkrétních situacích. V domácích úlohách si potom můžete sami ověřit, zda látce rozumíte. Kromě toho učiteli ukážete, že tomu rozumíte a jako vedlejší produkt učiteli také dokážete, že se předmětu věnujete soustavně a během semestru. Vzhledem k tomuto vás budu moci za domácí úlohy silně zvýhodnit u závěrečné zkoušky. Podle toho jak budete úspěšní si mnozí přinesete jistou dávku bodů ke zkoušce právě za domácí úkoly. Zejména slabším studentům nejistým v matematice tuto taktiku doporučuji. Zvýhodnění za domácí úkoly bude vlastně dvojí. Řešením domácích úloh se také budete připravovat na zkoušku, protože domácími úlohami jsem se snažil pokrýt celou látku předmětu a je proto přirozené, že zkouška bude sestavena z příkladů, které uvidíte v domácích úlohách.

Zkouška bude ve stylu openbook, tedy taková, že si můžete přinést literaturu. Dbejte na to, abyste měli dobře zorganizované zápisky a dokázali se v nich orientovat a vyhledat požadované informace. Více než na memorování snadno dohledatelných faktů se učte v informacích orientovat a pracovat s nimi.

COVID situace nám připomenula důležitost akademické pospolitosti. Nesnažte se s žádným předmětem, ani s matematikou, bojovat sami. Komunikujte. Pokud něco není jasné k organizaci předmětu nebo  k jeho náplni, ptejte se vyučujících. Pokud vám to jenom prostě zatím nejde, diskutujte o problémech s kolegy z vašeho oboru nebo z příbuzných oborů. Nebojte se při učení spolupracovat. Více hlav přece jenom více ví. Jediné, kdy se musíte spolehnout sami na sebe je práce, kterou budete odevzdávat jako své dílo. Ideální postup může být například prodiskutovat domácí úlohy v příjemném prostředí (hospůdka, arboretum, koleje, cokoliv jiného) se spolužáky a následovně své zadání samostatně vyřešit a odevzdat.

A co se naučíme? V prvních třech týdnech si osvojíme dovednost pracovat s pojmem rychlosti změny. Jednak najít rychlost s jakou se mění zadaná funkce, ale hlavně umět použít tuto rychlost při stanovení matematického modelu nějakého fyzikálního procesu. To druhé je neuvěřitelně zásadní, protože nám to umožní povýšit poněkud omezenou středoškolskou matematiku a fyziku na vědy umožňující řešit opravdu problémy reálného světa. Jako motivační obrázky tu máme měření vlhkosti dřeva elektrickou cestou, numerické výpočty na počítačích a problematiku optimalizace trámu vyřezaného z kulatiny, ale aplikací bude mnohem více.

V dalších dvou týdnech si proces obrátíme a budeme ze známé rychlosti hledat měnící se veličinu. V šestém týdnu  oba přístupy určitým způsobem spojíme a budeme se věnovat řešení modelů, ve kterých rychlost změny hledané veličiny souvisí s hledanou veličinou. To je naprostá většina fyzikálních zákonů, takové modely se naučíme ze slovního zadání formulovat matematicky už v úvodních týdnech a v šestém týdnu si tuto znalost prohloubíme a naučíme se tyto modely i řešit.

V dalších třech týdnech se naučíme zohledňovat to, že v přírodních materiálech je často přítomna jakási struktura, která těmto materiálům vtiskne to čemu říkáme anizotropní charakter. To je skutečnost, že v různých směrech má materiál různé vlastnosti. Kdo štípal dřevo, což je asi každý, tak se o této anizotropii mechanických vlastností přesvědčil na vlastní kůži.

V dalším týdnu spojíte do jisté míry předchozí poznatky a naučíme se neuvěřitelně užitečnou dovednost: naučíme se popisovat jakýkoliv transportní jev v materiálu. Potom jednotným přístupem budeme schopní modelovat tak rozdílné procesy jako je proudění podzemní vody, vedení tepla, difuze vody ve dřevě a sušení či vlhnutí dřeva, pohyb vzduchu v atmosféře a další.

V dalších týdnech si znalosti posuneme ještě o kousíček dále, ale tím bych nechtěl unavovat. Chtěl jsem hlavně říci, že všechno čemu se budeme věnovat má silnou motivaci spojenou s popisem fyzikálních polí v materiálu nebo s matematickou formulací fyzikálních zákonů. A strom, krajina, dřevo nebo voda, podzemní či povrchová, pochopitelně fungují podle těchto zákonitostí. Pokud máte k dispozici aparát pro modelování těchto zákonitostí, dojdete při poznání toho jak to ve dřevě, krajině nebo ve stromě funguje mnohem dál.

Jaké budou učební materiály? Náplň našeho předmětu, přednášek i cvičení, je na webu. Jsou zde i u každé přednášky a každého cvičení videolekce z běhu v zimním semestru 2020. Pro učení čehokoliv je však výhodnější nesoustředit se na sledování videí, ale pracovat s textem. Protože kvalita i rychlost práce s textem je naprosto někde jinde ve srovnání s dlouhým sledováním videí, ve kterých je komplikované vyhledávat informace. Abyste však nebyli odkázáni jenom na text nebo jenom na dlouhá videa, tak v roce 2021 budou přímo do přednášek přidávána stručnější videa, spíše jenom komentáře. To je, po pečlivé recenzi způsobu výuky v Česku i ve světě asi ten nejlepší přístup z hlediska studenta: psaný text na míru, doprovázený mikropřednáškami v délce cca 10 minut a doprovázený testovými úlohami. Tyto úlohy budou částečně přímo v textu a zejména v domácích úkolech. Ještě jednou: domácí úkoly jsou dobrovolné, ale doporučuji se jim věnovat, protože dokážou zásadním způsobem pomoci při závěrečné zkoušce. Bodovaný bude už i první domácí úkol, který je zaměřen kompletně na výuku práce se systémem.

Občas ve videu může posluchač nějaké slovo přeslechnout. Ať už zkomomelním při vyslovování, nebo špatným zafungováním systému pro potlačování šumu. To druhé zejména u slov a vět začínajících na nějakou šumějící slabiku. Tady bych využil toho, že většinu povídání mikropřednášek si předem připravuji na papír a proto není problém Vám text také nasdílet pod videem. Nebude to vždy stoprocentní přepis, protože někdy se ve videu odchýlím od předepsaného textu a ten potom už neopravuji, ale určitě to bude pomoc a jak se říká, lepší než drátem do oka. Někdy kvůli výslovnosti a porozumění vám možná bude připadat, že ve videu mluvím moc pomalu. Nezapomeňte, že video se dá na Youtube zrychli při přehrávání a navýšení rychlosti pomalého hovoru o 25 nebo i 50 procent většinou není na úkor srozumitelnosti. 

Přeji vám na cestě předmětem hodně úspěchů. Přeji vám, abyste pochopili, že matematika není nepřítel, ale pomocník. Přeji vám, abyste získali zejména základní představu, jak vám matematika pomůže dostat se dál v pochopení funkce dřeva, stromů, toků, půdy a dalších objektů zájmu oborů studovaných na LDF. Pokud nevíte, k čemu by vám matematika byla, nikdy její pomoc nepoužijete. Pokud víte, jak a s čím vám může pomoci, můžete na tuto pomoc dosáhnout sami, nebo si najít odborníka, který to uělá za vás. Proto ještě jednou: soustřeďte se na principy, metody, hlavní myšlenky, možnosti. Nesoustřeďte se na konkrétní počítání konkrétních příkladů. Ve cvičení dávejte pozor i v pasážích, kde se vysvětluje, proč se daný příklad počítá a čím je pro nás zajímavý. To vám dá do života mnohem více, než počtářské dovednosti při výpočtu derivací, integrálů, matic.

Tímto bych úvodní slovo ukončil a ještě důležitá výzva: kdyby byl s něčím problém, neváhejte se na mne obrátit. Buď naživo přímo při výuce nebo mailem či přes MS Teams. Věřím, že na dotazy budu schopen reagovat nejpozději do druhého dne. Pokud se neozvu ani druhý den, nejspíš dotaz spadl do SPAMu a zkuste jej zopakovat například jiným kanálem, ze školní emailové adresy a podobně.

Na shledanou a ať vás studium baví. Aspoň tak, jak mne bude bavit vás učit. To znamená hodně. :)

\end{document}

