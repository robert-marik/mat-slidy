\documentclass{article}

\usepackage[czech]{babel}
\usepackage[T1]{fontenc}
\usepackage[utf8]{inputenc}
\usepackage{amsmath,amsfonts,mathtools,multicol, url, booktabs}
\usepackage[a6paper, margin=20pt, landscape]{geometry}
\usepackage{graphicx, xcolor}
\usepackage{wrapfig, enumerate}
\parskip 10pt
\everymath{\displaystyle}
\parindent 0 pt

\def\zlomek{0.45}

\newcommand\obrazek[2][pixabay.com]{
  \clearpage
\begin{wrapfigure}{R}{\zlomek\linewidth}
  \begin{minipage}{1.0\linewidth}\parskip 0 pt
  \includegraphics[width=\linewidth]{#2}

  \vspace*{-10pt}
  \null\hfill{\color{gray}\footnotesize Zdroj: #1}

  \mezera
  \end{minipage}
\end{wrapfigure}
}

\let\oldtextbf\textbf
\def\textbf#1{%\newpage
  \oldtextbf{\color{red} #1}}

\def\mezera{\vspace*{-40pt}}

\def\stranka{\newpage}

\begin{document}

\rightskip 0 pt plus 1 em
\title{Cvičení Matematika LDF, bak. 1. ročník, 7.3.2019}
\date{}
\maketitle

\newpage

\def\tg{\mathop{\mathrm{tg}}}
\def\cotg{\mathop{\mathrm{cotg}}}
\def\arctg{\mathop{\mathrm{arctg}}}

\textbf{Základní lineární aproximace.}
Najděte lineární aproximace funkcí $\sin x$, $\cos x$ a~${(1+x)^n}$ v okolí nuly. Tím dokážete platnost následujících přibližných vzorců platných pro $x$ blízko nuly.
\begin{align*}
\sin x&\approx  x\\
\cos x&\approx  1\\
(1+x)^n&\approx  1+nx
\end{align*}
První dvě aproximace využijeme později pro odvození tvaru matice malých rotací, což je důležité při studiu deformace materiálů. Poslední můžeme využít například pro to, abychom z relativistického vzorce pro  celkovou energii extrahovali část závislou na rychlosti, tj. kinetickou energii.
\begin{multline*}
  E=mc^2=\frac{m_0}{\sqrt{1-\frac {v^2}{c^2}}}c^2=m_0c^2\left(1-\frac {v^2}{c^2}\right)^{-\frac 12}\\
  \approx m_0 c^2 \left(1+\left(-\frac 12\right)\left(-\frac{v^2}{c^2}\right)\right)=
m_0 c^2 \left(1+\frac 12\frac{v^2}{c^2}\right)
  =m_0c^2+\frac 12 m_0v^2
\end{multline*}
% Toto je klasický případ, kdy přesný těžkopádný vzorec ($E=mc^2$) nahrazujeme příjemnějším přibližným vyjádřením ($E_k=\frac 12 mv^2$).

\newpage

\textbf{Kinetika chemických reakcích pro malé koncentrace.} Rychlost mnoha chemických reakcí je dána vzorcem
\begin{equation}\label{1}
  f(x)=\frac {ax}{b+x},
\end{equation}
kde $x$ je koncentrace substrátu a $a$, $b$ jsou parametry (konstanty). Tento vzorec se nazývá kinetika  Michaelise a Mentenové. {\color{blue}Funkci můžeme přepsat do tvaru
\begin{equation*}
  f(x)=x\frac {a}{b+x}.
\end{equation*}
Pro malé koncentrace je $x$ ve jmenovateli zanedbatelné proti konstantě $b$ a proto
\begin{equation}\label{2}
  f(x)\approx x\frac {a}{b}.
\end{equation}
}Ve cvičení 28.2.2019 jsme vypočítali derivaci
\begin{equation*}
  \frac{\mathrm df}{\mathrm dx}=\frac{ab}{(b+x)^2}.
\end{equation*}
Použijte tento výpočet k lineární aproximaci funkce \eqref{1} a
ukažte, že vzorec \eqref{2} opravdu platí. Modře vyznačená úvaha,
která se zdá býti poněkud vágní, je tedy korektní. Po trošce
zkušeností se člověk, který to potřebuje často, naučí podobné vhodné
aproximace ``sypat z rukávu'' v okamžiku potřeby i bez diferenciálního
počtu. Vždy však je k~dispozici i přesný postup, založený na
derivacích.


\newpage
\textbf{Lokální extrémy bez slovního zadání.} V úlohách z praxe často
víme, že existuje optimální řešení a studovaná funkce má jediný bod s
nulovou derivací. Pokud studujeme funkci bez jakéhokoliv kontextu,
musíme posuzovat to, zda v daném bodě opravdu extrém je a
jaký. Nejlépe tak, že současně určíme i intervaly monotonie. Za
povšimnutí stojí, že při hledání bodů, kde jsou lokální extrémy,
vlastně ani nemusíme znát původní funkci. Stačí nám o ní informace
týkající se spojitosti a poté stačí znát derivaci. I~s~takovým
případem se v praxi setkáváme.


Najděte lokální extrémy a intervaly monotonie následujících funkcí. Spolu s funkcí je zadána i její derivace.

\begin{multicols}2

\def\priklad #1.#2. {$y=#1$, $y'=#2$}

\begin{enumerate}[(1)]
  \setlength\itemsep{20pt}
\item \priklad \frac x{(x+1)^2}.\frac{1-x}{(x+1)^3}.
\item \priklad \frac{x^2}{x+1}.\frac{x(x+2)}{(x+1)^2}.
  \item \priklad \frac{x^2}{x^2+1}.\frac{2x}{(x^2+1)^2}.
  \item \priklad (5-x)\sqrt x.\frac{1}{2\sqrt x}(5-3x).
  \item \priklad x^2 e^{-x}.-(x - 2)xe^{-x}.
  \item $y$ je spojitá na $\mathbb R\setminus\{2\}$,\\ $y'=\frac{(x^2+3)(x^2-3)}{2-x}$
  \end{enumerate}
\end{multicols}

\newpage

{
\def\zlomek{0.6}
\obrazek[vlastní]{krabicka.jpg}
\textbf{Krabička z papíru.}  V každém rohu papíru A4 vystřihneme
čtverec a zbylý papír podél stran poohýbáme nahoru, aby vznikla (až se
to slepí) krabička bez horního víka. Jak velké čtverce musíme
odstříhat, pokud chceme, aby výsledná
krabička měla co největší objem?


}

\obrazek{plot.jpg} \textbf{Plot ze tří stran pozemku.}  Chceme oplotit
pozemek obdélníkového tvaru, jehož jedna strana je rovná přirozená
hranice. Stavíme plot tedy jenom na zbylých třech stranách.
\begin{enumerate}[(1)]
\item Jaký tvar pozemku zvolit, pokud je dána délka pletiva a chceme mít plochu pozemku co největší?
\item Jaký tvar pozemku zvolit, pokud je dána plocha pozemku a chceme mít co nejmenší spotřebu pletiva?
\end{enumerate}
Než začnete řešit, tak si zkuste tipnout jestli optimální je čtverec
nebo obdélník. Pokud obdélník, tak zda podél přirozené hranice nebo
kolmo na ni. Také si zkuste tipnout, zda je řešení obou úloh stejné
(tj. stejný tvar obdélníku, například stejný poměr stran). Úlohy řešte
s co nejmenším množstvím parametrů. Uvažujte tedy, že máte jednu
délkovou jednotku pletiva v~prvním případě a že chcete oplotit pozemek
o jednotkovém obsahu v~případě druhém.


\def\mezera{\vspace*{-20pt}} \obrazek{pstruh.jpg} \textbf{Ryba
  migrující proti proudu.}  Ryba ve vodě vydává za časovou jednotku
energii úměrnou třetí mocnině rychlosti vzhledem k vodě. Pro překonání
určité vzdálenosti proti proudu o rychlosti $v$ je proto potřeba
energie $$E= k \frac 1x (x+v)^3,$$ kde $x$ je rychlost ryby vzhledem
ke břehu a $x+v$ rychlost vzhledem k vodě. Najděte pro rybu optimální
cestovní rychlost při migraci na dlouhé vzdálenosti, tj. rychlost, při
které je minimalizován nutný energetický výdaj.

Než začnete řešit, uvědomte si, že pokud měříme rychlosti v
jednotkách rychlosti vody v řece, platí $v=1$ a že vhodnou volbou
jednotek ve kterých měříme energii můžeme dosáhnout toho, že hledáme lokální minimum funkce
\begin{equation*}
  \frac{(x+1)^3}x.
\end{equation*}



(Podle Stewart, Day: Biocalculus. Calculus for the life siences.)

\newpage

Pozorování potvrdila, že migrující ryby ``znají'' řešení předchozího
příkladu a proto plavou proti proudu rychlostí o polovinu větší než
rychlost proudu. Vzhledem ke břehu je tedy jejich ``cestovní rychlost
proti proudu'' poloviční jako je rychlost proudu. Mimo jiné, v rychlé
vodě plavou rychle a v pomalejší pomaleji.

Příklad typu jaký jsme řešili u migrace ryb se ale ve skutečnosti
často objevuje naopak. Například následovně.

\begin{itemize}
\item Pozorujeme specifické chování ryb. Někdo si to toho nevšímá,
  někdo to bere jako fakt, ale někomu to vrtá hlavou. Proč to tak je?
  Asi si přirozeně minimalizují energii.
\item Jakou musíme učinit hypotézu aby tato hypotéza vedla k
  pozorovanému jevu? Jaká musí být souvislost energie s rychlostí, aby
  minimalizace energie vedla k tomu, co pozorujeme?
\item Po nalezení odpovědi na předchozí otázku je přirozené
  předpokládat, že jsme našli podstatu jevu. Tedy třeba, že energie je
  úměrná třetí mocnině rychlosti. V tomto smyslu matematika
  zviditelnila neviditelné.
\item Někdy je potřeba při konfrontaci s jinými pozorováními hypotézu
  poopravit, zpřesnit nebo bohužel zamítnout. To však je přirozené při
  poznávání světa.
\end{itemize}

\end{document}



% hughes - hallet, gleason, Lock, Applied calculus