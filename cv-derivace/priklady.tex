\documentclass{article}

\usepackage[czech]{babel}
\usepackage[T1]{fontenc}
\usepackage[utf8]{inputenc}
\usepackage{amsmath,amsfonts,mathtools,multicol, url, booktabs}
\usepackage[a6paper, margin=20pt, landscape]{geometry}
\usepackage{graphicx, xcolor}
\usepackage{wrapfig}
\parskip 10pt
\everymath{\displaystyle}
\parindent 0 pt

\newcommand\obrazek[2][pixabay.com]{
  \clearpage
\begin{wrapfigure}{R}{0.45\linewidth}
  \begin{minipage}{1.0\linewidth}\parskip 0 pt
  \includegraphics[width=\linewidth]{#2}

  \vspace*{-10pt}
  \null\hfill{\color{gray}\footnotesize Zdroj: #1}
    
  \end{minipage}
\end{wrapfigure}
}

\let\oldtextbf\textbf
\def\textbf#1{%\newpage
  \oldtextbf{\color{red} #1}}


\def\stranka{\newpage}

\begin{document}

\section{Cvičení 28.2.2019}

\def\tg{\mathop{\mathrm{tg}}}
\def\cotg{\mathop{\mathrm{cotg}}}
\def\arctg{\mathop{\mathrm{arctg}}}

\def\derivace#1;#2\par{$\displaystyle{(#1)'=\frac{\mathrm d}{\mathrm dx}(#1)=#2}$\par\smallskip}
Základní elementární funkce derivujeme pomocí následujících vzorců.

\hbox to \hsize{\begin{minipage}{0.5\linewidth}
\derivace c;0

\derivace x^n;n x^{n-1}

\derivace e^x;e^x

\derivace \sin x;\cos x

\derivace \cos x;-\sin x

\end{minipage}
\begin{minipage}{0.5\linewidth}
\derivace \tg x;\frac 1{\cos^2 x}

\derivace \ln x;\frac 1x

\derivace \arcsin x;\frac{1}{\sqrt {1-x^2}}

\derivace \arccos x;-\frac{1}{\sqrt {1-x^2}}

\derivace \arctg x;\frac 1{1+x^2}

\end{minipage}} Zde $c\in\mathbb R$ je konstanta a zbytek jsou vzorce,
které platí vždy, když je výraz napravo definovaný.

  Nechť $f$, $g$ jsou funkce a $c\in\mathbb R$ konstanta. Platí
  \begin{align*} 
    {\left[cf(x)\right]}'&=cf'(x),\\
    {\left[f(x)\pm g(x)\right]}'&=f'(x)\pm g'(x),\\
    {\left[f(x)g(x)\right]}'&=f(x)g'(x)+f'(x)g(x),\\
    {\Bigl[\frac{f(x)}{g(x)}\Bigr]}'&=\frac{f'(x)g(x)-g'(x)f(x)}{g^2(x)},\\
    \bigl[f(g(x))\bigr]'&=\frac{\mathrm df}{\mathrm dg}\frac{\mathrm dg}{\mathrm dx}=f'(g(x))g'(x)
  \end{align*}



\def\der #1.{$f(x)=#1$}
\textbf{Výpočet derivace.}
Určete derivace následujících funkcí, kde $a,b,\mu\in\mathbb{R}$.
\begin{multicols}3
\begin{enumerate}
\item \der x^3+2x^2-1.
\item \der \frac{x}{x^2+6}.
\item \der \frac{ax}{x+b}.
\item \der x\ln x.
\item \der \frac {2x^3}{x^2+1}.
\item \der 1-e^{bx}.
\item \der \frac{1}{\sqrt \pi}e^{ax^2}.
\item   \der \frac{a}{(\mu x+b)^2}.
\item \der (x^2-1)^4.
\item \der x\sqrt{x^2+1}.
\end{enumerate}
\end{multicols}

\def\der #1.#2.{$f(#1)=#2$}
\textbf{Výpočet derivace.}
Určete derivace následujících funkcí, kde $a,b\in\mathbb{R}$.
\begin{multicols}3
\begin{enumerate}
\item \der r. \frac 43\pi r^3. 
\item \der r. \pi r^2.
\item \der a. 6a^2.
\item \der v. \frac 12 mv^2.
\item \der r. \frac {a}{r^2}.  
\item \der y. ae^{by}.
\end{enumerate}
\end{multicols}


% Jiří Kameníček, https://brnensky.denik.cz
\obrazek{vate_pisky.jpg}
\textbf{Rychlost s jakou roste obsahu kruhu.}  Váté písky je bezlesý
pruh podél železniční trati nedaleko Bzence, kde je extrémní sucho
(Moravská Sahara). Rostou zde cenné rostliny a dobrovolníci, včetně
přednášejícího, zde od osmdesátých let pravidelně vysekávali náletové
borovice. V dřívějších dobách byly v pruhu podél železnice velmi časté
požáry kvůli provozu parních vlaků. Předpokládejme, že požár se ve
vysušené louce šíří ve tvaru kruhu. V určitém okamžiku je poloměr $50$
metrů a roste rychlostí $1.5$ metrů za minutu. Zapište zadání pomocí
derivací a určete jak rychle roste plocha zasažená ohněm.



\obrazek{room.jpg}
\textbf{Tepelná výměna.}  Teplota v místnosti kde se přestalo topit se mění
tepelnou výměnou s okolím. Rychlost, s jakou teplota v zimě klesá je
úměrná rozdílu teplot v místnosti a venku. Sestavte matematický model
popisující pokles teploty v této místnosti.


\obrazek{mladata.jpg}
\textbf{Model růstu úměrného velikosti chybějícího množství.}  Mnoho
živočichů roste tak, že mohou dorůstat jisté maximální délky a
rychlost jejich růstu je úměrná délce, která jim do této maximální
délky chybí (tj. kolik ještě musí do této maximální délky
dorůst). Sestavte matematický model popisující takovýto růst.


\obrazek{kontaminace.jpg}
\textbf{Kontaminace a čištění.}
%Hughes
Znečišťující látky se v kontaminované oblasti rozkládají rychlostí
$8\%$ za den. Kromě toho pracovníci odstraňují látky rychlostí $30$
galonů denně. Vyjádřete tento proces kvantitativně pomocí vhodného
modelu.

\obrazek{nemoc.jpg} \textbf{Hrubý model chřipkové epidemie.}  Počet
nemocných chřipkou $y$ v kraji o počtu obyvatel $M$ je úměrný současně
počtu nemocných a počtu zdravých jedinců. Sestavte model takového
šíření chřipky.


\obrazek{lov.jpg}
\textbf{Model drancování přírodních zdrojů.}
Při modelování růstu populace často pracujeme s populací žijící v prostředí s omezenou úživností. Často používáme model
$$\frac{\mathrm d x}{\mathrm dt}=rx\left(1-\frac xK\right),$$
kde $r$ a $K$ jsou parametry modelu (reálné konstanty).  Nakreslete
graf funkce $f(x)=rx\left(1-\frac xK\right)$ a ověřte, že pro velká
$x$ je $f(x)$ záporné a velikost populace proto klesá. Pokud populaci
lovíme konstantní rychlostí, sníží se pravá strana o konstantu, kterou
označíme $h$. Ukažte, že pro intenzivní lov bude pravá strana rovnice
pořád záporná a intenzivní lov tak způsobí vyhubení populace. Dá se
najít kritická hodnota lovu oddělující vyhynutí populace a její
trvalé přežívání?

\obrazek{kafe.jpg}
\textbf{Dokončování bakalářské práce.} Při dokončování bakalářské
práce roste spotřeba kávy při prosezených hodinách u počítače. Student
postupně zvyšuje svou denní spotřebu kávy rychlostí $0.5$ litru za
týden. Bohužel, situace na trhu se vyvinula tak, že roste i cena
jednoho litru kávy, a to rychlostí $0.20$ Kč za týden. Vyjádřete tato
pozorování pomocí derivací a určete, jak rychle rostou celkové výdaje
za kávu. Pokud nemáte všechny informace, rozhodněte, jaké další
informace jsou nutné pro to, aby úlohu bylo možno vyřešit.

\obrazek{kluzak.jpg}
\textbf{Rychlost klesání kluzáku.}
Teplota klesá s výškou o $2^\circ \mathrm C$ na kilometr. Pilot
kluzáku vidí, že teplota v okolí jeho kluzáku roste rychlostí
$10^{-4}{}^\circ \mathrm C/\mathrm{s}$. Vyjádřete tato pozorování pomocí
derivací a určete, jak rychle ztrácí kluzák výšku.

\obrazek{autobus.jpg} \textbf{Změna tlaku a lupání v uších.}
V dopravním prostředku, který se pohybuje do kopce nebo z kopce, se
mění tlak. Tím vznikne tlakový rozdíl mezi vnějším tlakem a tlakem ve
středním uchu. Vyrovnání tlaku při rychlé změně se projeví lupnutím
v uších.  Lupnutí tedy nastane, pokud je derivace
$\frac {\mathrm d p}{\mathrm dt}$ velká. (Velká v absolutní hodnotě,
tj. numericky hodně kladná nebo hodně záporná.)  Tuto veličinu však je
těžké měřit. Umíme měřit změnu nadmořské výšky $u$ a víme, jak se tlak
$p$ mění s nadmořskou výškou. Nechť například
$\frac{\mathrm dp}{\mathrm
  du}=-0.12\,\mathrm{g}\mathrm{cm}^{-2}\mathrm{m}^{-1}$ (údaj
meteorologů) a vezměme
$ \frac{\mathrm du}{\mathrm
  dt}=-3\,\mathrm{m}\,\mathrm{s}^{-1}$. Okomentujte význam toho, že
derivace jsou záporné a určete rychlost, s jakou rychlostí se mění
tlak vzduchu.

\stranka

\textbf{Derivace objemu koule a souvislost s povrchem. Derivace obsahu
  kružnice.}  Určete, jak souvisí rychlost změny poloměru kruhu
s rychlostí, jakou roste obsah kruhu. Měli byste vypozorovat, že
rychlost změny obsahu je rovna obvodu násobenému rychlostí změny
poloměru. Platí analogické tvrzení i pro obsah čtverce a délku strany
čtverce? Nakreslete si i obrázek pro vysvětlení tohoto
pozorování. Poté stejně prozkoumejte vztah mezi objemem koule,
povrchem koule a rychlostí změny poloměru koule. Zjistěte, jestli
anaogický vztah platí i pro objem krychle, povrch krychle a rychlost
změny délky hrany krychle.




\end{document}


% hughes - hallet, gleason, Lock, Applied calculus