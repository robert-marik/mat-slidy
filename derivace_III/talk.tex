\documentclass[12pt]{article}

\input ../talks.tex

\begin{document}

\section*{Úvod}

V této přednášce uzavřeme kapitolu věnovanou derivacím. Ukážeme si jak se dají derivace použit v optimalizačních úlohách, tedy jak se hledají maxima a minima funkce. Kromě toho si ukážeme několik málo dalších užitečných dovedností. Připomeneme si pojmy sudá a lichá funkce známé ze střední školy řekneme si, jak tyto znalosti využít v materiálovém inženýrství. Ukážeme si, že z fyzikálních jednotek veličin je možno odvodit fyzikální zákon popisující vztah mezi těmito veličinami. Toto se dá využít například k tomu, že namísto modelu v reálném měřítku zkoumáme model zmenšený. Příslušná poučka se nazývá Buckinghamův pí teorém. A na závěr se podíváme na vektorové funkce, které se používají pro práci s veličinami majícími směr a které budeme potřebovat později pro formulaci transportních dějů ne v jednorozměrném případě, jak jsme měli na úvodní přednášce v rovnici vedení tepla, ale ve dvou nebo trojrozměrném světě. 

\section*{Parita}

V praxi každého inženýra je situace, kdy studuje úlohu, která vykazuje určitou symetrii. Například pokud dáme prkno do pece pro tepelnou modifikaci, teplo prostupuje stejný způsobem do materiálu na horní i spodní straně. Při matematickém modelování tedy stačí vypočítat poměry v horní polovině vzorku a výsledky poté použít i pro polovinu spodní. Z matematického hlediska nám informaci o jisté symetrii funkce dává parita. Funkce které mají tu vlastnost, že vstupní data s opačným znaménkem generují stejné funkční hodnoty, se nazývají sudé. Jejich grafy jsou symetrické podél svislé osy. Funkce, kde se jiné znaménko ve vstupních datech projeví jiným znaménkem funkční hodnoty se nazývají liché a jejich grafy jsou symetrické podle počátku.

Typickým představitelem sudé funkce je sudá mocnina jako $x^2$ nebo $x^4$ a funkce kosinus. Typickým představitelem liché funkce je lichá mocnina jako $x^3$ nebo $x^5$ a funkce sinus.

Pokud máme symetrické prkno v peci pro teplotní modifikaci a umístíme souřadnou soustavu do středu prkna s osami ve směru hran, bude výsledná teplota funkcí všech tří prostorových proměnných a bude sudá v každé jednotlivé proměnné. Proto stačí vypočítat poměry v jedné osmině materiálu, například v prvním oktantu. Zbytek doplníme symetricky.

Sudost nebo lichost funkce nám může pomoci některé úkoly zjednodušit. V tomtom smyslu jsou sudé a liché funkce v jistém smyslu hezčí než obecné funkce. Je snadné ukázat, že libovolnou funkci je možné zapsat jako součet sudé a liché funkce. V tuto chvíli tato dovednost není nijak zásadní, ale zanedlouho podobný princip použijeme se složitějšími objekty, maticemi, abychom při deformaci materiálu rozlišili složku způsobující pootočení a složku způsobující změnu tvaru. Teď možná jenom stojí za zmínku, že rozdělením exponenciální funkce na sudou a lichou část dostaneme funkce, které jsou také často používané v technické praxi a proto mají speciální název: hyperbolický sinus a hyperbolický kosinus. 

\section*{Lokální extrémy}

Jednou z důležitých aplikací derivací je identifikace bodů, kde jsou funkční hodnoty nejvyšší jakých je možno dosáhnout nebo naopak nejnižší. Tomu se říká optimalizace. Snažíme se minimalizovat náklady nebo riziko selhání materiálu. O maximalizaci se snažíme u plánování zisku nebo u posuzování pevnosti součástky. Definici extrému vidíme na obrazovce. Rozlišujeme dva druhy extrémů, minimum a maximum. V obou případech srovnáváme funkční hodnoty v bodě s okolními hodnotami a proto se příslušné maximum nebo minimum nazývá lokální. Je-li například v bodě $x_0$ lokální maximum, znamená to, že v jistém okolí bodu $x_0$ neexistuje bod s vyšší funkční hodnotou než je funkční hodnota v bodě $x_0$. Všechny jsou menší nebo stejné. Analogicky s opačnou nerovností dostaneme definici lokálního minima.

Je zřejmé, že na lokální extrém není šance tam, kde je funkce rostoucí nebo klesající. Zejména tedy není možné mít lokální extrém tam, kde má funkce kladnou nebo zápornou derivaci. Pokud chceme mít body podezřelé z toho, že v nich extrém nastane, můžeme body s kladnou a zápornou derivací vyloučit. Zůstanou body kde je derivace nulová nebo neexistuje, což vyjadřuje Fermatova věta: V bodě kde má funkce lokální extrém derivace buď neexistuje nebo je tato derivace nulová. Toto je neuvěřitelně mocný princip, po eliminaci bodů s nenulovou derivací nám často zůstane jenom jeden kandidát, který je hledaným lokálním maximem nebo minimem.

Jako ukázku si představme úlohu vytesat z kulatiny trám obdélníkového průřezu tak, aby tuhost tohoto nosníku byla co nejvyšší, tedy aby se nosník co nejméně prohýbal. Z fyziky zjistíme, že je nutné dosáhnout toho, aby součin šířky a třetí mocniny výšky byl co největší. Vztah mezi výškou a šířkou určíme z Pythagorovy věty a z logické podmínky, že trám budeme řezat až do kraje. Je tedy nutno maximalizovat funkci
$$f(h)=h^3\sqrt{d^2-h^2},$$
kde $d$ je parametr.

Úloha je zvládnutelná tak jak je zadána a její řešení je možno najít v celé řadě učebnic. Nám však půjde ještě o jeden aspekt: naučíme se před řešením úlohu co nejvíce zjednodušit. To není nezbytné pro právě řešenou úlohu, je to však velice užitečná dovednost při studiu složitějších problémů.

Prvním trikem je speciální volba jednotek. Nebudeme měřit v centimetrech, nebudeme měřit v metrech, budeme měřit v jednotkách průměru kulatiny. To je obrovská výhoda, protože v takových jednotkách je průměr kulatiny $d$ roven jedné a můžeme pracovat s jedničkou namísto parametru $d$. Matematici často tuto pasáž odbudou frází "bez újmy na obecnosti položíme $d$ rovno jedné". Tím se značně zpřehlední výpočty a jediná nevýhoda je, že výsledek nevyjde v centimetrech, ale v násobcích průměru.

Druhým trikem je uvědomit si, že hledáme maximum kladné funkce s odmocninou. Pokud bychom funkci umocnili na druhou, odmocninu bychom odstranili. Protože funkce je kladná a druhá mocnina je na intervalu kladných čísel rostoucí, můžeme si tuto operaci dovolit. Z první přednášky víme, že rostoucí funkce nijak nenaruší platnost nerovnosti a proto nijak neovlivní podmínku z definice lokálního extrému. Vidíme tedy, že se úloha zjednodušila na hledání lokálního extrému funkce $$f(h)=h^6-h^8.$$ To je polynom, který je pro další zkoumání mnohem jednodušší než funkce která je součinem mocninné funkce a odmocniny z kvadratického dvojčlenu.

Nyní už je situace snadná. Vypočteme derivaci, Zjistíme kdy je derivace nulová a tím nám vyjde optimální výška nosníku. Z Pythagorovy věty poté najdeme i šířku. Výsledná šířka je polovina průměru a optimální nosník tedy bude mít jednu stranu rovnu poloměru kulatiny. 


\section*{Postačující podmínka pro lokální extrémy}

Pokud Fermatova věta nestačí k identifikaci lokálních extrémů nebo pokud není jasné zda se jedná o maximum či minimum, potřebujeme kriterium, které nám informace doplní. To je obsahem následující věty. Tato věta vyjadřuje, že pokud se spojitá funkce mění z rostoucí na klesající, je v bodech změny monotonie lokální maximum. Analogicky je možno formulovat podmínku pro lokální minimum. Monotonie je úzce svázána se znaménkem derivace a proto je žádoucí umět rozdělit definiční obor funkce na intervaly, kde funkce má kladnou derivaci a roste a na intervaly, kde má funkce derivaci zápornou a klesá. K tomu nám pomůže následující věta nazývaná Bolzanova věta. Tato věta vyjadřuje, že pokud se nějaký výraz spojitě mění z kladných hodnot na záporné, byl v některém okamžiku tento výraz roven nule.

Důsledkem je následující princip: Pokud studujeme nějaký výraz závisející na jedné proměnné a známe všechny nulové body a všechny body nespojitosti, je těmito body definiční obor rozdělen na několik intervalů. Bolzanova věta nám dává jistotu, že uvnitř těchto intervalů výraz zachovává znaménko. Stačí tedy kladnost nebo zápornost ověřit v jednom bodě a tím bude zajištěna stejná vlastnost, kladnost nebo zápornost, pro celý podinterval. 

Ukážeme si to na třech konkrétních příkladech, přičemž třetí bude z praxe.

V další ukázce se budeme věnovat problematice izolace trubek a válcovitých objektů. Ukážeme si, že i v příkladě který se zdá jasný protože to je něco ze života mohou být záludnosti. To co známe z praktického života je skutečnost, že čím víc izolace tím lépe, protože jsou menší tepelné ztráty. Ukážeme si ale, že přidávání izolace může naopak tepelné ztráty navyšovat, což ocení zejména obory, kde teplotní ztráty jsou naopak žádoucí. Typicky elektrotechnika, kdy potřebujeme, aby se dráty izolované kvůli ochraně před elektrickým proudem zbytečně moc nezahřívaly.

Uvažujme trubku o teplotě malé $T_1$ a poloměru $r$, která je obalena izolací o vnitřním poloměru $r$ a vnějším poloměru $R$. Na povrchu izolace je teplota $T_2$ a teplota okolí je $T_\infty$. Fyzika dodá vztah mezi geometrickými parametry, fyzikálními parametry a proudícím teplem. Jeden vztah charakterizuje podmínky v izolaci, kdy máme relaci vnitřním a vnějším poloměrem izolace, mezi teplotou na vnitřní a vnější straně a fyzikálními charakteristikami. Druhý vztah charakterizuje vyzařování z povrchu izolace do okolí. Spojením těchto vztahů můžeme vyloučit teplotu $T_2$ a získáme po trošce úprav vztah udávající přenesené teplo v závislosti na teplotě okolí, teplotě trubky a fyzikálních a geometrických charakteristikách.
Pokusme se najít extrém této funkce jako funkce proměnné $R$. Protože proměnná se vyskytuje ve vzorci pouze ve jmenovateli, stačí se věnovat tomuto jmenovateli. To znamená, že lokální maximum jmenovatele odpovídá lokálnímu minimu zlomku a naopak. Vypočítat derivaci není těžké, stačí si uvědomit, jaké funkce derivujeme: logaritmus a mocninnou funkci s exponentem $-1$. Pokud derivaci položíme rovnu nule, najdeme bod, kde je derivace nulová. Pro menší hodnoty $R$ je derivace záporná, pro vyšší hodnoty kladná. To znamená, že monotonie se mění z klesání na růst a funkce má lokální minimum. To znamená, že podíl má lokální maximum. Námi vypočtený lokální extrém je tedy hraniční hodnota. Pro menší poloměr $R$ než je tato hranice přidávání izolace naopak zvyšuje tepelné ztráty. To je důsledek toho, že zvyšování poloměru zvětšuje plochu ze které je vyzařováno teplo.

Pokud se vám zdá, že instalatéři s touto znalostí nepracují, nejste daleko od pravdy. Pro běžně používané materiály je hodnota kritického poloměru tak malá, že v praxi trubky rozvádějící horkou vodu jsou vždy silnější a uvedený efekt tedy nepozorujeme. Patrný může být například u dříve zmíněných vodičů elektrického proudu. 




\section*{Buckinghamův pí teorém}

Následující povídání nesouvisí přímo s derivacemi, vlastně pojem derivace ani nezmíníme. Ale souvisí s formulací modelů a ty jsou často přirozeně vyjadřovány právě pomocí derivací. Jedná se o princip, který říká, jak můžeme zvětšovat nebo zmenšovat jednotky při zachování stejného matematického modelu. Využití je dvojí. Jednak dokážeme v modelu najít veličiny, jejichž synchronizovaná modifikace neovlivní výsledek. Například když při pohybu autem zdvojnásobíme rychlost a čas zkrátíme na polovinu, ujetá dráha zůstane stejná. Díky tomu dokážeme také jenom pomocí fyzikálních jednotek odhadnout tvary závislostí mezi veličinami popisující systém. Další aplikací tohoto aparátu je schopnost identifikovat, jak se chování modelu změní při změně měřítka. Více se tomuto budeme věnovat později v kapitole věnované diferenciálním rovnicím, ale můžeme například zmínit, že v minulosti se mnoho vodohospodářských problémů řešilo simulací na zmenšeném modelu. Významným zmenšeným modelem byl model přehrady nad italským údolím Vajont, kde došlo k obrovské katastrofě a bylo nutno objasnit příčiny, aby se taková katastrofa neopakovala. Z nedávné doby můžeme zmínit zmenšeninu koncertního sálu připravovaného pro Brno, kdy měření akustických vlastností probíhalo na výstavišti na jaře 2021 na desetkrát zmenšeném přesném modelu.

Nebudeme uvažovat obecný Bugkinghamův teorém v plné obecnosti jako je v textu přednášky, ale zaměříme se na nejužitečnější speciální případ, kdy počet proměnných které charakterizují daný systém je o jedničku větší než počet nezávislých fyzikálních jednotek pomocí nichž se vyjadřují jednotky těchto veličin. V praxi to znamená, že existuje jediná možnost jak z veličin sestavit bezrozměrnou veličinu, veličinu, která nemá žádnou fyzikální jednotku. Správný fyzikální zákon je poté tvaru, kdy je tato bezrozměrná veličina rovna konstantě.

Představme si například těleso, které je jednoznačně dáno jedním délkovým parametrem $a$ a uvažujme objem $V$ takového tělesa. Je-li délka v metrech a objem v metrech krychlových, musíme délku umocnit na třetí, aby se jednotka vykrátila s jednotkou objemu. Proto vztah mezi objemem a rozměrem je $$\frac{V}{a^3}=k$$
kde $k$ je konstanta. Tedy objem je úměrný třetí mocnině rozměru s konstantou úměrnosti $k$. Vidíme, že jsme vůbec nemuseli udělat žádnou hypotézu o konkrétním tvaru tělesa. Může se jednat o kouli, krychli nebo kužel s fixovaným úhlem u vrcholu. V každém z těchto případů je pochopitelně jiná konstanta. Pro krychli a délku hrany je tato konstanta rovna jedné, pro poloměr a objem koule je rovna $\frac 43 \pi$. Po kužel s fixovaným úhlem u vrcholu je závislá na tomto úhlu, pro krychli a tělesovou uhlopříčku je možné ji vypočítat pomocí Pythagorovy věty a tak dále. Podobně je možné určit ve výše uvedených případech vztah mezi objemem a povrchem.

Další aplikací může být vztah mezi kvadratickým momentem průřezu nosníku a rozměry nosníku. Kvadratický moment je daný v jednotkách délky na čtvrtou a situace pro jednoparametrické průřezy jako je čtverec nebo kruh je podobná jako u objemu: Kvadratický moment je úměrný čtvrté mocnině průřezu. Odsud vidíme, že dvojnásobné zvětšení rozměrů vede k šestnáctinásobnému navýšení tuhosti.

Problematika nosníků obdélníkového průřezu není tak zřejmá, protože máme dvě věličiny s rozměrem délky (to jsou výška a šířka) a jednu veličinu s rozměrem délky na čtvrtou. Možností jak zařídit aby se jednotky vykrátily je nekonečně mnoho. Ani zde však situace není ztracená, pokud úvahou usoudíme, že na dva nosníky vedle sebe se síla rozloží tak, že na každý působí polovina, není těžké uhodnout, že kvadratický moment bude úměrný šířce a na výšku potom zůstává třetí mocnina. 


\section*{Vektorové funkce a gradient}

Pokud potřebujeme mít na výstupu zobrazení vektor, veličinu mající směr, zajímáme se většinou o jednotlivé komponenty tohoto vektoru. Formy zápisu jsou různé, ve 3D buď jako uspořádanou trojici nebo jako výraz udávající jak se vektor sestaví z jednotkových vektorů ve směru os. Například ve dvourozměrném světě je vektor $(2,-1)$ vektor, který z výchozího bodu směřuje o dvě jednotky doprava a jednu jednotku dolů.

S tímto požadavkem se setkáme pokud naše vektorová veličina charakterizuje, kterým směrem roste nebo klesá nějaká veličina jejíž nerovnoměrné rozložení příroda nevidí ráda. Například pokud v různých částech tělesa je různá teplota, dochází k toku tepla dokud nejsou teploty vyrovnány. Podobně funguje například obsah vody ve dřevě nebo množství vzduchu v atmosféře. V prvním případě vzniká difuzní tok z vlhčího místa do suššího, ve druhém případě vítr z místa o vysokém tlaku do místa s nízkým tlakem. Tento směr je snadné najít pomocí aparátu derivací: stačí uvažovat následující vektor sestavený z parciálních derivací. Ten ukazuje směr a intenzitu maximálního růstu veličiny $f$. Pokud je například v nějakém bodě tento vektor roven vektoru $(3,4)$ a má tedy podle Pythagorovy věty délku $5$ ve stupních celsia na metr, znamená to, že teplota roste směrem doprava nahoru pod úhlem o něco málo větším než 45 stupňů (tři dílky doprava a pět dílků nahoru) a na každém centimetru naroste o pět stupňů. V praxi nás zajímá záporně vzatý gradient, protože ten charakterizuje pokles a teplo se dává do pohybu ve směru klesající teploty. 

S gradientem se později setkáme například při popisu deformací tělesa až si budeme představovat aparát použitelný pro popis a modelování mechanického namáhání dřeva. Teď by mělo stačit omezit se na informaci, že pokud se po deformaci jednotlivé body materiálu posunou, přemístí v prostoru, je možné funkci popisující toto posunutí aproximovat lineární funkcí ovšem přítomnost více proměnných navyšuje složitost jednotlivých rovnic i jejich počet. V dalších týdnech si představíme nové objekty, matice, které umožní vrátit se k původnímu kompaktnímu tvaru, protože budeme schopni zapsat soustavu libovolného počtu lineárních rovnic v jedné maticové rovnici. Ale předtím se ještě budeme věnovat operacím opačným k derivacím. Integrálům. To jsou operace umožňující ze známe rychlosti změny najít buď časový průběh sledované veličiny nebo změnu za časový interval. Budu se na vás s tímto těšit příští týden. 


\end{document}
