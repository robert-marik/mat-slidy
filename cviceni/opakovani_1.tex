\documentclass{article}

\usepackage[czech]{babel}
\usepackage[T1]{fontenc}
\usepackage[utf8]{inputenc}
\usepackage{amsmath,amsfonts,mathtools,multicol, url, booktabs}
\usepackage[a6paper, margin=20pt, landscape]{geometry}
\usepackage{graphicx, xcolor}
\usepackage{wrapfig, enumerate}
\parskip 10pt
\everymath{\displaystyle}
\parindent 0 pt

\def\zlomek{0.45}

\let\rho\varrho

\def\nic{}


\newcommand\obrazek[2][pixabay.com]{
  \clearpage
  \def\test{#1}
\begin{wrapfigure}{R}{\zlomek\linewidth}
  \begin{minipage}{1.0\linewidth}\parskip 0 pt
  \includegraphics[width=\linewidth]{#2}

  \vspace*{-10pt}
  \ifx\test\nic\else
  \null\hfill{\color{gray}\footnotesize Zdroj: #1}
  \fi

  \mezera
  \end{minipage}
\end{wrapfigure}
}

\let\oldtextbf\textbf
\def\textbf#1{%\newpage
  \oldtextbf{\color{red} #1}}

\def\mezera{\vspace*{10pt}}

\def\stranka{\newpage}

\begin{document}

\rightskip 0 pt plus 1 em
\title{Cvičení Matematika LDF, bak. 1. ročník}
\date{4. dubna 2019}
\maketitle

Příklady, které se budou ve cvičení přeskakovat si projděte
samostatně. Řešení budou zveřejněna na webu předmětu \url{http://user.mendelu.cz/marik/mt}.

\newpage


\def\tg{\mathop{\mathrm{tg}}}
\def\cotg{\mathop{\mathrm{cotg}}}
\def\arctg{\mathop{\mathrm{arctg}}}


\newpage

\def\mezera{\vspace*{-20pt}}


\obrazek[Wikipedia]{vrt.jpg}

\textbf{Střední hodnota gravitační síly}
Gravitační síla od koule je stejná, jako by byla hmotnost soustředěna
v centru. Podle Newtonova gravitačního zákona je proto gravitační síla
nepřímo úměrná druhé mocnině vzdálenosti od středu.
Uvnitř koule je gravitační síla zase přímo úměrná vzdálenosti. Ve vzdálenosti $r$ od středu Země je tedy gravitační síla dána vzorcem
$$F=
\begin{cases}
    \frac{\alpha}{r^2} & r>R,\\
  \beta r & r\leq R.
\end{cases}
$$

\vspace*{-10pt}
\begin{itemize}\itemsep 0 pt
\item Určete hodnotu konstanty $\beta$ tak, aby tato funkce byla
  spojitá na $(0,\infty)$. 
\item Nejhlubší vrt je hluboký přibližně $12\,\mathrm{km}$ (Kolský superhluboký vrt) na obrázku. Určete
střední hodnotu $F$ na intervalu od povrchu Země do hloubky
12 kilometrů.  
\item Geostacionární družice létají přibližně 36 kilometrů nad
Zemí. Určete střední hodnotu $F$ na intervalu od povrchu
Země do výšky 36 kilometrů nad Zemí.
\item Porovnějte přibližnou změnu velikosti $F$, pokud se z povrchu Země ponoříme o malou hodnotu pod povrch a pokud se vzdálíme o stejnou hondotu nad povrch Země.
\end{itemize}

\newpage

\def\mezera{\vspace*{-20pt}}


\obrazek{horizont.jpg}



\textbf{Vzdálenost k horizontu}.
Vzdálenost k horizontu pro pozorovatele ve výšce $h$ je dána funkcí $H=\sqrt {2Rh},$ kde $R$ je poloměr Země (\url{https://aty.sdsu.edu/explain/atmos_refr/horizon.html}). S dosazením hodnot $$H=3.57\sqrt{h},$$ kde $h$ je v metrech a $H$ v kilometrech. Určete hodnotu této derivace $\frac{\mathrm d H}{\mathrm dh}$ pro $h=5\,\mathrm{m}$ v bodě $h=2\mathrm m$ (včetně jednotky) a slovní interpretaci této derivace.


\textbf{Výška rozhledny}. Mr. X chce postavit rozhlednu. Bez
dotací a proto se bude vybírat vstupné a rozhledna musí generovat
zisk.  Cena za postavení a údržbu rozhledny po celou dobu životnosti
bude úměrná výšce rozhledny.  Tržby ze vstupného za celou dobou
životnosti odhadněme, že budou úměrné vzdálenosti k horizontu (jak
daleko je vidět z rozhledny). Tato vzdálenost je úměrná odmocnině z
výšky rozhledny. Zjistěte, zda existuje nějaká optimální výška, kdy
zisk (rozdíl mezi tržbou a náklady) je nevyšší. Zanedbejte jevy, které
by v praxi úlohu také ovlivnily, jako je inflace, úročení peněz na
účtu apod.
% https://aty.sdsu.edu/explain/atmos_refr/horizon.html


\end{document}

\newpage



\textbf{Růst populace a jejich přežívání.}  Populace živočišného druhu
činí 5600 jedinců a tato populace roste rychlostí
$$R(t)=720 e^{0.1t}$$ jedinců za rok. (V tomto čísle je zahrnuta
přirozená natalita, mortalita a povolený lov.) Vlivem znečištění
životního prostředí se však jedinci dožívají kratšího věku, než je
zahrnuto v popsaném modelu. Zlomek populace, který přežije po době $t$
je $$S(t)=e^{-0.2t}.$$ Odhadněte počet živočichů za 10 let a
odhadněte, jaký by tento počet byl, kdyby k~ žádnému znečištění
nedocházelo, tj. kdyby bylo $S(t)=1$.

Napište jenom příslušné integrály a okomentujte, jakými metodami
bychom je počítali. Vlastní výpočet provádět nemusíte.

\textit{(Podle J. Stewart, T. Day: Biocalculus,  Calculus for Life Sciences.)}
% var('t')
% S(t)=exp(-0.2*t)
% print 5600+integrate(720*exp(0.1*t),(t,0,10))
% print 5600*S(10)+integrate(720*exp(0.1*t)*S(10-t),(t,0,10))


\newpage

\obrazek[https://slp.czu.cz]{strom.jpg}


\textbf{Mezinárodní den lesů.} Při obnově lesů je nutné velké množství
sadebního materiálu. Kromě školek hrají při obnově lesa důležitou roli
rodičovské stromy. Plošná hustota semen (například v počtu semen na
metr čtvereční) ve vzdálenosti $r$ od stromu je dána
funkcí $$D(r)=D_0 e^{-r^2/a^2}.$$ Pro vhodnou volbu jednotek dosáhneme
toho, že platí $a=1$. Pracujme proto s funkcí
$$D(r)=D_0 e^{-r^2}.$$ Určete množství semen uvnitř kruhu o
poloměru $R$.

Napište jenom příslušný integrál a okomentujte, jakou metodou bychom
ho počítali. Vlastní výpočet provádět nemusíte.

\textit{(Volně přeformulováno podle L. Edestein--Keshet: Differential calculus
for the life sciences. Strom na obrázku je rodičovský strom ekotypu
Posázavského smrku ztepilého. Slouží k záchraně genových zdrojů
lesních dřevin.)}

\newpage

\def\mezera{\vspace*{10pt}}

\obrazek{molo.jpg}

\textbf{Hmotnost dřeva s proměnnou vlhkostí.} Součástí mola je dřevěný
svislý metrový trám konstantního průřezu.  Blízkost hladiny, vlhkost,
občasné zašplouchání nebo zanesení kapek vody větrem, vynoření při
odlivu a další efekty způsobily, že směrem dolů roste vlhkost a tedy
i hustota dřeva. Předpokládejme, že hustota v bodě $h$ (měřeno shora
dolů) je dána funkcí
$$\rho(h)=\rho_0(1+kh),$$
kde $\rho_0$ je hustota dřeva nahoře (nejdál od hladiny, kde je trám
nejsušší) a $k$ je konstanta úměrnosti související s hustotou vody a s
tím, jak směrem dolů narůstá vlhkost. Potřebujeme odhadnout hmotnost
trámu bez zásahu do mola, tj. nemůžeme vážit na vahách. Určete
hmotnost trámu výpočtem.

Napište jenom příslušný integrál a okomentujte, jakou metodou bychom
ho počítali. Vlastní výpočet provádět nemusíte. Všimněte si, že úloha
je v podstatě stejná jako úloha o vysílači Kojál z minulého cvičení,
ale vzhledem k jinému tvaru funkce popisující hustotu tentokrát integrujeme
lineární funkci. Pro výpočet integrálu lineární funkce je možné využít
střední hodnotu, která je průměrem funkční hodnoty na začátku a na
konci oboru integrace (viz přednáška).

\newpage

\def\mezera{\vspace*{-20pt}}

\obrazek{mrkev.jpg}

\textbf{Mrkev a vitamín A.}  Mrkev má tvar rotačního tělesa, které
vznikne rotací křivky $$f(x)=\sqrt{14-x}$$ okolo osy $x$ na intervalu
$[0,12]$, kde $x$ je v~centimetrech. Koncentrace vitamínu $A$ se mění
podle vztahu
$$c(x)=\frac 1{12}e^{-x/12} \,\mathrm{mg}\,\mathrm{cm}^{-3}.$$ Jaký je
objem mrkve, obsah vitamínu A a průměrná koncentrace vitamínu A v
mrkvi?

Napište jenom potřebné integrály a vztahy, integrály nepočítejte.

\textit{(Volně přeformulováno podle University of British Columbia,
  Sessional Examinations April 2009.)}


\newpage

\def\mezera{\vspace*{-20pt}}
\obrazek[Wikipedie]{pesticidy.jpg}

\textbf{Pesticidy a játra býložravců.} Přibližná hodnota $C$ koncentrace
jistého pesticidu v~játrech býložravců (měřená v mikrogramech
pesticidu na gram jater) v čase $T$ po zanesení tohoto pesticidu do
životního prostředí je dána vztahem
$$C=e^{-0.25T}\int_0^T 0.32 e^{-0.64 t}\,\mathrm dt.$$
Vypočtěte hodnotu $C$ jako funkci $T$ a ukažte, že maximální hodnota
$C$ je přibližně po dvou letech. 



\textit{(Podle J. Berry, A. Norcliffe, S. Humble: Introductory mathematics through science applications.)}


\end{document}



% hughes - hallet, gleason, Lock, Applied calculus



H=340 # vyska vysilace
rho0=1.225 # https://en.wikipedia.org/wiki/Density_of_air
p0=101325
g=9.81
rho(h)=rho0*exp(-rho0*g*h/p0) # https://cs.wikipedia.org/wiki/Atmosf%C3%A9rick%C3%BD_tlak
S=3^2*sqrt(3.0)/2 # odhad pudorysu
print integrate(S*rho(h),(h,0,H))  # integrace
print S*H*rho0  # vypocet bez zmeny hustoty
S*p0/g*(1-exp(-rho0*g*H/p0))  # kontrola rucne pocitaneho vysledku



var('t')
print integrate(180+3*x,(x,0,20))
print integrate(10^3*(t-7),(t,4,6))
print integrate(4*t/100-3*(t/100)^2,t)
print integrate(3*2019/(20*sqrt(10))*sqrt(x),(x,0,10))