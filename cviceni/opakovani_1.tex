\documentclass{article}

\usepackage[czech]{babel}
\usepackage[T1]{fontenc}
\usepackage[utf8]{inputenc}
\usepackage{amsmath,amsfonts,mathtools,multicol, url, booktabs}
\usepackage[a6paper, margin=20pt, landscape]{geometry}
\usepackage{graphicx, xcolor}
\usepackage{wrapfig, enumerate}
\parskip 10pt
\everymath{\displaystyle}
\parindent 0 pt

\def\zlomek{0.45}

\let\rho\varrho

\def\nic{}


\newcommand\obrazek[2][pixabay.com]{
  \clearpage
  \def\test{#1}
\begin{wrapfigure}{R}{\zlomek\linewidth}
  \begin{minipage}{1.0\linewidth}\parskip 0 pt
  \includegraphics[width=\linewidth]{#2}

  \vspace*{-10pt}
  \ifx\test\nic\else
  \null\hfill{\color{gray}\footnotesize Zdroj: #1}
  \fi

  \mezera
  \end{minipage}
\end{wrapfigure}
}

\let\oldtextbf\textbf
\def\textbf#1{%\newpage
  \oldtextbf{\color{red} #1}}

\def\mezera{\vspace*{10pt}}

\def\stranka{\newpage}

\begin{document}

\rightskip 0 pt plus 1 em
\title{Cvičení Matematika LDF, bak. 1. ročník}
\date{4. dubna 2019}
\maketitle

Příklady, které se budou ve cvičení přeskakovat si projděte
samostatně. Řešení budou zveřejněna na webu předmětu \url{http://user.mendelu.cz/marik/mt}.

\newpage


\def\tg{\mathop{\mathrm{tg}}}
\def\cotg{\mathop{\mathrm{cotg}}}
\def\arctg{\mathop{\mathrm{arctg}}}

\textbf{Derivace.} Veličina $y$ je funkce proměnné $x$. Najděte její derivaci.
\begin{enumerate}[1)]
  \itemsep 10 pt
\item $y=x^2e^x$
\item $y=\frac{x^2+a}{x^3}$
\item $y=e^{-kx}$
\item $y=\pi x^3+2\pi x^2$
\item $y=\sqrt{x+1}$
\end{enumerate}

\newpage

\textbf{Lineární aproximace.} Veličina $y$ je funkce proměnné $x$. Najděte její lineární aproximaci v okolí zadaného bodu.
\begin{enumerate}[1)]
  \itemsep 10 pt
\item $y=xe^x$ v okolí bodu $x=0$
\item $y=rx\left(1-\frac xK\right)$ v okolí bodu $x=0$
\item $y=rx\left(1-\frac xK\right)$ v okolí bodu $x=K$
\item $y=\sqrt x$ v okolí bodu $x=1$
\item $y=\frac 1{\sqrt x}$ v okolí bodu $x=1$
\end{enumerate}

Ve druhém a třetím příkladě aproximujeme funkci modelující růst
populace v prostředí s nosnou kapacitou $K$. Aproximace v okolí bodu
$x=0$ odpovídá velmi malé populaci. Proto se konstanta úměrnosti ze
získané lineární aproximace nazývá \textit{invazní parametr}.

\newpage

\textbf{Integrály a střední hodnota.}

\begin{enumerate}[1)]
    \itemsep 10 pt
\item $\int x^2\sin(x^3)\,\mathrm dx$
\item $\int x^2+3\sqrt x\,\mathrm dx$
\item $\int_0^1 \frac 1{x+1}\,\mathrm dx$
\item $\int_{-1}^1 \frac 1{x^2+1}\,\mathrm dx$
\item Určete střední hodnotu funkce $\sin (kx)$ na intervalu $[0,\pi]$, kde $k$ je přirozené číslo. Vyřešte úlohu obecně a poté pro několik prvních hodnot parametru $k$.
\end{enumerate}

\newpage


\textbf{Diferenciální rovnice.}

\begin{enumerate}[1)]
  \itemsep 10 pt
\item $\frac{\mathrm dr}{\mathrm dt}=kr^2t$
\item $\frac{\mathrm dy}{\mathrm dx}=k\frac{y}{y^2+1}$
\item $\frac{\mathrm dy}{\mathrm dx}=\frac x{y}$

\end{enumerate}


\def\mezera{\vspace*{-20pt}}


\obrazek[Wikipedia]{vrt.jpg}

\textbf{Střední hodnota gravitační síly.}
Gravitační síla od koule je stejná, jako by byla hmotnost soustředěna
v centru. Podle Newtonova gravitačního zákona je proto gravitační síla
nepřímo úměrná druhé mocnině vzdálenosti od středu.
Uvnitř koule je gravitační síla zase přímo úměrná vzdálenosti. Ve vzdálenosti $r$ od středu Země je tedy gravitační síla dána vzorcem
$$F=
\begin{cases}
    \frac{\alpha}{r^2} & r>R,\\
  \beta r & r\leq R,
\end{cases}
$$
kde $R$ je poloměr Země.

\vspace*{-10pt}
\begin{enumerate}\itemsep 0 pt
\item Určete hodnotu konstanty $\beta$ tak, aby tato funkce byla
  spojitá na $(0,\infty)$. 
\item Nejhlubší vrt je hluboký přibližně $12\,\mathrm{km}$ (Kolský superhluboký vrt je na obrázku). Určete
střední hodnotu $F$ na intervalu od povrchu Země do hloubky
12 kilometrů.  
\item Geostacionární družice létají přibližně 36\,000 kilometrů nad
Zemí. Určete střední hodnotu $F$ na intervalu od povrchu
Země do výšky 36\,000 kilometrů nad Zemí.
\item Porovnejte přibližnou změnu velikosti $F$, pokud se z povrchu
Země ponoříme o~malou hodnotu pod povrch a pokud se vzdálíme o stejnou
hodnotu nad povrch Země.
\end{enumerate}

\newpage

\def\mezera{\vspace*{-20pt}}


\obrazek{horizont.jpg}



\textbf{Vzdálenost k horizontu.}
Vzdálenost k horizontu pro pozorovatele ve výšce $h$ je dána funkcí $H=\sqrt {2Rh},$ kde $R$ je poloměr Země (\url{https://aty.sdsu.edu/explain/atmos_refr/horizon.html}). Po dosazení hodnot $$H=3.57\sqrt{h},$$ kde $h$ je v metrech a $H$ v kilometrech. Určete hodnotu této derivace $\frac{\mathrm d H}{\mathrm dh}$ pro $h=5\,\mathrm{m}$ (včetně jednotky) a slovní interpretaci této derivace.


\textbf{Výška rozhledny.} Mr. X chce postavit rozhlednu. Bez
dotací a proto se bude vybírat vstupné a rozhledna musí generovat
zisk.  Cena za postavení a údržbu rozhledny po celou dobu životnosti
bude úměrná výšce rozhledny.  Tržby ze vstupného za celou dobou
životnosti odhadněme, že budou úměrné vzdálenosti k horizontu (jak
daleko je vidět z rozhledny). Tato vzdálenost je úměrná odmocnině z
výšky rozhledny. Zjistěte, zda existuje nějaká optimální výška, kdy
zisk (rozdíl mezi tržbou a náklady) je nejvyšší. Zanedbejte jevy, které
by v praxi úlohu také ovlivnily, jako je inflace, úročení peněz na
účtu apod.
% https://aty.sdsu.edu/explain/atmos_refr/horizon.html


\end{document}


\obrazek{mrkev.jpg}

\textbf{Mrkev a vitamín A.}  Mrkev má tvar rotačního tělesa, které
vznikne rotací křivky $$f(x)=\sqrt{14-x}$$ okolo osy $x$ na intervalu
$[0,12]$, kde $x$ je v~centimetrech. Koncentrace vitamínu $A$ se mění
podle vztahu
$$c(x)=\frac 1{12}e^{-x/12} \,\mathrm{mg}\,\mathrm{cm}^{-3}.$$ Jaký je
objem mrkve, obsah vitamínu A a průměrná koncentrace vitamínu A v
mrkvi?

Napište jenom potřebné integrály a vztahy, integrály nepočítejte.

\textit{(Volně přeformulováno podle University of British Columbia,
  Sessional Examinations April 2009.)}


