\documentclass{article}

\usepackage[czech]{babel}
\usepackage[T1]{fontenc}
\usepackage[utf8]{inputenc}
\usepackage{amsmath,amsfonts,mathtools,multicol, url, booktabs}
\usepackage[a6paper, margin=20pt, landscape]{geometry}
\usepackage{graphicx, xcolor}
\usepackage{wrapfig, enumerate}
\parskip 10pt
\everymath{\displaystyle}
\parindent 0 pt

\def\zlomek{0.45}

\let\rho\varrho

\def\nic{}


\newcommand\obrazek[2][pixabay.com]{
  \clearpage
  \def\test{#1}
\begin{wrapfigure}{R}{\zlomek\linewidth}
  \begin{minipage}{1.0\linewidth}\parskip 0 pt
  \includegraphics[width=\linewidth]{#2}

  \vspace*{-10pt}
  \ifx\test\nic\else
  \null\hfill{\color{gray}\footnotesize Zdroj: #1}
  \fi

  \mezera
  \end{minipage}
\end{wrapfigure}
}

\let\oldtextbf\textbf
\def\textbf#1{%\newpage
  \oldtextbf{\color{red} #1}}

\def\mezera{\vspace*{10pt}}

\def\stranka{\newpage}

\begin{document}

\rightskip 0 pt plus 1 em
\title{Cvičení Matematika LDF, bak. 1. ročník}
\date{11. dubna 2019}
\maketitle

Příklady, které se budou ve cvičení přeskakovat si projděte
samostatně. Řešení budou zveřejněna na webu předmětu \url{http://user.mendelu.cz/marik/mt}.

\newpage

\textbf{Násobení matic.}
Vynásobte matice $A$ a $B$ pro obě pořadí násobení.
\begin{equation*}
A=  \begin{pmatrix}
  1 & -2 & 3\\
  0 & 1 & 0\\
  1 & 2 & -2
\end{pmatrix},\qquad
B=  \begin{pmatrix}
  2 & -2 & 2\\
  -1 & 2 & -1\\
  0 & 1 & 3
\end{pmatrix}.
\end{equation*}
Vynásobte matice $B$ a $C$ pro obě pořadí násobení, je-li 
\begin{equation*}
C=  \begin{pmatrix}
  1 & 0 & 0\\
  0 & 3 & 0\\
  0 & 0 & 4
\end{pmatrix}.
\end{equation*}

\newpage

\textbf{Soustava rovnic jako násobení matic.}
Zapište soustavu rovnic pomocí maticového násobení
\begin{equation*}
  \begin{aligned}
2x_1-3x_2+2x_3&{}=12\\
2x_1+\phantom{1}x_2+\phantom{1}x_3&{}=21\\
-x_1+3x_2+\phantom{1}x_3&{}=0\\
\end{aligned}
\end{equation*}


\newpage

\def\tg{\mathop{\mathrm{tg}}}
\def\cotg{\mathop{\mathrm{cotg}}}
\def\arctg{\mathop{\mathrm{arctg}}}

\textbf{Matice rotace.}
Matice rotace o úhel $\theta$ v kladném smyslu je
\begin{equation*}R_\theta=
  \begin{pmatrix}
    \cos\theta & -\sin \theta\\
    \sin\theta & \cos\theta
  \end{pmatrix}.
\end{equation*}
Násobením ověřte, že matice otočení o úhel $-\theta$ je k této matici inverzní.

\textit{Návod:} Funkce kosinus je sudá funkce a funkce sinus je lichá funkce. Proto platí $$\cos(-\theta)=\cos \theta $$ a $$\sin(-\theta)=-\sin \theta.$$

\newpage

\textbf{Matice posunutí.}
Transformace pomocí násobení matic zachovává počátek a nemůže proto charakterizovat například posunutí roviny. Pokud chceme mít
pomocí maticového násobení realizováno i posunutí, musíme zavést
homogenní souřadnice a ztotožnit bod $(x,y)$ s vektorem
$(x,y,1)^T$. Ukažte, že matice
\begin{equation*}P_{a,b}=
\begin{pmatrix}
  1& 0& a\\
  0 & 1 & b\\
  0& 0& 1
\end{pmatrix}
\end{equation*}
je matice posunutí o $a$ doprava a $b$ nahoru. Odhadněte, jak bude
vypadat matice popisující opačnou transformaci a pro obě pořadí
součinu ověřte, že součin obou těchto matic je pro obě pořadí jednotková matice.


\newpage

\textbf{Matice, zachovávající význačné směry.} Dřevo má tři výrazné
směry a pokud máme možnost zvolit souřadnou soustavu tak, aby tyto
směry byly dány vektory $(1,0,0)$, $(0,1,0)$ a $(0,0,1)$, formulace
fyzikálních zákonů se zjednoduší. Najděte
\begin{enumerate}
\item  nejobecnější matici $3\times 3$, která zachovává směr vektoru $(1,0,0)$,
\item  nejobecnější symetrickou matici $3\times 3$, která zachovává směr vektoru $(1,0,0)$,
\item  nejobecnější symetrickou matici $3\times 3$, která zachovává směr vektorů $(1,0,0)$, $(0,1,0)$, $(0,0,1)$.
\end{enumerate}

\newpage

\textbf{Určete následující determinanty}

\begin{enumerate}
\item $D_1=
  \begin{vmatrix}
    2 & -1 \\ 4 &3
  \end{vmatrix}
  $
\item $D_2=
  \begin{vmatrix}
    2 & -1 \\ x-4 &y-3
  \end{vmatrix}
  $

  ($D_2=0$ je přímka daná bodem $(4,3)$ a směrovým vektorem $(2,-1)$)
\item $D_3=
  \begin{vmatrix}
    2-\lambda & -1 \\ 4 & 3-\lambda
  \end{vmatrix}
  $ (charakteristický polynom matice z prvního bodu)
\item
  $D_4=
  \begin{vmatrix}
    1 & -1 & 0\\ 2 & 3 & 1 \\ -1 &-1 & 2\end{vmatrix}
  $
\item
  $D_5=
  \begin{vmatrix}
    a & -1 & 0\\ 2 & 3 & 1 \\ -1 &-1 & 2\end{vmatrix}
  $  
\item
  $D_5=
  \begin{vmatrix}
    2-\lambda & 0 & 0\\ 0 & 3-\lambda & 0 \\ 0 & 0& 7-\lambda \end{vmatrix}
  $  (charakteristický polynom diagonální matice)
\end{enumerate}

\newpage

\textbf{Matice projekce.}
Matice $P=
\begin{pmatrix}
  \cos^2 \alpha & \cos \alpha \sin \alpha \\
  \cos\alpha\sin\alpha & \sin^2 \alpha
\end{pmatrix}
$ reprezentuje projekci na přímku jdoucí středem pod úhlem $\alpha$.
\begin{enumerate}
\item Ukažte, že platí $P^2=P$. To znamená, že body na přímce se
  zobrazí samy na sebe.
\item Ukažte, (nemusíte výpočtem, například graficky, nebo využitím
  toho, že každý bod přímky se zobrazí sám na sebe) že dva různé body
  se projekcí mohou zobrazit na stejný bod a proto není naděje na to
  mít inverzní zobrazení. Proto neexistuje inverzní matice, což můžete
  ověřit výpočtem determinantu.
\end{enumerate}

\newpage

\textbf{Matice derivování}
Ukažte, že matice 
$A=
\begin{pmatrix}
  0 & 0 & 0 \\
  2 & 0 & 0 \\
  0 & 1 & 0
\end{pmatrix}
$
je matice derivování polynomů stupně nejvýše $2$, pokud polynom $ax^2+bx+c$ ztotožníme s vektorem $
\begin{pmatrix}
  a \\ b\\c
\end{pmatrix}
$. Ověřte, že derivování je ekvivalentní maticovému násobení pro
polynomy $x^2$, $x$ a $1$ a poté vysvětlete, proč se stačí zaměřit na
tyto tři polynomy. Ukažte, že je možno výše uvedenou vlastnost ověřit
i pro obecný polynom $ax^2+bx+c$.  Vysvětlete, jak bychom
interpretovali matici $A^2$ a $A^3$ a tyto matice vypočtěte.


\end{document}

