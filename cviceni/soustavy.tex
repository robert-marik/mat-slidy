\documentclass{article}

\usepackage[czech]{babel}
\usepackage[T1]{fontenc}
\usepackage[utf8]{inputenc}
\usepackage{amsmath,amsfonts,mathtools,multicol, url, booktabs}
\usepackage[a6paper, margin=20pt, landscape]{geometry}
\usepackage{graphicx, xcolor, tikz}
\usepackage{wrapfig, enumerate}
\parskip 10pt
\everymath{\displaystyle}
\parindent 0 pt


\def\zlomek{0.45}

\let\rho\varrho

\def\nic{}


\newcommand\obrazek[2][pixabay.com]{
  \clearpage
  \def\test{#1}
\begin{wrapfigure}{R}{\zlomek\linewidth}
  \begin{minipage}{1.0\linewidth}\parskip 0 pt
  \includegraphics[width=\linewidth]{#2}

  \vspace*{-10pt}
  \ifx\test\nic\else
  \null\hfill{\color{gray}\footnotesize Zdroj: #1}
  \fi

  \mezera
  \end{minipage}
\end{wrapfigure}
}

\let\oldtextbf\textbf
\def\textbf#1{%\newpage
  \oldtextbf{\color{red} #1}}

\def\mezera{\vspace*{10pt}}

\def\stranka{\newpage}

\input mat-mak
\begin{document}

\rightskip 0 pt plus 1 em
\title{Cvičení Matematika LDF, bak. 1. ročník}
\date{18. dubna 2019}

\maketitle

Řešení budou zveřejněna na webu předmětu \url{http://user.mendelu.cz/marik/mt}.
Další soustavy zařadí do výuky cvičící.

\newpage


\textbf{Soustava s jediným řešením.} Vyřešte soustavu rovnic.

\shorthandoff{-}
\begin{equation*}
\soustava
1 2 2 3
2 2 -1 1
2 3 1 -1

\end{equation*}

\newpage
\textbf{Soustava s nekonečně mnoha řešeními.} Vyřešte soustavu rovnic.

\begin{equation*}
\soustava
3 -1 -1 -1 0
2 1 1 -2 -4
1 -2 -2 1 4
3 -1 -1 1 6

\end{equation*}

\newpage
\textbf{Vlastní čísla a vektory matice $2\times 2$.}

Uvažujme matici
\begin{equation*} A=
  \begin{pmatrix}
  3 & -1\\
  -1 & 1
\end{pmatrix}.
\end{equation*}
\begin{enumerate}
\item Určete vlastní čísla a jednotkové vlastní vektory této matice.
\item Sestavte matici $P$ tak, aby ve sloupcích obsahovala jednotkové vlastní vektory.
Pokud je to možné, napište matici $P$ tak, aby její determinant byl kladný.
\item Ověřte, že  $P^TAP=D$  je diagonální matice.
\end{enumerate}

V případě nutnosti pracujte s~desetinnými čísly.

\newpage \textbf{Komentář k předchozímu.} Zobrazení popsané maticí $A$
je naznačeno na obrázku a jak vidno, mění tvar modrého jednotkového čtverce.

\hbox to \hsize{\begin{minipage}[t]{0.35\linewidth}\vspace*{0pt}
\begin{tikzpicture}[scale=1]
  \draw[-latex](0,0)--(3,0);
  \draw[-latex](0,0)--(0,1.5);

  \fill[blue, opacity=0.5] (0,0) rectangle (1,1);
  \fill[red, opacity=0.5] (0,0) -- (3,-1) -- (2,0) -- (-1,1) -- cycle;
  \draw[ultra thick,blue,-latex] (0,0)--(1,1) node[right] {$u$};
  \draw[ultra thick,red,-latex] (0,0)--(2,0) node[above right] {$u'$};

  \draw[ultra thick,blue,-latex] (0,0)--(0,1) node[right] {$v$};
  \draw[ultra thick,red,-latex] (0,0)--(-1,1) node[above right] {$v'$};
  \draw (-0.5,1.7) node [right, gray] {\footnotesize Vzor je modře, obraz červeně.};
\end{tikzpicture}

Vektor $u$ se při zobrazení otočí po směru a $v$ proti směru
hodinových ručiček. Někde mezi nimi bude vektor, který při zobrazení
směr nemění. To bude vlastní vektor a z obrázku je možné odhadnout, že
bude příslušný menší vlastní hodnotě.
\end{minipage}\hfil\vrule\hfil
\begin{minipage}[t]{0.62\linewidth}\vspace*{-10pt}
  \begin{tikzpicture}[scale=1]
      \draw (1.2,1.2) node [right,gray, text width=4cm] {\footnotesize Vzor je vybarven, obraz je znázorněn jenom obrysem, barvy obrazu a vzoru si odpovídají.};
  \draw[-latex](0,0)--(3,0);
  \draw[-latex](0,0)--(0,1.5);

  \fill[blue, opacity=0.5] (0,0) rectangle (1,1);
  \draw[ultra thick, blue] (0,0) -- (3,-1) -- (2,0) -- (-1,1) -- cycle;

  \fill[green, opacity=0.5] (0,0)--(0.382683432365090,0.923879532511287)--
  (-0.541196100146197, 1.30656296487638) --
  (-0.923879532511287,0.382683432365090)--cycle;

  \draw[ultra thick, green]  (0,0)--(0.224170764583983, 0.541196100146197)--(-2.93015126531497, 1.84775906502258)-- (-3.15432202989895, 1.30656296487638) -- cycle;
\end{tikzpicture}

Zobrazení symetrickou maticí $A$ uvažovanou v tomto příkladě si můžeme představit tak, že máme elastickou pásku, která se působením síly deformuje tak, že se natahuje v jednom směru a zužuje v příčném směru. Čtverec nakreslený na pásce tak, že jeho strany leží v těchto směrech (zelený) se deformuje na obdélník. Obecný čtverec (modrý) se deformuje tak, že se mění úhly. Na přednášce jsme si naznačili, jak se ukáže, že deformace v těchto směrech je extremální, vrátíme se k~tomu v záverečném shrnutí na konci semestru, protože jde o kombinaci úlohy z lineární algebry a diferenciálního počtu.

\end{minipage}}

\newpage
\textbf{Vlastní čísla a vektory matice $3\times 3$.}

V minulém cvičení jsme ukázali, že nejobecnější symetrická matice zachovávající směr vektoru  $(1,0,0)^T$ má v prvním řádku a prvním sloupci jenom jeden nenulový prvek, prvek v hlavní diagonále.

Uvažujme matici
\begin{equation*}
  \begin{pmatrix}
  5 & 0 & 0\\
  0 & 2 & 1\\
  0 & 1 & 3
\end{pmatrix},
\end{equation*}
která je tohoto typu.
Určete vlastní čísla a zbylé vlastní vektory matice.


\end{document}



https://sagecell.sagemath.org/?z=eJxtkEFvgyAYhu8m_geSHaoTDYgiLOHQ07KDt92MWWxnVxImBq2b_vphbbpulYQvAZ73e9-PYdJmFF6BICphgS8V24ps9V3nYfhDoIiwmDKSkJjQFHF7G_GYsIynJE4xjllmqRBFaYIxpxghnFDMMwhwRBBNacxpwjJK2IL908K7_nOGrfiseiO_vaIgMJzDhXPG-cl1DtqAHMgGFNvyyXWAXXpnqmkUxVDve2086T_mYMbkjJ3nKRfwJX8WrZZN7y0SCPZaaSM2pn7fQNDJqRYJ8oNWq_FDN-tUpdpjJezEfqBkU69B_q-d3cFieQ5yNbnyO3Wq7wWL_0VykEqJV3Oqb8xX9d1RfwEvvzl5tp1Vda39mjdT9VIL7P8AuOmQhw==&lang=sage&interacts=eJyLjgUAARUAuQ==