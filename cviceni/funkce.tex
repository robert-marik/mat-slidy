\documentclass{article}

\usepackage[czech]{babel}
\usepackage[T1]{fontenc}
\usepackage[utf8]{inputenc}
\usepackage{amsmath,amsfonts,mathtools,multicol, url, booktabs}
\usepackage[a6paper, margin=20pt, landscape]{geometry}
\usepackage{graphicx, xcolor, tikz}
\usepackage{wrapfig, enumerate}
\parskip 10pt
\everymath{\displaystyle}
\parindent 0 pt


\def\zlomek{0.45}

\let\rho\varrho

\def\nic{}


\newcommand\obrazek[2][pixabay.com]{
  \clearpage
  \def\test{#1}
\begin{wrapfigure}{R}{\zlomek\linewidth}
  \begin{minipage}{1.0\linewidth}\parskip 0 pt
  \includegraphics[width=\linewidth]{#2}

  \vspace*{-10pt}
  \ifx\test\nic\else
  \null\hfill{\color{gray}\footnotesize Zdroj: #1}
  \fi

  \mezera
  \end{minipage}
\end{wrapfigure}
}

\let\oldtextbf\textbf
\def\textbf#1{%\newpage
  \oldtextbf{\color{red} #1}}

\def\mezera{\vspace*{10pt}}

\def\stranka{\newpage}


\begin{document}

\rightskip 0 pt plus 1 em
\title{Cvičení Matematika LDF, bak. 1. ročník}
\date{2. května 2019}

\maketitle

Řešení budou zveřejněna na webu předmětu
\url{http://user.mendelu.cz/marik/mt}.  Další příklady (průběhy nebo
dokončení vybraných příkladů z minulých cvičení) zařadí do výuky
cvičící.


\def\postup{\textbf{Postup.}
  \begin{itemize}
  \item Definiční obor, nulové body, znaménko funkce (pokud máme definiční obor a nulové body), limity v bodech, kde to umíme a kde to má smysl.
  \item Body, kde je nulová derivace, intervaly monotonie, lokální extrémy. Kontrola, že získané výsledky nejsou ve sporu s výsledky z předchozího bodu.
  \item Body, kde je nulová druhá derivace, intervaly konvexnosti a konkávnosti, inflexní body. Kontrola, že získané výsledky nejsou ve sporu s výsledky z předchozích bodů.
  \item Graf. Schematicky. Raději výrazně vyznačit kvalitativní vlastnosti, než se snažit o dodržení měřítka na ose $y$.  Kontrola, že získaný graf není ve sporu s výsledky z~předchozích bodů.
  \end{itemize}
}
\newpage

\textbf{Příklad 1.}
Vyšetřete průběh funkce $$y=\frac{x^3}{x-1},$$ když víme, že
$$y'=\frac{{\left(2 \, x - 3\right)} x^{2}}{{\left(x - 1\right)}^{2}},\quad
y''=\frac{2 \, {\left(x^{2} - 3 \, x + 3\right)} x}{{\left(x - 1\right)}^{3}}.$$

\postup

\newpage


\textbf{Příklad 2.}
Vyšetřete průběh funkce $$y=\frac{x}{x^2+1},$$ když víme, že
$$y'=\frac{1-x^2}{(x^2+1)^2},\quad
y''=\frac{2 x (x^2-3)}{(x^2+1)^3}.$$

\postup


\newpage



\textbf{Příklad 3.}
Vyšetřete průběh funkce $$y=\frac{x}{(x+1)^2},$$ když víme, že
$$y'=-\frac{x - 1}{{\left(x + 1\right)}^{3}},\quad
y''=\frac{2 \, {\left(x - 2\right)}}{{\left(x + 1\right)}^{4}}.$$

\postup


\end{document}

