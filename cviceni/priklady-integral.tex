\documentclass{article}

\usepackage[czech]{babel}
\usepackage[T1]{fontenc}
\usepackage[utf8]{inputenc}
\usepackage{amsmath,amsfonts,mathtools,multicol, url, booktabs}
\usepackage[a6paper, margin=20pt, landscape]{geometry}
\usepackage{graphicx, xcolor}
\usepackage{wrapfig, enumerate}
\parskip 10pt
\everymath{\displaystyle}
\parindent 0 pt

\def\zlomek{0.45}

\newcommand\obrazek[2][pixabay.com]{
  \clearpage
\begin{wrapfigure}{R}{\zlomek\linewidth}
  \begin{minipage}{1.0\linewidth}\parskip 0 pt
  \includegraphics[width=\linewidth]{#2}

  \vspace*{-10pt}
  \null\hfill{\color{gray}\footnotesize Zdroj: #1}

  \mezera
  \end{minipage}
\end{wrapfigure}
}

\let\oldtextbf\textbf
\def\textbf#1{%\newpage
  \oldtextbf{\color{red} #1}}

\def\mezera{\vspace*{-20pt}}

\def\stranka{\newpage}

\begin{document}

\rightskip 0 pt plus 1 em
\title{Cvičení Matematika LDF, bak. 1. ročník}
\date{2019--03--14 (14. března slavíme $\pi$-day)}
\maketitle



\newpage


\def\tg{\mathop{\mathrm{tg}}}
\def\cotg{\mathop{\mathrm{cotg}}}
\def\arctg{\mathop{\mathrm{arctg}}}


\textbf{Výpočet integrálu.} Najděte následující integrály.

\begin{multicols}3

\def\priklad #1. {$\int#1\,\mathrm dx$}

\begin{enumerate}[(1)]
  \setlength\itemsep{20pt}
\item \priklad x^2+2x.
\item \priklad \sqrt{x}\Bigl(x+\sqrt{x}\Bigr).
\item \priklad \frac 1{\sqrt x}+\sqrt x.
\item \priklad \frac{x^2-1}x.
\item \priklad e^x+e^{2x}.
\item \priklad \sin\left(x+\frac \pi 3\right).
\item \priklad \frac 1{4x^2}.
\item \priklad \frac 1{4+x^2}.
\item \priklad \frac 1{1+4x^2}.
\item $\int \frac 1{r^2}-\frac 1{r^6}\,\mathrm dr$
\item \priklad _0^{\frac \pi 2} \cos x.
\item \priklad _0^1(x-1)^3.
\item \priklad _{-1}^{1} 3x^2+x^5.
\item $\int_0^{10} e^{-0.1 t}\,\mathrm dt$
\item $\int_{-a}^{a} u^3\,\mathrm du$

  \end{enumerate}
\end{multicols}

\newpage

\textbf{Práce na pružině.} Síla působící na pružinu je úměrná
deformaci pružiny. Natáhneme-li pružinu z rovnovážného stavu o hodnotu
$x$, je nutno působit silou $kx$, kde $k$ je konstanta (tuhost
pružiny). Vypočtěte práci nutnou k natažení pružiny o jednotkovou
délku a poté o délku $l$.


\newpage

\textbf{Výpočet $\pi$.}  Pro $n\neq -1$ vypočtěte integrály
$$\int_0^1 x^n\,\mathrm dx \qquad \text{a} \qquad \int_0^1
\frac{1}{1+x^2}\,\mathrm dx.$$

\textbf{Poznámka:} Vzorec pro součet geometrické
řady s kvocientem $x^2$ je $$\frac{1}{1+x^2}=1-x^2+x^4-x^6+\cdots$$ po integrování (a po
zapojení teorie nekonečných řad, která ospravedlní integrování člen po
členu a to, že v horní mezi je $x=1$, přestože řada pro $x=1$ nekonverguje) dává
$$
\int_0^1 \frac{1}{1+x^2}\,\mathrm dx
=
\int_0^1 1\,\mathrm dx
-
\int_0^1 x^2\,\mathrm dx
+
\int_0^1 x^4\,\mathrm dx
-
\int_0^1 x^6\,\mathrm dx
+\cdots .
$$
Po zintegrování vlevo dostaneme veličinu obsahující $\pi$ a vpravo
součet racionálních čísel. Tím je možné odhadnout hodnotu $\pi$. Tato
technika, používaná v~jistých obměnách v~17. a 18. století, je mnohem
efektivnější pro výpočet $\pi$, než starší metoda pravidelných
mnohoúhelníků vepsaných do kružnice. Dnes máme k dispozici řady,
které k hodnotě $\pi$ konvergují mnohem rychleji.

\newpage

\obrazek{slovnik.jpg}

\textbf{Napouštění nádrže.} Chemikálie teče do nádrže rychlostí
$180+3t$ litrů za minutu, kde $t\in [0,60]$. Určete, kolik chemikálie
nateče do nádrže během prvních 20 minut. (Podle Stewart: Calculus.)

\textbf{Prasklá kanalizace }způsobila znečištění jezera v rekreační
oblasti. Koncentrace bakterií $C(t)$ (v bakteriích na kubický
centimetr) se po ošetření úniku pro $t\in[0,6]$ vyvíjí
rychlostí $$C'(t)=10^3(t-7).$$ Jaká je změna koncentrace bakterií mezi
čvrtým a šestým dnem? (Podle Mardsen, Weinstein: Calculus I.)

\textbf{Rychlost učení.} Nechť $W(t)$ je počet francouzských slovíček,
které se naučíme po $t$ minutách. Typicky může být
$$W(0)=0\quad \text {a} \quad W'(t)=\frac{4t}{100}-3\left (\frac  t{100}\right)^2.$$ Najděte pomocí inetgrálu funkci $W(t)$. (Podle Mardsen, Weinstein: Calculus I.)


\end{document}



% hughes - hallet, gleason, Lock, Applied calculus