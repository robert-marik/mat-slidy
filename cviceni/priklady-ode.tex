\documentclass{article}

\usepackage[czech]{babel}
\usepackage[T1]{fontenc}
\usepackage[utf8]{inputenc}
\usepackage{amsmath,amsfonts,mathtools,multicol, url, booktabs}
\usepackage[a6paper, margin=20pt, landscape]{geometry}
\usepackage{graphicx, xcolor}
\usepackage{wrapfig, enumerate}
\parskip 10pt
\everymath{\displaystyle}
\parindent 0 pt

\def\zlomek{0.45}

\let\rho\varrho

\def\nic{}


\newcommand\obrazek[2][pixabay.com]{
  \clearpage
  \def\test{#1}
\begin{wrapfigure}{R}{\zlomek\linewidth}
  \begin{minipage}{1.0\linewidth}\parskip 0 pt
  \includegraphics[width=\linewidth]{#2}

  \vspace*{-10pt}
  \ifx\test\nic\else
  \null\hfill{\color{gray}\footnotesize Zdroj: #1}
  \fi

  \mezera
  \end{minipage}
\end{wrapfigure}
}

\let\oldtextbf\textbf
\def\textbf#1{%\newpage
  \oldtextbf{\color{red} #1}}

\def\mezera{\vspace*{10pt}}

\def\stranka{\newpage}

\begin{document}

\rightskip 0 pt plus 1 em
\title{Cvičení Matematika LDF, bak. 1. ročník}
\date{28. března 2019}
\maketitle

Příklady, které se budou ve cvičení přeskakovat si projděte
samostatně. Řešení budou zveřejněna na webu předmětu \url{http://user.mendelu.cz/marik/mt}.

\newpage


\def\tg{\mathop{\mathrm{tg}}}
\def\cotg{\mathop{\mathrm{cotg}}}
\def\arctg{\mathop{\mathrm{arctg}}}


\textbf{Řešení ODE a IVP.} 

\begin{multicols}2

\def\priklad #1. {$#1$}

\begin{enumerate}[(1)]
  \setlength\itemsep{20pt}
\item \priklad \frac{\mathrm dy}{\mathrm dx}=xy^2.
\item \priklad \frac{\mathrm dy}{\mathrm dx}=\frac{\sin x}{y^2}.
\item \priklad \frac{\mathrm dy}{\mathrm dx}=x\sqrt y.
\item \priklad \frac{\mathrm dy}{\mathrm dt}=te^y.
\item \priklad \frac{\mathrm dx}{\mathrm dt}=x^2-x^2t^3.
\item \priklad \frac{\mathrm dy}{\mathrm dx}=x\sqrt y,\ \ y(0)=1.
\item \priklad \frac{\mathrm dy}{\mathrm dx}=(xy)^2,\ \ y(0)=-1.
\item \priklad \frac{\mathrm dr}{\mathrm dt}=kr^3,\ \ r(0)=r_0.
\item \priklad \frac{\mathrm dm}{\mathrm dt}=m+2,\ \ m(0)=0.
\item \priklad \frac{\mathrm dm}{\mathrm dt}=m+2,\ \ m(0)=-2.
\end{enumerate}
\end{multicols}


\newpage


\obrazek[www.rodovystatek.cz]{voda_plastovky.jpg}


\textbf{Vypouštění nádrže.} Z fyziky je známo, že rychlost s jakou
vytéká tekutina otvorem u dna nádoby je úměrná odmocnině výšky hladiny
(protože se mění potenciální energie úměrná výšce na kinetickou
energii úměrnou druhé mocnině rychlosti). Proto je i rychlost s jakou
se zmenšuje objem vody v nádrži úměrná odmocnině výšky
hladiny. Modelujte proces pomocí diferenciální rovnice
\begin{itemize}
\item pro nádrž
cylindrického tvaru (válec postavený na podstavu), 
\item pro nádrž ve tvaru
kvádru 
\item pro nádrž ve tvaru kužele otočeného vrcholem dolů (trychtýř).
\end{itemize}



\newpage

\obrazek[]{pokros.jpg}


\textbf{Stavebniny vedle čebínského nádraží.} Hromada sypkého
materiálu má tvar kužele. Úhel u~vrcholu je konstantní, daný
mechanickými vlastnostmi materiálu a je nezávislý na
objemu. Předpokládejme, že personál stavebnin přisypává na hromadu
materiál konstantní rychlostí (v jednotkách objemu za jednotku
času). Tato hromada je však v poměrně otevřené krajině a vítr
rozfoukává materiál po okolí. Je rozumné předpokládat, že rozfoukávání
se děje rychlostí úměrnou povrchu, tj. rychlostí úměrnou druhé mocnině
některého délkového parametru, například průměru, poloměru nebo
výšky. Modelujte proces pomocí diferenciální rovnice. Sestavte
diferenciální rovnici pro objem hromady.
\begin{itemize}
\itemsep 0 pt
\item Může hromada skončit i při neustálém přisypávání celá rozfoukaná?
\item Mohou pracovníci navršit hromadu do libovolné výšky anebo pro velkou hromadu je již rozfoukávání rychlejší než přisypávání?
\end{itemize}
Odpovědi zjistíte, když
rozhodnete, zda existuje konstantní řešení a zda je toto řešení
stabilní.



\newpage

\textbf{Ropná skvrna.} Kruhová ropná skvrna na hladině se rozšiřuje
tak, že poloměr roste rychlostí, která je nepřímo úměrná druhé mocnině
poloměru. Sestavte diferenciální rovnici popisující tento proces a
vyřešte ji.

\textbf{Učení.} Rychlost učení (tj. časová změna objemu osvojené látky
nebo procento z~maximální manuální zručnosti) je úměrná objemu dosud
nenaučené látky. Sestavte diferenciální rovnici modelující proces
učení probíhající podle těchto pravidel.


\textbf{Chemická směs.} Chemikálii rozpouštíme v nádrži tak, že do
nádrže pumpujeme vodu a směs odčerpáváme. Objem směsi roste podle
vztahu $20+2t$. Množství chemikálie $y$ klesá rychlostí, která je
úměrná $y$ a nepřímo úměrná objemu roztoku v nádrži.


\textbf{Růst buňky.} Buňka ve tvaru koule o poloměru $r$ získává
živiny rychlostí úměrnou povrchu a spotřebovává živiny rychlostí
úměrnou objemu. Získávání živin a spotřeba živin jsou tedy úměrné po
řadě $r^2$ a $r^3$. Předpokládejme, že rychlost s jakou roste objem je
úměrná rozdílu mezi příjmem a výdejem. Sestavte diferenciální rovnici
pro poloměr buňky, najděte její konstantní řešení a posuďte jeho
stabilitu. Sestavte i diferenciální rovnici pro objem buňky. (Podobnou
úvahu lze provést i pro jiné živé organismy a odsud plynou omezení
daná efektivitou stavby těla. Například buňky větší než
$1\,\mathrm{mm}$ se nevyskytují příliš často. Volně podle
L.~Edelstein-Keshet: Differential Calculus for the Life Sciences.)




\end{document}

