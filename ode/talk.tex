S diferenciální rovnicemi jsme se vlastně už setkali. Diferenciální rovnice jsou rovnice, kde neznámou v rovnici je funkce a v rovnici figuruje i rychlost změny této funkce, tedy její derivace. Viděli jsme že tyto rovnice jsou v některých případech přirozenými matematickými modely reálně probíhajících dějů v přírodě a vlastně jsme se naučili tyto rovnice formulovat už na přednáškách a cvičeních věnovaných derivacím. 

Nyní si tyto znalosti budeme sumarizovat a prohlubovat. Formálně tedy difference dolní rovnicí rozumíme rovnici tvaru 1 kdy na levé straně je derivace veličiny y podle x a na pravé straně je nějaký výraz který může obsahovat veličiny x a y. 

Někdy bývá zvykem derivaci označovat pomocí čárky. Jindy bývá zvykem z derivací pracovat formálně jako s podílem diferenciálů a převést rovnici do tvaru bez zlomku na levé straně. Diferenciální rovnice je v praxi vlastně scénář který popisuje jak se studovaný systém vyvíjí. Abychom dokázali dávat předpovědi jaké budou hodnoty veličin po uplynutí určitého času, je nutné kromě scénáře vývoje, tedy kromě diferenciál rovnice, mít k dispozici i výchozí stav. Tento výchozí stav se zadává ve formě počáteční podmínky kdy pro daný bod přepisujeme funkční hodnotu. Počáteční podmínka se někdy též nazývá Cauchyova podmínka a úlohu najít řešení diferenciální rovnice které splňuje počáteční podmínku se nazývá počáteční úloha nebo též Cauchyova úloha. 

Diferenciální rovnice tedy může například udávat rychlost ochlazování horkého nápoje podle Newtonova zákona který Vyjadřuje to že rychlost poklesu teploty je úměrná teplotnímu rozdílu. Počáteční podmínkou by byla například teplota na začátku sledování procesu. Řešením by byla funkce, která by po zadání času vracela hodnotu teploty nápoje. 

Diferenciální rovnice může například udávat rychlost růstu populace živočišného druhu a počáteční podmínka může udávat velikost populace na začátku sledování. Řešením této rovnice bychom získali funkci časové proměnné která po dosazení jakéhokoliv budoucího nebo minulého času dává odpovídající hodnotu velikosti populace. 

S počátečními úlohami pro diferenciální rovnice je úzce spjata problematika existence a jednoznačnost řešení počáteční úlohy. Ukazuje se že pokud je funkce dostatečně pěkná vzhledem k závislé proměnné potom má úloha právě jedno řešení. V jakém smyslu má být funkce dostatečně pěkná ukazuje například tato existenční věta která říká že existence a jednoznačnost řešení je zaručena ohraničenou parciální derivací pravé strany diferenciální rovnice podle y. Pokud by podmínky této věty nebyly splněny Můžeme o jednoznačnost přijít. Například pokud je funkce pouze spojitá potom je zaručena pouze existence a není zaručena jednoznačnost. 

Protože pro daný systém existuje nekonečně mnoho možností jakou zadat počáteční podmínku , je zřejmé že funkcí,  které vyhovují zadané diferenciální rovnici, existuje nekonečně mnoho. Formule která zahrnuje všechna tato řešení obsahuje nějakou konstantu v roli parametru. Obvykle používáme velké C a taková funkce se nazývá obecné řešení. Například všechny funkce které jsou rovný své vlastní derivaci, tedy všechny funkce vyhovující rovnici 3, jsou tvaru 4 kde C je libovolné reálné číslo. Pokud máme obecné řešení , nebývá zpravidla problém pomocí počáteční podmínky z obecného řešení najít řešení zadané počáteční úlohy. Takové řešení se nazývá partikulární řešení.
