V této přednášce se seznámíme s diferenciálními rovnicemi. To není nic jiného, než správný název pro to, čemu jsme dříve říkali matematické modely formulované pomocí derivace. Viděli jsme že tyto modely jsou v některých případech přirozeným matematickým aparátem pro popis reálně probíhajících dějů v přírodě. V přednášce se seznámíme se základním názvoslovím spojeným s touto problematikou a naučíme se numerické i analytické řešení. Protože se často setkáváme s modely nezávislými na čase, budeme se problematice těchto modelů věnovat podrobněji. Tyto modely mají tu vlastnost, že se ohraničená řešení po čase ustálí okolo stabilní hodnoty. Naučíme se hledat hodnoty odpovídající ustáleným řešením a z nich vybrat ty stabilní, k nimž systém může konvergovat, nebo nestabilní, které oddělují oblasti, ze kterých systém dospívá k jednotlivým stabilním stavům.

\section*{ODR prvního řádu}

Formálně diferenciální rovnicí rozumíme rovnici tvaru 1 kdy na levé straně je derivace veličiny y podle x a na pravé straně je nějaký výraz který může obsahovat veličiny x a y. 

Někdy bývá zvykem derivaci označovat pomocí čárky. Jindy bývá zvykem z derivací pracovat formálně jako s podílem diferenciálů a převést rovnici do tvaru bez zlomku na levé straně. Diferenciální rovnice je v praxi vlastně scénář který popisuje jak se studovaný systém vyvíjí. Abychom dokázali dávat předpovědi jaké budou hodnoty veličin po uplynutí určitého času, je nutné kromě scénáře vývoje, tedy kromě diferenciál rovnice, mít k dispozici i výchozí stav. Tento výchozí stav se zadává ve formě počáteční podmínky kdy pro daný bod přepisujeme funkční hodnotu. Počáteční podmínka se někdy též nazývá Cauchyova podmínka a úlohu najít řešení diferenciální rovnice které splňuje počáteční podmínku se nazývá počáteční úloha nebo též Cauchyova úloha. 


Diferenciální rovnice může například udávat rychlost ochlazování horkého nápoje podle Newtonova zákona který Vyjadřuje to že rychlost poklesu teploty je úměrná teplotnímu rozdílu. Počáteční podmínkou by byla například teplota na začátku sledování procesu. Řešením by byla funkce, která by po zadání času vracela hodnotu teploty nápoje. 

Diferenciální rovnice může například udávat rychlost růstu populace živočišného druhu a počáteční podmínka může udávat velikost populace na začátku sledování. Řešením této rovnice bychom získali funkci časové proměnné která po dosazení jakéhokoliv budoucího nebo minulého času dává odpovídající hodnotu velikosti populace. 

S počátečními úlohami pro diferenciální rovnice je úzce spjata problematika existence a jednoznačnost řešení počáteční úlohy. Ukazuje se že pokud je funkce dostatečně pěkná vzhledem k závislé proměnné potom má úloha právě jedno řešení. V jakém smyslu má být funkce dostatečně pěkná ukazuje například tato existenční věta která říká že existence a jednoznačnost řešení je zaručena ohraničenou parciální derivací pravé strany diferenciální rovnice podle y. Pokud by podmínky této věty nebyly splněny Můžeme o jednoznačnost přijít. Například pokud je funkce pouze spojitá potom je zaručena pouze existence a není zaručena jednoznačnost. 

Protože pro daný systém existuje nekonečně mnoho možností jakou zadat počáteční podmínku , je zřejmé že funkcí,  které vyhovují zadané diferenciální rovnici, existuje nekonečně mnoho. Formule která zahrnuje všechna tato řešení obsahuje nějakou konstantu v roli parametru. Obvykle používáme velké C a taková funkce se nazývá obecné řešení. Například všechny funkce které jsou rovný své vlastní derivaci, tedy všechny funkce vyhovující rovnici 3, jsou tvaru 4 kde C je libovolné reálné číslo. Pokud máme obecné řešení , nebývá zpravidla problém pomocí počáteční podmínky z obecného řešení najít řešení zadané počáteční úlohy. Takové řešení se nazývá partikulární řešení.

\section*{Aplikace}

Newtonův zákon tepelné výměny říká, že rychlost, s jakou se mění teplota tělesa při tepelné výměně s okolím, je úměrná teplotnímu rozdílu. Protože rychlost je vlastně derivace, je Newtonův zákon vlastně vztah mezi derivací tepoty podle času a rozdílu teplot. Jak si pohlídat znaménka tak, abychom například při ochlazování měli správně vyjádřeno, že teplota klesá, jsme si ukázali v kapitole o derivacích a tam jsme si naformulovali model, který vidíte v prezentaci. Obecné řešením zahrnuje všechny možné způsoby, jak se může teplota tělesa o teplotě jiné než teplota okolí ustálit na teplotě okolí. Pokud přidáme i počáteční podmínku a najdeme partikulární řešení, dostaneme funkci, která po zadání času na vstup vrací teplotu v daném čase na výstupu. Ještě zmíníme jednu zajímavou a důležitou věc: volba jednotek teploty je plně v naší moci. Pokud jednotky zvolíme tak, že jsou spojeny s úlohou, můžeme úlohu formálně zjednodušit. V tomto případě, pokud posuneme nulovou hodnotu teploty na teplotu okolí, potom při odečítání teploty okolí odečítáme nulu, tedy vlastně neděláme nic. V těchto jednotkách je tedy rychlost poklesu teploty úměrná teplotě. To je zajímavý a důležitý obrat, který je tím užitečnější, čím složitější a komplexnější je model. Pro model tepelné výměny tento obrat nijak zásadní není, model umíme vyřešit i bez tohoto zjednodušení. Ale je příjemnější se velké myšlenky učit na malých věcech, tak proto zdůrazňujeme obrat s vhodnou volbou jednotek na tomto místě.

Dalším notoricky známým modelem je model radioaktivního rozpadu. Z fyziky je známo, že rychlost radioaktivního rozpadu nestabilní látky je úměrná množství dosud nerozpadnuté látky. To je opět role jako šitá pro derivaci podle času, protože slyšíme slovo rychlost. V tomto případě máme derivaci množství nerozpadnuté látky a ta je úměrná množství. Pokud toto množství označíme y a požadujeme, jak je obvyklé, aby konstanta úměrnosti byla kladná, potom máme s přihlédnutím k tomu, že se jedná o pokles model, který vidíte v prezentaci. Nejčastěji tato problematika bývá zmiňována při problematice datování vzorků. Takto je to asi v každé učebnici diferenciálních rovnic. Mnohem důležitější aplikace je však například modelování rozpadu radioaktivního radonu v budovách, jako prevence před rakovinou plic v oblastech s výskytem přírodního uranu.

Formálně naprosto stejná rovnice jako je rovnice pro radioaktivní rozpad je rovnice popisující vyplavování nečistot z nádrže, do které teče čistá voda a mísí se s nečistotami, které jsou vyplavovány ven. Vzhledem k ekologickým aplikacím se tato rovnice nazývá rovnice samočištění jezer a je založena na tom, že rychlost, klesá množství nečistot je rychlost, s jakou jsou nečistoty vyplavovány. Tato rychlost je úměrná koncentraci a koncentrace je úměrná množství.

Místo vody jezera a nečistot si můžeme představit krevní oběh, krev a krevní tělíska, jako krvinky nebo destičky. Mechanismus je stejný jako u jezera a matematický model tedy také. Zejména, rychlost úbytku nečistot nebo krevních tělísek je malá, pokud je malé i jejich celkové množství. V tom spočívá myšlenka před operací krev pacienta naředit, aby ztráty krvácením při operaci byly sníženy. Praktická realizace této myšlenky je operační metoda nazývaná akutní normovolemická hemodiluce a bez odhadů založených na diferenciálníćh rovnicích by nebylo možno tuto metodu uvést do praxe.

To že rychlost poklesu veličiny je úměrná množství není nic neobvyklého. Naopak, zdá se že to je závislost, kterou má příroda ve značné oblibě. Kromě tepelné výměny formulované pro vhodné jednotky, kromě radioaktivního rozpadu a kromě samočištění jezer ji najdeme například i v RC obvodu, který umožňuje měřit odpor látek s malou vodivostí, jako je například dřevo. Rovnici odvozenou z fyzikálních zákonů vidíte v prezentaci a osamostatněním derivace proudu na levé straně dostáváme rovnici, kdy je derivace proudu úměrná záporně vzatému proudu s konstantou úměrnosti 1/RC.

Diferenciální rovnice mají uplatnění v matematické biologii resp. v populační ekologii při studiu populací. Derivace je rychlost růstu populace. Tato rychlost je úměrná velikosti a zpravidla i volnému místu v daném životním prostředí. Je-li velikost populace y a nosná kapacit prostředí K, je procento obsazeného místa y/K, volného místa je 1-y/K. To že velikost roste rychlostí úměrnou současně volnému místu a současně velikosti  populace vyjadřuje rovnice, kterou vidíte v prezentaci a nazývá se logistická rovnice. Koeficienty se tradičně označují r a K a toto označení dalo pojmenování běžně používané pro různé strategie, se kterými se živočišný druh vyrovnává se změnami v životním prostředí. Populace, které jsou r stratégy se dokáží rychle namnožit v příhodných podmínkách, protože mají vysoký koeficient r. Typicky si můžeme v říši rostlin představit plevel. Oproti tomu K stratégové představují dlouhodoběji žijící organismy, které se dokáží vyrovnávat se změnami životního prostředí lépe než r stratégové, jejich populace jsou stabilnější a nepotřebují vysoké r. Tato rovnice je základní rovnicí pro modelování obnovitelných zdrojů a bývá zvykem ji modifikovat na konkrétní situace. Například pro populaci vystavenou lovu na pravou stranu přidáme člen, který lovem způsobí snížený rychlosti růstu populace. Takový model slouží například k nalezení dlouhodobě udržitelné strategie lovu.

O něco náročnější aplikace diferenciálních rovnic je propočítávání drah meteoritů. To je disciplína, ve které čeští vědci patří k absolutní světové špičce a metoda propočítávání dráhy je založena na řešení diferenciálních rovnic.


\section*{Geometrická interpretace a transformace}

Pro zkoumání vlastností diferenciálních rovnic je nesmírně cenná geometrická představa. Diferenciální rovnice je vlastně vztah pro derivaci neznámé funkce a tedy pro směrnici ke grafu této funkce. Pokud hrubou silou v relativně velké množině bodů tuto směrnici najdeme, můžeme nakreslit kratičké lineární elementy s touto směrnicí a tím dostaneme směrové pole diferenciální rovnice. Jedná se o systém lineárních elementů, které jsou tečné ke grafům řešení, k integrálním křivkám, a dávají dobrou představu o chování řešení.

Další dovednost spojená s numerickým počítáním je takzvaná nondimenzionalizace. Platí, že při vhodné volbě jednotek se matematický model představovaní diferenciální rovnicí může zjednodušit. Už jsme si něco naznačili v úvodu, teď si tyto dovednosti rozšíříme na konkrétním příkladě a poté zobecníme. Ukazuje se, že volba jednotek, která dokáže model zjednodušit, spočívá v tom, že jako výchozí jednotky jsou spojeny parametry systému. Potom měříme například v násobcích počáteční teploty a takto definovaná teplota nemá fyzikální rozměr. Odsud název nondimenzionalizace.

Uvažujme rovnici modelující tepelnou výměnu tělesa o teplotě T s okolím o teplotě T_\infty a počáteční teplotou T_0. Už víme, že pokud teplotní stupnici posuneme tak, aby $T_\infty$ bylo rovno nule, rovnice se zjednoduší na rovnici s přímou úměrnost na pravé straně. Kromě posunu teplotní stupnice můžeme volit i délku dílku. Přeškálujme tedy teplotu tak, aby $T_0$ bylo rovno jedné. Opět z modelu vypadl jeden parametr a model se zjednodušil. Konstanta $k$ udává rychlost, s jakou klesá teplota za jednotku času na počátku, když je teplota rovna jedné. Volba jednotky času ovlivní numerickou hodnotu této rychlosti. Například pokud proces probíhá rychlostí pět stupňů za minutu, je to stejná rychlost jako jeden stupeň za 12 sekund. Pokud bychom tedy jako výchozí jednotku času vzali místo minuty jenom pětinu, 12 sekund, potom se rychlost numericky pětkrát zmenší. Podobně, pokud jednotku měření času zmenšíme $k$-krát, zmenší se $k$-krát i rychlost a tedy i konstanta úměrnosti v rovnici. A protože se konstanta $k$ zmenší $k$-krát, bude z ní jednička.

Vidíme, že model se zjednodušil z tvaru se třemi konstantami na model bez konstant. Abychom tohoto dosáhli, sledujeme místo teploty vlastně teplotní rozdíl proti okolí a měříme ho v procentech teplotního rozdílu na počátku. Jednotku času volíme v násobcích času nutného por srovnání teplot v případě, že by proces probíhal konstantní rychlostí stejnou jako na začátku. V

aplikacích matematik zpravidla uvedený postup odbude formulací jako ``bez újmy na obecnosti položíme $T_\infty$ rovno nule, $T_0=1$ a $k=1$'' nebo ``vhodnou volbou jednotek dosáhneme toho, že $T_\infty=0$ a $T_0=k=1$''. Pokud bychom chtěli být důkladnější, uvědomíme si, jak se derivace chová vzhledem k součtu, násobení konstantou a skládáním funkcí. Tím dostaneme vztahy udávající, jak se chová derivace pokud místo nezávislé a závislé veličiny dosadíme jejich konstantní násobky, což přesně odpovídá změně jednotek. Transformační vzorce máte v prezentaci a za povšimnutí stojí genialita zápisu derivací pomocí podílu diferenciálů, protože transformace formálně připomínají počítání se zlomky. Je to jako bychom konstanty k1 a k2 dali před zlomek stejně, jako se pracuje s konstantními násobky. Využití je například při studiu, jak se systém chová vzhledem ke změně měřítka. Toto bylo obvyklé při studiu proudění tekutin. Například v  roce 1963 došlo v údolí Vajont v italských Alpách k sesuvu hory do přehradní nádrže pro připravovanou vodní elektrárnu. To vyvolalo obrovskou tsunami, která smetla z povrchu několik vesnic, které jí stály v cestě. Sesuv se očekával, ovšem pomalý a s vlnou mnohonásobně menší, kterou by zastavila hráz. Vše bylo ověřeno na zmenšeném modelu. Ten je možné vidět v několika filmech o této katastrofě a zajímavým důsledkem změny rozměrů bylo, že i čas ubíhal v simulaci jinou rychlostí než v reálu. Rychlost sesuvu byla odhadnuta na jednu minutu, čemuž v simulaci měly odpovídat čtyři vteřiny. Tento model byl zdařilý, ovšem problém byl, že nepracoval s realisticky odhadnutými hodnotami. Ve skutečnosti sesuv trval kratší dobu, než byla zmíněná minuta a vinou toho měla vlna výšku 200 metrů namísto očekávaných 20 metrů. Geologický průzkum nakonec objasnil, že oblast není vhodná pro takové vodní dílo. Z jiného soudku můžeme zmínit přípravu koncertní síně pro Brno. Akustika se zkoušela na jaře 2021 na zmenšeném modelu. I v tomto zmenšeném modelu jinak plyne čas a proto bylo v desetkrát menším modelu použít desetkrát vyšší frekvence. To si zase neslo komplikace s chováním vodních par a proto byl model naplněn dusíkem. Na těchto příkladech vidíme, že hodně se s diferenciálními rovnicemi dá dělat, aniž bychom je vůbec řešili. Také to ukazuje užitečnost na první pohled obtížných veličin jako je bezrozměrný čas, bezrozměrná vzdálenost a bezrozměrná teplota, se kterými se setkáte v aplikacích. Tyto veličiny vyjadřují například délku v násobcích délky zkoumaného vzorku materiálu nebo teplotu jako procenta teplotního rozdílu na počátku. 



\section*{Autonomní ODE}

Autonomní rovnice jsou rovníce, kde pravá strana nezávisí na čase. Rychlost růstu je tedy jednozančně dána stavem systému. To je poměrně běžný případ. Například rychlost ochlazování kávy souvisí s teplotním rozdílem nápoej a okolí a nesouvisí s tím, jestli je pondělí nebo pátek.

Protože derivace konstantní funkce je nula a nekonstantní funkce mají derivaci nenulovou, je u autonomních rovnic relativně jednoduché určit konstantní řešení: jsou to nulové body pravé strany rovnice. Tato řešeníse nazývají stacionární body. Představují ustálený stav, do kterého systém časem přejde. Mimo stacionární stav můžeme podle znaménka prvé strany rovnice identifikovat znaménko derivace a tím i monotonii řešení. Odsud poznáme, kde řešení roste a ke klesá. Pokud nás toto informace dovede k tomu, že při vychýlení ze stacionárního stavu se systtém snaží rovnováhu obnovit, mluvíme o stabilním stacionárním řešení. Pro stabilitu je nutné, aby při snížení funkce rostla zpět a pro zvýšení klesala zpět. Tomu odpovídá požadavek, že pravá strana musí být nalevo od stabilního stacionárního bodu kladná a napravo záporná. Tedy pravá strana rovnice být ve stabilním stacionárním bodě klesající, což je možné zaručit podmínou na zápornost derivace.


Uvažujme logistickou rovnici pro vývoj populace s konstantním lovem intenzity $h$. První část rovnice reprezentuje parabolu otočenou vrcholem nahoru. Lov repreyentuje posunutí směrem dolů. Podle intenzity tohoto posunutí se parabola buď částečně nebo celá dostane pod vodorovnou osu. Tam, kde je parabola nad osou populace roste a tam kde pod osou populace klesá. 

