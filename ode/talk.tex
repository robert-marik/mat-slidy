\documentclass[12pt]{article}

\input ../talks.tex

\begin{document}

\section*{Diferenciální rovnice}

V této přednášce se seznámíme s diferenciálními rovnicemi. To není nic jiného, než správný název pro to, čemu jsme dříve říkali matematické modely formulované pomocí derivace. Viděli jsme, že tyto modely jsou v některých případech přirozeným matematickým aparátem pro popis reálně probíhajících dějů v přírodě. V přednášce se seznámíme se základním názvoslovím spojeným s touto problematikou a naučíme se identifikovat některé kvalitativní vlastnosti a hledat analytické řešení. Protože se často setkáváme s modely nezávislými na čase, budeme se problematice těchto modelů věnovat podrobněji. Tyto modely mají tu vlastnost, že se ohraničená řešení po čase ustálí okolo stabilní hodnoty. Naučíme se hledat hodnoty odpovídající ustáleným řešením a z nich vybrat ty stabilní, k nimž systém může konvergovat, nebo nestabilní, které oddělují oblasti, ze kterých systém dospívá k jednotlivým stabilním stavům.

\section*{ODR prvního řádu}

Formálně diferenciální rovnicí rozumíme rovnici tvaru 1 kdy na levé straně je derivace veličiny $y$ podle $x$ a na pravé straně je nějaký výraz který může obsahovat veličiny $x$ a $y$. 

Někdy bývá zvykem derivaci označovat pomocí čárky. Jindy bývá zvykem z derivací pracovat formálně jako s podílem diferenciálů a převést rovnici do tvaru bez zlomku na levé straně. Tady vidíte různé druhy zápisu jedné a té samé diferenciální rovnice.

Diferenciální rovnice je v praxi vlastně scénář který popisuje jak se studovaný systém vyvíjí. Abychom dokázali dávat předpovědi jaké budou hodnoty veličin po uplynutí určitého času, je nutné kromě scénáře vývoje, tedy kromě diferenciální rovnice, mít k dispozici i výchozí stav. Tento výchozí stav se zadává ve formě počáteční podmínky kdy pro daný bod přepisujeme funkční hodnotu. Počáteční podmínka se někdy též nazývá Cauchyova podmínka a úlohu najít řešení diferenciální rovnice které splňuje počáteční podmínku se nazývá počáteční úloha nebo též Cauchyova úloha.

Diferenciální rovnice může například udávat rychlost ochlazování horkého nápoje podle Newtonova zákona. Newtonův zákon ochlazování vyjadřuje to, že rychlost poklesu teploty je úměrná teplotnímu rozdílu. Počáteční podmínkou je například teplota na začátku sledování procesu. Řešením je funkce, která po zadání času vrací jako funkční hodnotu přímo teplotu nápoje v tomto čase. 

Diferenciální rovnice může například udávat rychlost růstu populace živočišného druhu. Počáteční podmínka poté udává velikost populace na začátku sledování. Řešením této rovnice je funkce, která po dosazení jakéhokoliv budoucího nebo minulého času dává odpovídající hodnotu velikosti populace. 

S počátečními úlohami pro diferenciální rovnice je úzce spjata problematika \textit{existence a jednoznačnosti} řešení počáteční úlohy. Ukazuje se že pokud je pravá strana rovnice dostatečně pěkná vzhledem k závislé proměnné, potom má úloha právě jedno řešení. V jakém smyslu má být funkce dostatečně pěkná ukazuje například tato existenční věta. Věta říká, že existence a jednoznačnost řešení je zaručena ohraničenou parciální derivací pravé strany diferenciální rovnice podle $y$. Pokud by podmínky této věty nebyly splněny, můžeme o jednoznačnost přijít. Například pokud je funkce pouze spojitá, potom je zaručena pouze existence a není zaručena jednoznačnost. V dalším textu uvidíme ještě jednu větu o existenci a jednoznačnosti, která je aplikovatelná pouze na některé rovnice, ale její ověření je jednodušší. To bude v kapitolce věnované diferenciálním rovnicím se separovanými proměnnými. 

Protože pro daný systém modelovaný diferenciální rovnicí existuje nekonečně mnoho možností jakou zadat počáteční podmínku, je zřejmé že existuje nekonečně mnoho funkcí, které vyhovují dané diferenciální rovnici. Formule, která zahrnuje všechna tato řešení, obsahuje nějakou konstantu v roli parametru. Taková formule se nazývá obecné řešení a jako parametr obvykle používáme velké $C$. Například všechny funkce které jsou rovný své vlastní derivaci, tedy všechny funkce vyhovující rovnici 3, jsou tvaru 4, kde $C$ je libovolné reálné číslo. Pokud máme obecné řešení, nebývá zpravidla problém pomocí počáteční podmínky z obecného řešení najít řešení zadané počáteční úlohy. Takové řešení se nazývá partikulární řešení.

\section*{Aplikace}

Newtonův zákon tepelné výměny říká, že rychlost, s jakou se mění teplota tělesa při tepelné výměně s okolím, je úměrná teplotnímu rozdílu. Protože rychlost je vlastně derivace, je tento zákon vlastně vztah mezi derivací tepoty podle času a rozdílem teplot. Jak si pohlídat znaménka tak, abychom například při ochlazování měli správně vyjádřeno, že teplota klesá, jsme si ukázali v kapitole o derivacích. Tam jsme si také naformulovali model, který vidíte v prezentaci. Obecné řešení zahrnuje všechny možné způsoby, jak se může teplota tělesa o teplotě jiné než teplota okolí ustálit na teplotě okolí. Pokud přidáme i počáteční podmínku a najdeme partikulární řešení, dostaneme funkci, která po zadání času na vstup vrací teplotu v daném čase na výstupu.

Ještě u tohoto modelu zmíníme jednu zajímavou a důležitou věc: volba jednotek teploty je plně v naší moci. Pokud jednotky zvolíme tak, že jsou spojeny s úlohou, můžeme úlohu formálně zjednodušit. V tomto případě, pokud posuneme nulovou hodnotu teploty na teplotu okolí, potom při odečítání teploty okolí odečítáme nulu, tedy vlastně neděláme nic. V těchto jednotkách je tedy rychlost poklesu teploty úměrná teplotě. To je zajímavý a důležitý obrat, který je tím užitečnější, čím složitější a komplexnější je model. Pro model tepelné výměny tento obrat nijak zásadní není, model umíme vyřešit i bez tohoto zjednodušení. Ale je příjemnější se velké myšlenky učit na malých věcech, tak proto zdůrazňujeme obrat s vhodnou volbou jednotek na tomto místě.

Dalším notoricky známým modelem je model radioaktivního rozpadu. Z fyziky je známo, že rychlost radioaktivního rozpadu nestabilní látky je úměrná množství dosud nerozpadnuté látky. To je opět role jako šitá pro derivaci podle času, protože slyšíme slovo rychlost. V tomto případě máme derivaci množství nerozpadnuté látky a ta je úměrná množství. Pokud toto množství označíme $y$ a požadujeme, jak je obvyklé, aby konstanta úměrnosti byla kladná, potom máme s přihlédnutím k tomu, že se jedná o pokles model, který vidíte v prezentaci. Nejčastěji tato problematika bývá zmiňována při problematice datování vzorků. Takto je to asi v každé učebnici diferenciálních rovnic. Mnohem důležitější aplikace je však například modelování rozpadu radioaktivního radonu v oblastech s výskytem přírodního uranu. Uran se totiž rozpadá a jeden z produktů rozpadu je radon. Radon je radioaktivní prvek, který díky plynnému skupenství snadno proniká do budov a má díky vysoké hustotě tendenci se kumulovat ve sklepních prostorách a v přízemí. Protože se po kouření jedná  nejvýznamnější faktor způsobující rakovinu plic, snažíme se před tímto plynem bránit. A účinná obrana není možná bez pochopení dynamiky vzniku a zániku radonu radioaktivním rozpadem.

Všimněte si, že rovnice pro tepelnou výměnu s okolím o nulové teplotě je naprosto stejná jako rovnice radioaktivního rozpadu. To není nic neobvyklého.

Formálně naprosto stejná rovnice jako je rovnice pro radioaktivní rozpad je i rovnice popisující vyplavování nečistot z nádrže, do které teče čistá voda a mísí se s nečistotami, které jsou vyplavovány ven. Vzhledem k ekologickým aplikacím se tato rovnice nazývá \textit{rovnice samočištění jezer} a je založena na tom, že rychlost s jakou klesá množství nečistot je rychlost, s jakou jsou nečistoty vyplavovány. Tato rychlost je úměrná koncentraci a koncentrace je úměrná množství.

Rovnice samočištění jezer má uplatnění i mimo ekologii. Místo vody, jezera a nečistot si můžeme představit krevní oběh, krev a krevní tělíska, jako krvinky nebo destičky. Mechanismus je stejný jako u jezera a matematický model tedy také. Zejména, rychlost úbytku nečistot nebo krevních tělísek je malá, pokud je malé i jejich celkové množství. V tom spočívá myšlenka před operací krev pacienta naředit, aby ztráty krvácením při operaci byly sníženy. Praktická realizace této myšlenky je operační metoda nazývaná akutní normovolemická hemodiluce a bez odhadů založených na diferenciálních rovnicích by nebylo možno tuto metodu uvést do praxe.

Jak už jsem říkal, to, že rychlost poklesu veličiny je úměrná množství, není nic neobvyklého. Naopak, zdá se že to je závislost, kterou má příroda ve značné oblibě. Kromě tepelné výměny formulované pro vhodné jednotky, kromě radioaktivního rozpadu a kromě samočištění jezer ji najdeme například i v RC obvodu, který umožňuje měřit odpor látek s malou vodivostí, jako je například dřevo. Rovnici odvozenou z fyzikálních zákonů vidíte v prezentaci a osamostatněním derivace proudu na levé straně dostáváme rovnici, kdy je derivace proudu úměrná záporně vzatému proudu s konstantou úměrnosti 1/RC.

Diferenciální rovnice mají uplatnění i v matematické biologii. Nejčastěji je vidíme v populační ekologii při studiu populací. Derivace je rychlost růstu populace. Tato rychlost je úměrná velikosti a zpravidla i volnému místu v daném životním prostředí. Je-li velikost populace $y$ a nosná kapacita prostředí $K$, je procento obsazeného místa $y/K$ a volného místa je $1-y/K$. To že velikost roste rychlostí úměrnou současně volnému místu a současně velikosti  populace vyjadřuje rovnice, kterou vidíte v prezentaci a nazývá se logistická rovnice. Koeficienty se tradičně označují $r$ a $K$ a toto označení dalo pojmenování strategiím, jak se živočišný druh vyrovnává se změnami v životním prostředí. Populace, které jsou $r$ stratégy se dokáží rychle namnožit v příhodných podmínkách, protože mají vysoký koeficient $r$. Typicky si můžeme v říši rostlin představit plevel. Oproti tomu $K$ stratégové představují dlouhodoběji žijící organismy, které se dokáží vyrovnávat se změnami životního prostředí lépe než $r$ stratégové, jejich populace jsou stabilnější a nepotřebují vysoké $r$. Právě představená rovnice je základní rovnicí pro modelování obnovitelných zdrojů a bývá zvykem ji modifikovat na konkrétní situace. Například pro populaci vystavenou lovu na pravou stranu přidáme člen, který lovem způsobí snížený rychlosti růstu populace. Takový model slouží například k nalezení dlouhodobě udržitelné strategie lovu.

O něco náročnější aplikace diferenciálních rovnic je propočítávání drah meteoritů. To je disciplína, ve které čeští vědci patří k absolutní světové špičce a metoda propočítávání dráhy je založena na řešení diferenciálních rovnic.


\section*{Geometrická interpretace a transformace}

Pro zkoumání vlastností diferenciálních rovnic je nesmírně cenná geometrická představa. Diferenciální rovnice je vlastně vztah pro derivaci neznámé funkce a tedy pro směrnici ke grafu této funkce. Pokud hrubou silou v relativně velké množině bodů tuto směrnici najdeme, můžeme nakreslit kratičké lineární elementy s touto směrnicí a tím dostaneme směrové pole diferenciální rovnice. Jedná se o systém lineárních elementů, které jsou tečné ke grafům řešení, k integrálním křivkám, a dávají dobrou představu o chování řešení.

Další dovednost spojená s numerickým počítáním je takzvaná nondimenzionalizace. Platí, že při vhodné volbě jednotek se matematický model představovaný diferenciální rovnicí může zjednodušit. Už jsme si něco naznačili v úvodu v souvislosti s volbou jednotek teploty v Newtonově zákoně ochlazování. Teď si tyto dovednosti rozšíříme a poté zobecníme. Ukazuje se, že volba jednotek, která dokáže model zjednodušit, spočívá v tom, že jako výchozí jednotky jsou použity parametry systému. Potom měříme například v násobcích počáteční teploty a takto definovaná teplota nemá fyzikální rozměr. Odsud název nondimenzionalizace.

Uvažujme rovnici modelující tepelnou výměnu tělesa o teplotě $T$ s okolím o teplotě $T_\infty$ a počáteční teplotou $T_0$. Už víme, že pokud teplotní stupnici posuneme tak, aby $T_\infty$ bylo rovno nule, rovnice se zjednoduší na rovnici s přímou úměrností na pravé straně. Kromě posunu teplotní stupnice můžeme volit i délku dílku. Se zachováním nuly natáhneme tedy délku jendoho dílku na naší teplotní stupnici tak, aby $T_0$ bylo rovno jedné. Opět z modelu vypadl jeden parametr a model se zjednodušil.

Konstanta $k$ udává rychlost, s jakou klesá teplota za jednotku času na počátku, když je teplota rovna jedné. Volba jednotky času ovlivní numerickou hodnotu této rychlosti. Například pokud proces probíhá rychlostí pět stupňů za minutu, je to stejná rychlost jako jeden stupeň za 12 sekund. Pokud bychom tedy jako výchozí jednotku času vzali místo minuty jenom pětinu, 12 sekund, potom se rychlost numericky pětkrát zmenší. Podobně, pokud jednotku měření času zmenšíme $k$-krát, zmenší se $k$-krát i rychlost a tedy i konstanta úměrnosti v rovnici. A protože se konstanta $k$ zmenší $k$-krát, bude z ní jednička.

Vidíme, že model se zjednodušil z tvaru se třemi konstantami na model bez konstant. Abychom tohoto dosáhli, sledujeme místo teploty vlastně teplotní rozdíl proti okolí a měříme ho v procentech teplotního rozdílu na počátku. Jednotku času volíme v násobcích času nutného pro srovnání teplot v případě, že by proces probíhal konstantní rychlostí stejnou jako na začátku. V aplikacích matematik zpravidla uvedený postup odbude formulací jako ``bez újmy na obecnosti položíme $T_\infty$ rovno nule, $T_0=1$ a $k=1$'' nebo ``vhodnou volbou jednotek dosáhneme toho, že $T_\infty=0$ a $T_0=k=1$''. Pokud bychom chtěli být důkladnější, uvědomíme si, jak se derivace chová vzhledem k součtu, násobení konstantou a skládáním funkcí. Tím dostaneme vztahy udávající, jak se chová derivace pokud místo nezávislé a závislé veličiny dosadíme jejich konstantní násobky. A konstantní násobky představují transformaci, která přesně odpovídá změně jednotek. Transformační vzorce máte v prezentaci a za povšimnutí stojí genialita zápisu derivací pomocí podílu diferenciálů, protože transformace formálně připomínají počítání se zlomky. Je to jako bychom konstanty $k_1$ a $k_2$ dali před zlomek stejně, jako se pracuje s konstantními násobky.

Využití zvětšování a zmenšování jednotek je například i při studiu, jak se systém chová vzhledem ke změně měřítka. Toto bylo ještě pře 50 lety obvyklé při studiu proudění tekutin. Například v  roce 1963 došlo v údolí Vajont v italských Alpách k sesuvu hory do přehradní nádrže pro připravovanou vodní elektrárnu. To vyvolalo obrovskou tsunami, která se přelila přes hráz a smetla z povrchu několik vesnic, které jí stály v cestě. Sesuv se očekával, ovšem pomalý a s vlnou mnohonásobně menší, kterou by zastavila hráz. Vše bylo ověřeno na zmenšeném modelu. Ten je možné vidět v několika filmech o této katastrofě a zajímavým důsledkem změny rozměrů bylo, že i čas ubíhal v simulaci jinou rychlostí než v reálu. Doba sesuvu hory do nádrže byla odhadnuta na jednu minutu, čemuž v simulaci měly odpovídat čtyři vteřiny. Tento model byl zdařilý, ovšem problém byl, že nepracoval s realisticky odhadnutými hodnotami. Ve skutečnosti sesuv trval kratší dobu, než byla zmíněná minuta a vinou toho měla vlna výšku 200 metrů namísto očekávaných 20 metrů. Geologický průzkum nakonec objasnil, že oblast není vhodná pro takové vodní dílo a hráz v údolí Vajont, jedna z nejvyšších hrází na světě, sice stojí dodnes, ale na suchu.

Z jiného soudku můžeme zmínit přípravu koncertní síně pro Brno. Akustika se zkoušela na jaře 2021 na zmenšeném modelu. I v tomto zmenšeném modelu jinak plyne čas a proto bylo nutno v desetkrát menším modelu použít desetkrát vyšší frekvence. To si zase neslo komplikace s chováním vodních par a proto byl model naplněn dusíkem. Na těchto příkladech vidíme, že hodně se s diferenciálními rovnicemi dá dělat, aniž bychom je vůbec řešili. Také to ukazuje užitečnost na první pohled obtížných veličin jako je bezrozměrný čas, bezrozměrná vzdálenost a bezrozměrná teplota, se kterými se setkáte v aplikacích. Tyto veličiny vyjadřují například délku v násobcích délky zkoumaného vzorku materiálu nebo teplotu měřenou procentem teplotního rozdílu na počátku. 



\section*{Autonomní ODE}

Autonomní rovnice jsou rovnice, kde pravá strana nezávisí na čase. Rychlost růstu je tedy jednoznačně dána stavem systému. To je poměrně běžný případ. Například rychlost ochlazování kávy souvisí s teplotním rozdílem nápoje a okolí a nesouvisí s tím, jestli je pondělí nebo pátek.

Protože derivace konstantní funkce je nula a nekonstantní funkce mají derivaci nenulovou, je u autonomních rovnic relativně jednoduché určit konstantní řešení: jsou to nulové body pravé strany rovnice. Tato řešení se nazývají stacionární body. Představují ustálený stav, do kterého systém časem přejde. Mimo stacionární stav můžeme podle znaménka prvé strany rovnice identifikovat znaménko derivace a tím i monotonii řešení. Odsud poznáme, kde řešení roste a ke klesá. Pokud nás toto informace dovede k tomu, že při vychýlení ze stacionárního stavu se systém snaží rovnováhu obnovit, mluvíme o stabilním stacionárním řešení. Pro stabilitu je nutné, aby při snížení funkce rostla zpět a pro zvýšení klesala zpět. Tomu odpovídá požadavek, že pravá strana musí být nalevo od stabilního stacionárního bodu kladná a napravo záporná. Tedy pravá strana rovnice být ve stabilním stacionárním bodě klesající, což je možné zaručit podmínkou na zápornost derivace.

Uvažujme logistickou rovnici pro vývoj populace s konstantním lovem intenzity $h$. První část rovnice reprezentuje parabolu otočenou vrcholem nahoru. Lov reprezentuje posunutí směrem dolů. Podle intenzity tohoto posunutí se parabola buď částečně nebo celá dostane pod vodorovnou osu. Tam, kde je parabola nad osou populace roste a tam kde pod osou populace klesá. Na vodorovné ose je velikost populace a růst je doprava, pokles doleva.

Pokud je parabola celá pod vodorovnou osou, není žádný průsečík funkce z pravé strana a vodorovné osy. Není proto žádný stacionární bod, derivace populace je stále záporná a velikost populace stále klesá. Populace nevyhnutelně s takovou intenzitou lovu časem vyhyne.

Pokud je lov takový, že vrchol paraboly je nad vodorovnou osou, vidíme dva nulové body funkce definující pravou stranu a tedy dva stacionární body diferenciální rovnice. Levý je nestabilní. Teoreticky odpovídá konstantnímu řešení, ale musíme si uvědomit, že v reálu neustále dochází k výkyvům dolů nebo nahoru. A protože nalevo od tohoto nulového bodu dochází k poklesu a napravo k nárůstu, jakákoliv náhodná perturbace se začne zvětšovat a populace buď vymře, nebo její velikost začne růst a doroste k dalšímu stacionárnímu bodu. Levý stacionární bod tedy představuje jakousi kritickou hranici: pod ní populace vymře, není schopna reprodukcí dorovnávat ztráty z lovu. Nad kritickou hranicí přežije a dokonce její velikost roste až k dalšímu stacionárnímu bodu. Ten je stabilní, což pěkně charakterizují šipky. Vidíme, že náhodná relativně malá změna nás vrací zpět. Nad tímto stacionárním bodem se projeví již silně mezidruhová konkurence, která populaci vrací zpět do stacionárního bodu. Takový stacionární bod odpovídá stavu, do kterého systém přejede po uplynutí dostatečně dlouhého času. Vzdálenost mezi stabilním a nestabilním bodem je jakýmsi měřítkem odolnosti vůči náhodným změnám. Pokud změny budou relativně malé a nezmění se náhlým výkyvem hodnoty populace pod kritickou hodnotu, nasazená intenzita lovu je v pořádku a je trvale udržitelná. 

V modelu jako byl tento často proti sobě hrají dva faktory, jeden podporuje a druhý snižuje hodnotu studované veličiny. Proto bývá pravá strana ve tvaru rozdílu. Někdy bývá vhodné nesrovnávat rozdíl funkcí s nulou, ale ekvivalentně srovnávat obě funkce z rozdílu mezi sebou. 

Například teplotní bilanci Země je možno modelovat rovnicí udávající, jak se dopadající záření a vyzářená energie projeví na změně teploty. Oba faktory závisí na teplotě, protože efekt dopadajícího záření souvisí s odrazivostí, ta souvisí s podílem ledu a vody a ten souvisí s teplotou. Že i vyzařování energie souvisí s teplotou je dobře známo fyzikům jak Stefanův Bolzmanův vyzařovací zákon. Křivky mohou vypadat například jako na obrázku. Červeně je ohřev, modře ochlazování. Pro malé teploty je dominantní ohřev, rozdíl na pravé straně ohřev minus ochlazování je kladný, teplota roste a Země nemůže zmrznout na nulu. Pro vysoké teploty je dominantní vyzařování, pravá strana rovnice je záporná protože od menšího výrazu odečítáme vyšší a teplota klesá. To je opět dobrá zpráva, Země se nemůže přehřát a shořet. Vidíme tři průsečíky, v nich je rozdíl pravé strany roven nule a odpovídají stacionárnímu stavu. Pokud uvážíme znaménko rozdílu ohřev minus ochlazování a z toho vyplývající monotonii pro teplotu, vidíme, že prostřední stacionární bod je nestabilní a separuje oblasti přitažlivosti dvou stabilních stacionárních bodů 


Sami si rozmyslete, jak byste argumentovali u logistické rovnice s lovem konstantní intenzity, kdy růst je dán logistickou rovnicí, tedy máme parabolu otočenu vrcholem nahoru, a lov je konstantní intenzity, tedy máme vodorovnou přímku. Jsou kvalitativně dva různé případy vzájemného vztahu mezi přímkou a parabolou a buď jsou nebo nejsou průsečíky. Závěr by měl být stejný jako v případě, kdy jsme pravou stranu brali jako celek. Trvalou udržitelnost máme pro malou intenzitu lovu a dostatečně velkou velikost populace.

V poznámce máte též tyto poznatky shrnuty obecně, ale pravděpodobně je lepší se spolehnout na znalost souvislosti znaménka derivace s monotonií, než se soustředit co je v daném konkrétním případě funkce $g$, co je funkce $h$ a podobně.

Jestli jste problematice porozuměli si můžete vyzkoušet v připravených testových úkolech a poté tyto znalosti použít v domácích úlohách. 


\section*{Separovatelné DR}

Nejjednodušší diferenciální rovnice jsou rovnice se separovanými proměnnými. Tyto rovnice umíme vyřešit v tom smyslu, že dokážeme zapsat řešení pomocí integrálů funkcí vystupujících v rovnici. Pokud tyto integrály dokážeme i vypočítat, nalezneme analytické řešení rovnice, tj. řešení rovnice vyjádřené funkčním předpisem používajícím v našem přiblížení základní elementární funkce. Separovatelné diferenciální rovnice jsou rovnice, kde pokud hledáme $y$ jako funkci $x$, je pravá strana součinem funkce nezávislé na $y$ a funkce nezávislé na $x$. V tomto smyslu říkáme, že proměnné jsou odseparovány. Pro konstantní funkci $f$ vidíme, že zde jsou zahrnuty i autonomní rovnice a tedy vše co řekneme o separovatelných DR je aplikovatelné i na DR autonomní.

To že pro tyto rovnice máme speciální název napovídá, že ne každá rovnice musí být separovatelná. Pokud však je, její řešení je formálně snadné. Samostatně nalezneme konstantní řešení jako nulové body komponenty na pravé straně závislé na $y$. To je stejné jako u autonomních diferenciálních rovnic, kterým jsme se věnovali před chvílí.

Pro praktické modelování je zajímavější pracovat s nekonstantními řešeními, protože ty znamenají, že se v systému něco děje. Ani nekonstantní řešení však není těžké najít. Derivaci zapíšeme jako podíl diferenciálů a na chvíli dostaneme licenci, nebo povolenku, pracovat s tuto derivací jako s klasickým podílem. Pomocí násobení a dělení oddělíme proměnné a potom stačí každou stranu zintegrovat podle proměnné, která na této straně figuruje. Na jednu stranu napíšeme integrační konstantu a tím máme nalezeno obecné řešení. Pokud potřebujeme řešení partikulární, které splňuje počáteční podmínku, stačí tuto počáteční podmínku dosadit do obecného řešení a učíme vhodnou hodnotu konstanty z obecného řešení, pro kterou je tato počáteční podmínka splněna. Příslušnou hodnotu potom použijeme v obecném řešením.

Detailněji si toto vyzkoušíte ve cvičení a my se zaměříme ještě na problematiku jednoznačnosti. Pochopitelně platí to co platí pro obecné rovnice, že pokud má pravá strana ohraničenou parciální derivaci podle druhé proměnné, potom má úloha právě jedno řešení. Pro diferenciální rovnice se separovanými proměnnými však existuje i jednodušší kriterium: pokud je pravá funkce závislá na $y$ nenulová, má počáteční úloha právě jedno řešení. Evidentně takto nemůžeme posoudit jednoznačnost konstantního řešení, ale ve všech ostatních případech je toto kriterium většinou jednodušší, protože nepotřebuje derivaci.

Požadavek jednoznačnosti se zdá být přirozený v aplikacích, přesto se však někdy setkáváme i s nejednoznačností. Například při růstu vodní kapky tato kapka roste tak, že objem se zvětšuje rychlostí úměrnou povrchu díky tomu, že kondenzace páry na vodu probíhá na povrchu kapky. Odsud můžeme sestavit příslušnou diferenciální rovnici, přičemž využijeme toho, že pro kulovou kapku je povrch koule úměrný vhodné mocnině objemu. To jsme si říkali ve třetím týdnu jako jeden z důsledků Buckinghamova pi teorému. Máme tedy následující diferenciální rovnici. Dokonce je to autonomní rovnice. Pokud kapka právě začíná růst, je výchozí hodnota objemu nula. Pravá strana nula a má neohraničenou derivaci v nule, protože derivováním se exponent zmenší o jedničku, což je v tomto případě $-\frac 13$ a nulu nemůžeme umocnit na záporné číslo. Tedy pravá strana není ani nenulová a ani nemá ohraničenou derivaci. Ani jedna z našich vět o jednoznačnosti není aplikovatelná, protože nejsou splněny její předpoklady. Dokonce je možné najít dvě různá řešení počáteční úlohy začínající v nule. Jedno dostaneme separací proměnných a jedno je konstantní nulové řešení. Tato nejednoznačnost má fyzikální interpretaci při změnách skupenství například v podobě přechlazené páry. Za určitých podmínek pára nekondenzuje na kapalinu, pokud v ní nejsou kondenzační tělíska, kde by kondenzace nastartovala. Souvislost znečištění ovzduší s kondenzací znají době například v Londýně, kde se při topení uhlím v lokálních topeništích produkovalo mnoho popílku, jehož vinou voda v ovzduší ochotněji kondenzovala a vytvářela pověstnou londýnskou mlhu.


\section*{Diferenciální rovnice vyšších řádů a metoda konečných diferencí}

Na závěr povídání o diferenciálních rovnicích dvě kratičké poznámky, přičemž druhá z nich bude sloužit i jako motivace pro další přednášky.

V mechanice hraje ústřední roli druhá derivace polohy, protože druhá derivace polohy je fyzikálně zrychlení a to přímo souvisí s působící silou podle druhého Newtonova pohybového zákona. Často síla sama souvisí s polohou a potom je matematická formulace Newtonova zákona rovnicí, ve které figuruje dokonce druhá derivace neznámé funkce. Takové rovnice se nazývají diferenciální rovnice druhého řádu a z úhlu pohledu matematika se vlastně téměř celá mechanika redukuje na řešení diferenciálních rovnic druhého řádu. Můžete se s těmito rovnicemi setkat snadno při studiu nosníků, při studiu namáhání stromů vlivem větru nebo v souvislosti s problematikou kmitání konstrukcí. 

Například úloha při namáhání nosníku na vzpěr je zmíněna jako úloha na diferenciální rovnici druhého řádu i v Požgajově monografii a pokud například uvažujeme rovnici ve tvaru jako je na obrazovce a namísto druhé derivace dosadíme její přibližné vyjádření pomocí konečné diference, jak jsme odvodili ve druhé přednášce, dostaneme namísto rovnice obsahující druhou derivaci rovnici obsahující funkční hodnoty hledané funkce ve studovaném bodě a v bodech o kousíček doprava a o kousíček doleva. Pokud toto provedeme pro všechny body dostatečně nahusto rozeseté podél nosníku, máme celkem rozumnou aproximaci diferenciální rovnice a vůbec nepotřebujeme derivaci. Jedná se jenom o lineární rovnice. To zní dobře, ale trošku zbrzdím nadšení: tímto postupem snadno skončíte u soustavy obsahující desetitisíce rovnice a desetitisíce neznámých. To není něco, co by bylo zvládnutelné prostředky, které máme teď k dispozici.

Nikdy vás nikdo nebude nutit soustavu rovnic s řádově desetitisíci neznámými řešit, toho se bát nemusíte. Ale je nutné znát vyjadřovací jazyk, kterým se dají takové obrovské soustavy popsat. Je nutné znát vyjadřovací prostředky s touto problematikou spojené. To je obsahem další oblasti matematiky, lineární algebry. Této problematice se budeme věnovat od příští přednášky. Získáme nástroj na efektivní formulaci soustav s libovolným počtem neznámých, ale také aparát umožňující zohlednit skutečnost, že materiál ve kterém nás zajímají hodnoty fyzikálních veličin má v různých směrech různé vlastnosti. Toto je výrazné u dřeva, ale pozorujeme to například i u půdy. 

\end{document}
